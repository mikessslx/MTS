\section{Notes 9 - 11.10}

\subsection{Closure and Limit Points}

\begin{recap}[Closure]
Let \(A \subset X\). We defined \(\overline{A} = \bigcap C_\alpha\) where \(C_\alpha \supset A\) are closed sets.
\end{recap}

\begin{theorem}
\begin{enumerate}
\item[(a)] \(x \in \overline{A} \iff\) every open set \(U \subset X\) such that \(x \in U\) intersects \(A\) (that is, \(U \cap A \neq \emptyset\)).
\item[(b)] For a basis \(\mathcal{B}\) of \(\mathcal{T}\), we have \(x \in \overline{A} \iff\) for every \(B \in \mathcal{B}\) with \(x \in B\), we have
\[B \cap A \neq \emptyset.\]
\end{enumerate}
\end{theorem}

\begin{remark}
Proof is commented out!
\end{remark}

% \begin{proof}
% (a) is same as: \(x \notin \overline{A} \iff \exists U\) open, \(x \in U\) such that \(U \cap A = \emptyset\).
% \begin{itemize}
% \item[\(\Rightarrow\)]
% Take \(x \notin \overline{A}\), then \(U = X \setminus \overline{A}\) is open (since \(\overline{A}\) is closed) and contains \(x\).

% Since \(\overline{A} \supset A\), \(U \cap A = \emptyset\).

% \item[\(\Leftarrow\)]
% Take \(U \ni x\), \(U \cap A = \emptyset\), \(U\) open. Then \(X \setminus U\) is closed and \(X \setminus U \supset A\).

% Since \(\overline{A}\) is the intersection of all closed sets containing \(A\), \(X \setminus U \supset \overline{A}\) and \(x \notin \overline{A}\).
% \end{itemize}

% (b) Exercise.
% \end{proof}

\begin{definition}[Limit Points]
Another way to describe closures.
For \(A \subset X\), \(x \in X\) is a \textbf{limit point} of \(A\) if every open set \(U \ni x\) also contains some \(y \in A\) with \(y \neq x\).
Define the set of limit points by \(A'\).
\end{definition}

\begin{theorem}
\(\overline{A} = A \cup A'\).
\end{theorem}

\begin{remark}
Proof is commented out!
\end{remark}

% \begin{proof}
% \begin{itemize}
% \item If \(x \in A'\), every neighborhood of \(x\) intersects \(A\). Then \(x \in \overline{A}\) by the previous theorem. Thus, \(A' \subset \overline{A}\), \(A \subset \overline{A} \implies A \cup A' \subset \overline{A}\).
% \item Reverse: take \(x \in \overline{A}\). Then \(x \in A \subset A \cup A'\). Otherwise, if \(x \notin A\), every open neighborhood of \(x\) intersects \(A\) (at some \(y\)). Since \(x \notin A\), \(y \neq x\), then \(x \in A'\).
% \end{itemize}
% \end{proof}

\begin{corollary}
\(A\) is closed \(\iff A\) contains all its limit points (\(A' \subset A\)).
\end{corollary}

\subsection{Convergence of Sequences}

\begin{definition}[Convergence of Sequences]
For a sequence \(x_n = (x_1, x_2, \cdots)\), \(x_i \in X\), \(a \in X\) is a \textbf{limit} of \(x_n\) (\(x_n \stackrel{n \to \infty}{\longrightarrow} a\) or \(a = \lim_{n \to \infty} x_n\)) if for every open \(U \ni a\), there exists \(N \in \mathbb{N}\) such that \(x_n \in U\) for all \(n > N\).
\end{definition}

\begin{remark}
This is equivalent to: every neighborhood of \(a\) contains \textbf{ALMOST} all elements of \(x_n\) (all except finitely many).
\end{remark}

\begin{example}[Examples of Convergence]
\leavevmode
\begin{enumerate}
\item[\bfseries Ex 1.] \textbf{Discrete Topology}
\(X\) arbitrary with discrete topology.
\(x_n \to a \iff x_n = a\) for all \(n > N\) (\(n \gg 0\), sufficiently large).

\item[\bfseries Ex 2.] \textbf{Rationals}
\(X = \mathbb{Q}\), topology defined by usual metric.
Sequence \(3, 3.1, 3.14, 3.141, \cdots\) (approximating \(\pi\)). \textbf{DOES NOT converge!} (since \(\pi \notin \mathbb{Q}\)).

\item[\bfseries Ex 3.] \textbf{Lower Limit Topology (\(\mathbb{R}_\ell\))}
\(\mathbb{R}_\ell\) is \(\mathbb{R}\) with lower limit topology: \([a, b)\) are basis open sets. What sequences \(x_n \in \mathbb{R}\) are convergent?

A sequence converges in \(\mathbb{R}_\ell \iff\) it converges in \(\mathbb{R}\) \textbf{AND} satisfies an extra condition:
\[x_n \ge a \ \text{for } n \gg 0\]
Example: \(\frac{(-1)^n}{n}\) is not convergent to 0 in \(\mathbb{R}_\ell\).
For every \(\varepsilon\), \([0, \varepsilon)\) does not satisfy the condition for limits (negative terms are outside).
\end{enumerate}
\end{example}

\subsection{Comparing Topologies and Convergence}

Let \(\mathcal{T}, \mathcal{T}'\) be topologies on \(X\).
\(\mathcal{T}'\) is \textbf{finer} than \(\mathcal{T}\) if \(\mathcal{T}' \supset \mathcal{T}\) (\(\mathcal{T}\) is coarser than \(\mathcal{T}'\)).

\begin{proposition}[Convergence in Finer Topology]
\(x_n\) converges in \(\mathcal{T}'\) (finer) \(\implies x_n\) converges in \(\mathcal{T}\) (coarser).
\end{proposition}

(Since open sets in \(\mathcal{T}\) are also open in \(\mathcal{T}'\), the condition is easier to satisfy in \(\mathcal{T}\)).

\begin{example}[\(\mathbb{R}_\ell\) vs \(\mathbb{R}\)]
We know \(\mathbb{R}_\ell\) is finer than \(\mathbb{R}\).
\begin{itemize}
\item \(\mathbb{R}\) is \textbf{NOT} finer than \(\mathbb{R}_\ell\):
    
For \([a, b) \ni x\) (basis in \(\mathbb{R}_\ell\)), does there exist \((a', b') \ni x\) such that \((a', b') \subset [a, b)\)?

If \(x=a\), then \((a', b')\) must contain points less than \(a\), which are not in \([a, b)\). Impossible.
    
\item \(\mathbb{R}_\ell\) \textbf{IS} finer than \(\mathbb{R}\): For any \((a, b) \ni x\), \([x, b)\) is a basis element in \(\mathbb{R}_\ell\) and \([x, b) \subset (a, b)\).
\end{itemize}

Implication: Convergence in \(\mathbb{R}_\ell \implies\) Convergence in \(\mathbb{R}\).

(As seen in Ex 3, convergence in \(\mathbb{R}_\ell\) is strictly stronger).
\end{example}

\subsection{More Limit Examples}

\begin{example}[More Limit Examples]
\leavevmode
\begin{enumerate}
\item[\bfseries Ex 4.] \textbf{Non-Hausdorff Space (\(X = \mathbb{N}\))}
Let \(U_m = \{0, 1, 2, \cdots, m\}\).
Topology \(\mathcal{T} = \{ U_m \mid m \in \mathbb{N} \} \cup \{ \mathbb{N}, \emptyset \}\).
Convergence \(x_n \to a\):
\begin{align*}
&\iff \forall U_m, m \ge a, x_n \in U_m \text{ for } n \gg 0 ;\\
&\iff \forall m \ge a: x_n \le m \text{ for } n \gg 0.
\end{align*}

Equivalently: \(x_n \le a\) for almost all \(n\). Thus, \(a\) is an \textbf{essential upper bound} of \(x_n\).
\begin{itemize}
\item Convergent \(\iff\) Bounded.
\item \textbf{Limit is not unique!} If \(x_n \to a\), then \(x_n \to b\) for any \(b \ge a\).
\end{itemize}

\item[\bfseries Ex 5.] \textbf{Finite Complement Topology on \(\mathbb{R}\)}
\(\mathcal{T} = \{ \emptyset, \mathbb{R} \} \cup \{ \mathbb{R} \setminus \{p_1, \cdots, p_k\} \}\).
\begin{enumerate}
\item
If \(x_n = a\) for \(n \gg 0\): \(x_n \to a\).

\item
If \(x_n\) assumes two values infinitely many times (e.g., \(x_n = (-1)^n\)): \textbf{NO LIMIT}.

(Any open set excluding one value will fail to contain the tail).
            
\item
If \(x_n\) assumes every value finitely many times (all terms distinct): \(x_n \to a, \forall a \in \mathbb{R}\).

(Proof: Take any \(a\). Any open \(U \ni a\) is \(\mathbb{R} \setminus F\) where \(F\) is finite. Since \(x_n\) takes values in \(F\) only finitely many times, eventually \(x_n \notin F\), so \(x_n \in U\).)
\end{enumerate}
\end{enumerate}
\end{example}
