\section{Notes 16 - 12.08}

\subsection{Norms on \(\mathbb{R}^n\)}

\begin{definition}[Norm]
A \textbf{norm} on \(\mathbb{R}^n\) is a function \(\| \cdot \|: \mathbb{R}^n \to \mathbb{R}\) satisfying
\begin{enumerate}
    \item \(\|x\| \ge 0\), and \(\|x\| = 0 \iff x = 0\).
    \item \(\|x + y\| \le \|x\| + \|y\|\) (Triangle Inequality).
    \item \(\|\lambda x\| = |\lambda| \cdot \|x\|\) for all \(x \in \mathbb{R}^n, \lambda \in \mathbb{R}\).
\end{enumerate}
\end{definition}
\begin{note}
Norm \(\to\) Metric \(\to\) Topology.
\end{note}

\begin{example}
    \leavevmode
    \begin{itemize}
        \item \(\|x\|_1 = \sum |x_i|\)
        \item \(\|x\|_2 = (\sum x_i^2)^{1/2}\) (Euclidean)
        \item \(\|x\|_\infty = \max |x_i|\)
    \end{itemize}
\end{example}

\begin{definition}[Equivalence of Norms]
Two norms \(\| \cdot \|_a, \| \cdot \|_b\) are \textbf{equivalent} if \(\exists m, M \in \mathbb{R}_{>0}\) such that
\[m \|x\|_a \le \|x\|_b \le M \|x\|_a\]
This implies \(U_\varepsilon^{(b)}(0) \subseteq U_{M \varepsilon}^{(a)}(0) \subseteq U_{(M/m) \varepsilon}^{(b)}(0)\) (Topology is the same).
\end{definition}

\begin{proposition}
Equivalence of norms is an equivalence relation.
\end{proposition}

\begin{theorem}
All norms on \(\mathbb{R}^n\) are equivalent.
\end{theorem}

\begin{remark}
Proof is commented out!
\end{remark}

% \begin{proof}
% Given \(\| \cdot \|_a, \| \cdot \|_b\). Let \(\| \cdot \|_b\) be the Euclidean norm \(\| \cdot \|_2\).

% Define a function \(f: \mathbb{R}^n \setminus \{0\} \to \mathbb{R}_{>0}\), \(f(x) = \frac{\|x\|_a}{\|x\|_b}\), which is a continuous function on \(\mathbb{R}^n \setminus \{0\}\).

% \(f(x) = f(\lambda x)\) for any \(\lambda \neq 0\) (homogeneity cancels out).

% So \(f(x)\) is completely determined by \(f|_{S^{n-1}}\), where \(S^{n-1} = \{ x \mid \|x\|_b = 1 \}\).

% (\(S^{n-1}\) is compact (closed and bounded in Euclidean metric).)

% Then \(f(x)\) has maximum and minimum values on \(S^{n-1}\). Let \(m = \min_{S^{n-1}} f\), \(M = \max_{S^{n-1}} f\).

% Since \(\|x\|_a > 0\) for \(x \neq 0\), \(m > 0\), then \(0 < m \le \frac{\|x\|_a}{\|x\|_b} \le M < \infty\). Thus, \(\| \cdot \|_a \sim \| \cdot \|_b\).
% \end{proof}

\begin{remark}
In infinite-dimensional spaces, norms are \textbf{not} always equivalent. \(S^\infty\) is \textbf{not compact} (closed and bounded, but not compact). Example: \(\ell_1 = \{ (x_k) \mid \sum |x_k| < \infty \}\).
\end{remark}

\subsection{Uniform Continuity}

\begin{remark}[Goal]
Prove that any continuous map between metric spaces \((X, d_X) \to (Y, d_Y)\), with \(X\) \textbf{compact}, is \textbf{uniformly continuous}.
\end{remark}

\begin{lemma}[The Lebesgue Number Lemma]
Let \((X, d)\) be a metric compact space and \(\mathcal{A}\) be an open covering of \(X\).

Then \(\exists \delta > 0\) (called a \textbf{Lebesgue number}) such that for any subset \(Y \subset X\) with \(\text{diam}(Y) < \delta\), we have \(Y \subset A\) for some \(A \in \mathcal{A}\). (\(\text{diam}(Y) = \sup \{ d(x, y) \mid x, y \in Y \}\)).
\end{lemma}

\begin{remark}
Proof is commented out!
\end{remark}

% \begin{proof}
% If \(X \in \mathcal{A}\), nothing to prove (any \(\delta\) works).

% Otherwise: take a finite subcovering \(\{A_1, \cdots, A_n\} \subset \mathcal{A}\).

% Let \(C_i = X \setminus A_i\), closed subsets in compact \(X\), and hence compact.

% For any \(x \in X\), \(x \in A_i\) for some \(i \implies x \notin C_i \implies d(x, C_i) \ge 0\).

% Define \(f(x) = \frac{1}{n} \sum_{i=1}^n d(x, C_i)\) (average distance). Show that \(f(x) > 0\):

% \(\forall x \in X\), choose \(i\) such that \(A_i \ni x\). Choose \(\varepsilon\) so that \(U_\varepsilon(x) \subset A_i, d(x, C_i) \ge \varepsilon\). \(f(x) \ge \frac{\varepsilon}{n}\).

% Let \(\delta = \min f(x) > 0\), then \(\delta\) is a Lebesgue number:

% Take \(B \subset X, \text{diam}B < \delta, x_0 \in B\). \(B \subset U_\delta(x_0)\).

% \(\delta \le f(x_0) \le d(x_0, C_m)\) (take \(\max d(x_0, C_i)\)) for some \(1 \le m \le n\).

% \(U_\delta(x_0) \subset A_m = X \setminus C_m\), so \(B \subset A_m\).
% \end{proof}

\begin{definition}[Uniform Continuity]
A function \(f: (X, d_X) \to (Y, d_Y)\) is \textbf{uniformly continuous} if
\(\forall \varepsilon > 0, \exists \delta > 0\) such that for each \(x, x' \in X\),
with \(d_X(x, x') < \delta\), we have \(d_Y(f(x), f(x')) < \varepsilon\).
\end{definition}

\begin{theorem}
Let \(f: X \to Y\) be a continuous map of metric spaces.
If \(X\) is \textbf{compact}, then \(f\) is uniformly continuous.
\end{theorem}

\begin{remark}
Proof is commented out!
\end{remark}

% \begin{proof}
% Take \(\varepsilon > 0\). Consider the covering of \(Y\) by open balls \(\{ U_{\varepsilon / 2}(y) \}_{y \in Y}\). Take \(\mathcal{A}\) to be the collection of preimages: \(\mathcal{A} = \{ f^{-1}(U_{\varepsilon / 2}(y)) \mid y \in Y \}\).

% This is an open covering of \(X\). Let \(\delta\) be a \textbf{Lebesgue number} of this covering \(\mathcal{A}\). Then if \(x, x' \in X\) and \(d_X(x, x') < \delta\), then the set \(\{x, x'\}\) has diameter \(< \delta\) and \(\{x, x'\} \subset f^{-1}(U_{\varepsilon / 2}(y))\) for some \(y \in Y \implies f(x), f(x') \in U_{\varepsilon / 2}(y) \implies d_Y(f(x), y) < \varepsilon / 2\) and \(d_Y(f(x'), y) < \varepsilon / 2\).

% By triangle inequality, \(d_Y(f(x), f(x')) < \varepsilon\).
% \end{proof}

\subsection{Cardinality of Compact Spaces}

\begin{theorem}[Cardinality of Compact Sets]
Let \(X\) be a non-empty \textbf{compact Hausdorff} space \textbf{without isolated points} (perfect space).
Then \(X\) is \textbf{uncountable}.
\end{theorem}

\begin{remark}
Proof is commented out!
\end{remark}

% \begin{proof}
% \textbf{Step 1}: Given \(x \in U\) open, show that \(\exists\) open \(V \subset U\), \(V \neq \emptyset\), such that \(\overline{V} \subset U\) and \(x \notin \overline{V}\).

% Choose \(y \in U, y \neq x\). Let \(W_1, W_2\) be disjoint open neighborhoods of \(x\) and \(y\) and \(V = W_2 \cap U\).

% Then \(V\) is open and non-empty. Thus, \(x \notin \overline{V}\) (since \(W_1 \cap V = \varnothing\)).

% \textbf{Step 2}: Given \(f: \mathbb{N} \to X\). We show \(f\) is not surjective.

% Apply Step 1 to \(x_0 = f(0), U = X\) and choose \(V_0\) open, \(\overline{V_0} \subset X, f(0) \notin \overline{V_0}\).

% Choose \(V_1 \subset V_0\) such that \(\overline{V_1} \subset V_0\) and \(f(1) \notin \overline{V_1}\).

% Inductively, \(\overline{V_0} \supset \overline{V_1} \supset \overline{V_2} \supset \cdots\).

% This is a nested sequence of non-empty closed sets in a compact set \(X \implies \bigcap_{n \in \mathbb{N}} \overline{V_n} \neq \emptyset\).

% Let \(x \in \bigcap \overline{V_n}\), then \(x \neq f(n)\) for any \(n\) (since \(x \in \overline{V_n} \subset V_{n-1}\) and \(f(n) \notin \overline{V_n}\)).

% Thus, \(f\) is not surjective.
% \end{proof}

\begin{corollary}
\([a, b] \subset \mathbb{R}\) is uncountable.
\end{corollary}

\begin{note}[One-Point Compactification (Preview)]
Idea: Given locally compact \(X\), construct \(Y = X \cup \{\infty\}\).
Topology:
\begin{enumerate}
    \item Open in \(X \implies\) Open in \(Y\).
    \item \(Y \setminus C\) where \(C \subset X\) compact \(\implies\) Open in \(Y\).
\end{enumerate}
Need to check topology axioms. \(Y\) is Compact Hausdorff. (Next Time.)
\end{note}
