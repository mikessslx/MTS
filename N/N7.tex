\section{Notes 7 - 11.03}

\subsection{Basis of Topology}

\begin{definition}[Basis of Topology]
Let \(X\) be a set. A collection \(\mathcal{B} \subset 2^X\) is a basis for a topology on \(X\) if
\begin{enumerate}
\item \(\forall x \in X\), \(\exists B \in \mathcal{B}\) such that \(x \in B\).
\item If \(x \in B_1 \cap B_2\) with \(B_1, B_2 \in \mathcal{B}\), then \(\exists B_3 \in \mathcal{B}\) such that \(x \in B_3 \subset B_1 \cap B_2\).
\end{enumerate}

Then \(\mathcal{B}\) is a \textbf{basis of topology} on \(X\). The topology generated by \(\mathcal{B}\) is the collection \(\mathcal{U}\) of all unions of elements from \(\mathcal{B}\) (including \(\emptyset\)).
\end{definition}

\begin{proposition}
This collection \(\mathcal{U}\) is a topology on \(X\).
\end{proposition}

\begin{remark}
Proof is commented out!
\end{remark}

% \begin{proof}
% \begin{enumerate}
% \item \(\emptyset \in \mathcal{U}\); \(X \in \mathcal{U}\) (by property 1, \(X = \bigcup_{B \in \mathcal{B}} B\)).
% \item \textbf{Union}: Obvious (union of unions is a union).
% \item \textbf{Intersection}: Need to show \(U_1, U_2 \in \mathcal{U} \implies U_1 \cap U_2 \in \mathcal{U}\). Let \(x \in U_1 \cap U_2\).
% \begin{itemize}
% \item \(x \in U_1 \implies \exists B_1 \in \mathcal{B}\) such that \(x \in B_1 \subset U_1\).
% \item \(x \in U_2 \implies \exists B_2 \in \mathcal{B}\) such that \(x \in B_2 \subset U_2\).
% \end{itemize}

% By property (2), \(\exists B_3(x) \subset B_1 \cap B_2 \subset U_1 \cap U_2\) with \(x \in B_3(x)\).

% Thus \(U_1 \cap U_2 = \bigcup_{x \in U_1 \cap U_2} B_3(x)\), so it is in \(\mathcal{U}\).
% \end{enumerate}
% \end{proof}

\subsection{Examples and Lemma}

\begin{example}[Examples of Topologies]
\leavevmode
\begin{enumerate}
\item[\bfseries Ex 1.]
Any metric space \(X\) is a topological space, with \(\mathcal{U}\) defined as:
\(U \in \mathcal{U}\) if \(\forall x \in U, \exists \varepsilon > 0: U_\varepsilon(x) \subset U\).
(Open sets are defined ``as usual for metric spaces'').

\item[\bfseries Ex 2a.]
\(X\) arbitrary, \(\mathcal{U} = 2^X\) (discrete topology). Every subset is open.

\item[\bfseries Ex 2b.]
\(X\) arbitrary, \(\mathcal{U} = \{\varnothing, X\}\) (``antidiscrete topology'').

\item[\bfseries Ex 3.]
\(X\) infinite set (say, \(\mathbb{R}\)).
\(U \in \mathcal{U}\) if \(X \setminus U\) is finite, or \(U = \varnothing\).
Check axioms:
\begin{enumerate}
\item \(\varnothing\) is open (def). \(X \setminus X = \varnothing\) (finite) \(\implies X\) open.
\item Union: \(X \setminus (\bigcup U_\alpha) = \bigcap (X \setminus U_\alpha)\). Intersection of finite sets is finite.
\item Intersection: \(X \setminus (U_1 \cap U_2) = (X \setminus U_1) \cup (X \setminus U_2)\). Union of finite sets is finite.
\end{enumerate}
\end{enumerate}
\end{example}

\begin{remark}[Warning]
No ``minimality'' is required from \(\mathcal{B}\).
\end{remark}

\begin{example}[Standard Topology on \(\mathbb{R}^n\)]
Basis generating the standard topology on \(\mathbb{R}^n\):
\[\{ U_\varepsilon(x) \mid x \in \mathbb{Q}^n, \varepsilon \in \mathbb{Q}, \varepsilon > 0 \} = \mathcal{B}\]
This \(\mathcal{B}\) is countable \(\implies \mathbb{R}^n\) is second-countable.
\end{example}

\begin{lemma}
Let \(X\) be a topological space and \(\mathcal{C}\) be a collection of open sets such that \(\forall\) open \(U \subset X\), \(\forall x \in U\), we have \(x \in C \subset U\) for some \(C \in \mathcal{C}\).
Then \(\mathcal{C}\) is a basis of \(X\).
\end{lemma}

\begin{remark}
Proof is commented out!
\end{remark}

% \begin{proof}
% Need to check \(\mathcal{C}\) is a basis.
% \begin{enumerate}
% \item Take \(U=X\) (open). Lemma says \(\forall x \in X, \exists C \in \mathcal{C}, x \in C \subset X\).
% \item Take \(C_1, C_2 \in \mathcal{C}\) (open \(\implies C_1 \cap C_2\) open), then
% \[\forall x \in C_1 \cap C_2, \exists C_3 \in \mathcal{C} \text{ s.t. } x \in C_3 \subset C_1 \cap C_2.\]
% \end{enumerate}
% \end{proof}

\subsection{Comparing Topologies}

Consider topologies \(\mathcal{T}, \mathcal{U}\) on \(X\).

\begin{definition}[Comparing Topologies]
If \(\mathcal{T} \subset \mathcal{U}\), then
\begin{itemize}
\item \(\mathcal{U}\) is \textbf{finer} than \(\mathcal{T}\).
\item \(\mathcal{T}\) is \textbf{coarser} than \(\mathcal{U}\).
\end{itemize}
(Every set open in \(\mathcal{T}\) is open in \(\mathcal{U}\), but possibly not vice versa).
\end{definition}

\begin{note}[Extremes]
\begin{itemize}
\item Discrete topology \(\mathcal{U} = 2^X\) is the \textbf{finest} topology.
\item Anti-discrete topology \(\mathcal{U} = \{\emptyset, X\}\) is the \textbf{coarsest} one.
\end{itemize}
\end{note}

\begin{definition}[Topology Generated by \(\mathcal{C}\)]
Let \(\mathcal{T}\) be the topology generated by \(\mathcal{C}\) and define the initial topology on \(X\) by \(\mathcal{U}\), then
\[\mathcal{U} \subseteq \mathcal{T}.\]

Any element from \(\mathcal{U}\) can be obtained as a union of elements from \(\mathcal{C}\).
Indeed, for any \(U \in \mathcal{U}, x \in U\), we have \(x \in C_x \subset U\). So \(\bigcup C_x = U\).
We need the opposite: \(\mathcal{T} \subset \mathcal{U}\).
But \(\mathcal{C} \subset \mathcal{U}\), and \(\mathcal{U}\) is closed under unions (and intersections, by the Lemma/Basis property), so \(\mathcal{T} \subseteq \mathcal{U}\).
\end{definition}

\begin{lemma}
Let \(\mathcal{B}\) and \(\mathcal{C}\) be bases for \(\mathcal{T}\) and \(\mathcal{U}\). Then TFAE:
\begin{enumerate}
\item \(\mathcal{U}\) is finer than \(\mathcal{T}\).
\item \(\forall x \in X, B \in \mathcal{B}\), if \(x \in B\), then \(\exists C \in \mathcal{C}\) s.t. \(x \in C \subset B\).
\end{enumerate}
\end{lemma}

\begin{remark}
Proof is commented out!
\end{remark}

% \begin{proof}
% \begin{itemize}
% \item \((1) \implies (2)\)

% Let \(x \in X\) and \(B \in \mathcal{B}\) with \(x \in B\).
% Since \(\mathcal{B} \subset \mathcal{T}\) and \(\mathcal{T} \subset \mathcal{U}\), we have \(B \in \mathcal{U}\).
% Since \(\mathcal{C}\) is a basis for \(\mathcal{U}\), \(\exists C \in \mathcal{C}\) such that \(x \in C \subset B\).

% \item \((2) \implies (1)\)

% Let \(U \in \mathcal{T}\). Show that \(U \in \mathcal{U}\).
% \begin{itemize}
% \item \(\forall x \in U\), since \(\mathcal{B}\) is a basis for \(\mathcal{T}\), \(\exists B \in \mathcal{B}\) such that \(x \in B \subset U\).
% \item By condition (2), \(\exists C_x \in \mathcal{C}\) such that \(x \in C_x \subset B \subset U\).
% \end{itemize}

% Since \(\mathcal{C}\) is a basis for \(\mathcal{U}\), each \(C_x\) is open in \(\mathcal{U}\).
% Thus \(U = \bigcup_{x \in U} C_x\) is in \(\mathcal{U}\).
% \end{itemize}
% \end{proof}

\begin{example}[Lower Limit Topology in \(\mathbb{R}\)]
\begin{itemize}
\item Standard topology: Basis \(\{(a, b) \mid a < b\}\).
\item Lower limit topology (\(\mathbb{R}_\ell\)): Basis \(\{[a, b) \mid a < b\}\).
\end{itemize}
Then \(\mathbb{R}_\ell\) is finer than \(\mathbb{R}\). \([a, b)\) is open in \(\mathbb{R}_\ell\), but not open in \(\mathbb{R}\): take \(x=a\). There is no \((a', b') \subset [a, b)\), \(x \in (a', b') \implies \mathbb{R}_\ell \neq \mathbb{R} \implies \mathbb{R}_\ell\) strictly finer than \(\mathbb{R}\).

Ex. \(\mathbb{R}_\ell\) does not have a countable basis.
\end{example}

\begin{definition}[Order Topology]
\(X\) completely ordered set.
\[\mathcal{B} = \{ (a, b) \} \cup \{ [a_0, b) \} \cup \{ (a, b_0] \}\]
where \((a, b) = \{ x \in X \mid a < x < b \}\), \(a_0 = \min X\) (if exists), \(b_0 = \max X\) (if exists).
\end{definition}

\subsection{Product Topology}
\((X, \mathcal{U}), (Y, \mathcal{V})\) topological spaces.
\(X \times Y = \{(x, y) \mid x \in X, y \in Y\}\).

\begin{note}[Idea]
Take \(\mathcal{U} \times \mathcal{V}\) to be a \textbf{basis} of topology on \(X \times Y\). Closed under intersection:
\[(U_1 \times V_1) \cap (U_2 \times V_2) = (U_1 \cap U_2) \times (V_1 \cap V_2).\]
\end{note}

\begin{definition}[Product Topology]
The topology on \(X \times Y\) defined by basis \(\mathcal{U} \times \mathcal{V}\) is called the \textbf{product topology} on \(X \times Y\).
\end{definition}

\begin{theorem}
Let \(\mathcal{B} \subset \mathcal{U}\) be a basis of \(\mathcal{U}\), \(\mathcal{C} \subset \mathcal{V}\) be a basis of \(\mathcal{V}\).
\[\mathcal{D} = \mathcal{B} \times \mathcal{C} = \{ B \times C \subset X \times Y \mid B \in \mathcal{B}, C \in \mathcal{C} \}\]
is a basis of the product topology on \(X \times Y\).
\end{theorem}

\begin{remark}
Proof is commented out!
\end{remark}

% \begin{proof}
% Need to show: for every open \(W \subset X \times Y\), we have
% \[(x,y) \in B \times C \in \mathcal{D}.\]

% But \(\exists\) basis element \(U \times V \subset W\).
% Now
% \begin{align*}
% \exists B &: x \in B \subset U, B \in \mathcal{B}; \\
% \exists C &: y \in C \subset V, C \in \mathcal{C}.
% \end{align*}

% So \((x,y) \in B \times C \subset U \times V \subset W\), \(\mathcal{B} \times \mathcal{C} = \mathcal{D}\) is a basis.
% \end{proof}

\begin{example}
\(X \times Y\). Projections \(\pi_1, \pi_2\).

If \(U\) is open in \(X\), then \(\pi_1^{-1}(U) = \{ (x,y) \mid x \in U \} = U \times Y\) open in \(X \times Y\).

\(\pi_1\) is a continuous map! Same with \(\pi_2\).

\(U \times V = \pi_1^{-1}(U) \cap \pi_2^{-1}(V)\) open.

So \(S = \{ \pi_1^{-1}(U) \mid U \subset X \text{ open} \} \cup \{ \pi_2^{-1}(V) \mid V \subset Y \text{ open} \}\) is a basis (subbasis) for \(\mathcal{U} \times \mathcal{V}\).

``The product topology is the \textbf{coarsest} topology s.t. \(\pi_1, \pi_2\) are continuous.''
\end{example}
