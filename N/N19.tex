\section{Notes 19 - 12.19}

\subsection{Countability Axioms Analysis}

\begin{theorem}
Let \(X\) be a second-countable space. Then:
\begin{enumerate}
\item Every covering of \(X\) by open sets has a countable subcovering (\(X\) is a \textbf{Lindelöf space}).
\item There exists a countable dense subset of \(X\): \(A \subset X\) such that \(\bar{A} = X\) (\(X\) is \textbf{separable}).
\end{enumerate}
\end{theorem}

\begin{remark}
The implications are as follows:
\begin{align*}
\text{2nd countable} &\implies \text{Lindelöf} \\
\text{2nd countable} &\implies \text{Separable}
\end{align*}
The converse does not hold.
    
A counterexample is the Sorgenfrey line \(\mathbb{R}_\ell\) (lower limit topology):
\begin{itemize}
\item \(\mathbb{R}_\ell\) is 1st-countable.
\item \(\mathbb{R}_\ell\) is Lindelöf and separable.
\item \(\mathbb{R}_\ell\) is \textbf{not} 2nd countable.
\end{itemize}
\end{remark}

\begin{example}[Analysis of \(\mathbb{R}_\ell\)]
\leavevmode
\begin{enumerate}
\item \textbf{First Countability:}
For \(x \in \mathbb{R}_\ell\), the collection \(\{[x, x + \frac{1}{n})\}_{n \in \mathbb{N}}\) is a countable local basis at \(x\). Thus, \(\mathbb{R}_\ell\) is first countable.

\item \textbf{Second Countability:}
\(\mathbb{R}_\ell\) is not second countable. If \(\mathcal{B}\) is a basis, then for any \(x \in \mathbb{R}\), there exists \(B_x \in \mathcal{B}\) such that \(x \in B_x \subseteq [x, x+1)\). This implies \(x = \min B_x\). Since all \(x\) are distinct, the map \(x \mapsto B_x\) is injective from the uncountable set \(\mathbb{R}\) to \(\mathcal{B}\). Thus \(\mathcal{B}\) must be uncountable.

\item \textbf{Separability:}
\(\mathbb{Q} \subset \mathbb{R}_\ell\) is dense. Thus \(\mathbb{R}_\ell\) is separable.
\end{enumerate}
\end{example}

\begin{proposition}
\(\mathbb{R}_\ell\) is a Lindelöf space.
\end{proposition}

\begin{remark}
Proof is commented out!
\end{remark}

% \begin{proof}
% Let \(\mathcal{A}\) be a covering of \(\mathbb{R}_\ell\) by basis elements \([a_\alpha, b_\alpha)\). We need to find a countable subset of \(\mathcal{A}\) covering \(\mathbb{R}_\ell\).

% Let \(C = \bigcup_\alpha (a_\alpha, b_\alpha)\). Note that \(C\) is open in \(\mathbb{R}\) with the standard topology.

% \textit{Claim:} \(\mathbb{R} \setminus C\) is countable.

% \textit{Proof of Claim:} Take \(x \in \mathbb{R} \setminus C\). Since \(x\) is covered by \(\mathcal{A}\), \(x \in [a_\beta, b_\beta)\) for some \(\beta\). Since \(x \notin C\) (and thus \(x \notin (a_\beta, b_\beta)\)), it must be that \(x = a_\beta\).
% Choose a rational number \(q_x \in (a_\beta, b_\beta) \cap \mathbb{Q}\). Then \(x < q_x\).
% If \(x, y \in \mathbb{R} \setminus C\) with \(x < y\), then \(q_x < q_y\). (Otherwise, if \(q_x > y\), then \(x < y < q_x\), implying \(y \in (a_\beta, b_\beta) \subset C\), a contradiction).
% The map \(x \mapsto q_x\) is an injection from \(\mathbb{R} \setminus C\) to \(\mathbb{Q}\). Hence \(\mathbb{R} \setminus C\) is countable.

% Since \(\mathbb{R} \setminus C\) is countable, we can choose a countable subcollection \(\mathcal{A}' \subset \mathcal{A}\) covering \(\mathbb{R} \setminus C\).
% Now, consider \(C\) as a subspace of \(\mathbb{R}_{std}\). \(C\) is covered by the collection \(\{(a_\alpha, b_\alpha)\}\), which are open in \(\mathbb{R}_{std}\). Since \(\mathbb{R}_{std}\) is second countable (and thus Lindelöf), there exists a countable subcover corresponding to \(\mathcal{A}'' \subset \mathcal{A}\).
% Then \(\mathcal{A}' \cup \mathcal{A}''\) is a countable subcover of \(\mathbb{R}_\ell\).
% \end{proof}

\begin{example}[Lindelöf Subspaces]
A subspace of a Lindelöf space is not necessarily Lindelöf.
Consider \(I_{ord}^2 = [0, 1] \times [0, 1]\) with the order topology given by the lexicographical order
\[(x, y) < (x', y') \iff x < x' \lor (x = x' \land y < y')\]

This space is compact (hence Lindelöf).
However, the subspace \([0, 1] \times (0, 1)\) is not Lindelöf. It can be covered by \(\{ \{x\} \times (0, 1) \}_{x \in [0, 1]}\), which has no countable subcover.
\end{example}

\begin{proposition}
\(\mathbb{R}_\ell^2 = \mathbb{R}_\ell \times \mathbb{R}_\ell\) is not Lindelöf.
\end{proposition}

\begin{note}
The line \(L = \{(x, -x) : x \in \mathbb{R}\}\) is closed and discrete in \(\mathbb{R}_\ell^2\), preventing a countable subcover for the complement cover strategy.
\end{note}

\subsection{Separation Axioms}

\begin{definition}[Separation Axioms]
A topological space \(X\) can satisfy the following separation axioms
\begin{enumerate}
\item[\textbf{(T1)}] \textbf{(Fréchet)}
Given \(x \neq y \in X\), there exists open \(U\) s.t. \(x \in U, y \notin U\).
        
Equivalent to: \(\{x\}\) is closed for every \(x \in X\).

\item[\textbf{(T2)}] \textbf{(Hausdorff)}
Given \(x \neq y \in X\), there exist open \(U, V\) s.t. \(x \in U, y \in V, U \cap V = \emptyset\).

\item[\textbf{(T3)}] \textbf{(Regular)}
\(X\) satisfies (T1) and for any \(x \in X\) and closed set \(B \not\ni x\), there exist open \(U, V\) s.t. \(x \in U, B \subset V, U \cap V = \emptyset\).

\item[\textbf{(T4)}] \textbf{(Normal)}
\(X\) satisfies (T1) and for any disjoint closed sets \(A, B\), there exist open \(U, V\) s.t. \(A \subset U, B \subset V, U \cap V = \emptyset\).
\end{enumerate}
\end{definition}

\begin{lemma}[Characterization of Regularity and Normality]
Let \(X\) be a (T1) topological space, then
\begin{enumerate}
\item \(X\) is regular iff \(\forall x \in X\) and open \(U \ni x\), there exists open \(V\) such that \(x \in V \subset \overline{V} \subset U\).
\item \(X\) is normal iff \(\forall\) closed \(A\) and open \(U \supset A\), there exists open \(V\) such that \(A \subset V \subset \overline{V} \subset U\).
\end{enumerate}
\end{lemma}

\begin{remark}
Proof is commented out!
\end{remark}

% \begin{proof}
% \textit{Proof of (a):}
% \begin{itemize}
% \item[\(\Rightarrow\)]
% Let \(X\) be regular and \(x \in U \subset X\) open. Let \(B = X \setminus U\), which is closed.

% Since \(x \notin B\), there exist disjoint open sets \(V, W\) such that \(x \in V\) and \(B \subset W\).

% Since \(V \cap W = \emptyset\), \(V \subset X \setminus W\). Since \(X \setminus W\) is closed, \(\overline{V} \subset X \setminus W \subset X \setminus B = U\).

% Thus, \(\overline{V} \subset U\).

% \item[\(\Leftarrow\)]
% Converse: Let \(x \in X, B\) be a closed set not containing \(x\). Let \(U = X \setminus B\) be open.

% By hypothesis, there exists open \(V\) s.t. \(x \in V \subset \overline{V} \subset U\).

% Let \(W = X \setminus \overline{V}\). Then \(W\) is open and contains \(B\) (since \(\overline{V} \subset U \implies X \setminus U \subset X \setminus \overline{V}\)).

% Also \(V \cap W = \emptyset\). Thus, \(X\) is regular.
% \end{itemize}

% \textit{Proof of (b) is similar.}
% \end{proof}

\begin{theorem}[Preservation Properties]
\leavevmode
\begin{enumerate}
\item \(X\) is Hausdorff \(\implies\) Every subspace \(Y \subset X\) is Hausdorff.
\item \(X_\alpha\) Hausdorff \(\implies \prod X_\alpha\) is Hausdorff.
\item \(X\) is Regular \(\implies\) Every subspace \(Y \subset X\) is Regular.
\item \(X_\alpha\) Regular \(\implies \prod X_\alpha\) is Regular.
\end{enumerate}
\end{theorem}

\begin{remark}
\leavevmode
\begin{enumerate}
\item If \(X\) is Normal, a subspace \(Y \subset X\) may \textbf{not} be normal.
\item If \(X_1, X_2\) are Normal, \(X_1 \times X_2\) may \textbf{not} be normal.
\end{enumerate}
\end{remark}

\begin{remark}
Proof is commented out!
\end{remark}

% \begin{proof}[Proof of (d)]
% Let \(X_\alpha\) be regular. This implies \(X_\alpha\) is Hausdorff, so \(\prod X_\alpha\) is Hausdorff.

% Thus, points are closed sets.

% To show regularity, use the Lemma. Let \(x = (x_\alpha) \in \prod X_\alpha\). Let \(U\) be a neighborhood of \(x\).

% There exists a basic open set \(\prod U_\alpha\) such that \(x \in \prod U_\alpha \subset U\), where \(U_\alpha = X_\alpha\) for all but finitely many \(\alpha\). For each \(\alpha\), since \(X_\alpha\) is regular, choose \(V_\alpha\) such that \(x_\alpha \in V_\alpha \subset \overline{V_\alpha} \subset U_\alpha\). (If \(U_\alpha = X_\alpha\), take \(V_\alpha = X_\alpha\)).

% Let \(V = \prod V_\alpha\). Then \(\overline{V} = \prod \overline{V_\alpha} \subset \prod U_\alpha \subset U\).
% Thus, \(\prod X_\alpha\) is regular.
% \end{proof}

\begin{example}[Normal Spaces]
\textbf{Show that \(\mathbb{R}_\ell\) is normal.}
    
\textit{Fact:} \(\mathbb{R}_\ell^2\) is not normal (but it is regular).
\end{example}

\begin{example}[Hausdorff but not Regular]
\(\mathbb{R}_K\) (K-topology).

Basis given by \((a, b)\) and \((a, b) \setminus K\), where \(K = \{1/n \mid n \geq 1\}\).

\(\mathbb{R}_K\) is Hausdorff but not regular. (\(K\) is closed, but cannot be separated from \(0\).)

Suppose \(U \ni 0\), \(V \supset K\), and \(U \cap V = \emptyset\).

\(U\) must contain some basis element \((a, b) \setminus K\) containing 0.

Choose \(n\) sufficient large such that \(1/n \in (a, b)\).

Since \(1/n \in K \subset V\), take a neighborhood of \(1/n\) inside \(V\), say \((c, d)\).

We can find points common to both neighborhoods leading to contradiction or limit point arguments showing separation is impossible.
\end{example}
