\section{Notes 15 - 12.05}

\subsection{Metrizability of Infinite Products}

\begin{recap}
\begin{enumerate}
\item \(\mathbb{R}^\omega\) is \textbf{not} metrizable in the \textbf{box topology}.
\item If \(J\) is uncountable, \(\mathbb{R}^J\) is \textbf{not} metrizable in the \textbf{product topology}.
\end{enumerate}
\end{recap}

\begin{theorem}
\(\mathbb{R}^\omega\) is \textbf{metrizable} in the \textbf{product topology}.
\end{theorem}

\begin{remark}
Proof is commented out!
\end{remark}

% \begin{proof}
% \(\mathbb{R}^\omega = \{(x_1, x_2, \cdots) \mid x_i \in \mathbb{R}\}\). Let \(\bar{d}(a, b) = \min(|a - b|, 1)\) (standard bounded metric on \(\mathbb{R}\)).

% Define \(D(x, y) = \sup_{i \in \mathbb{Z}_{>0}} \left\{ \frac{\bar{d}(x_i, y_i)}{i} \right\}\).

% \paragraph*{(1) This is a metric}
% \begin{itemize}
% \item Positive, zero iff equal, symmetric: clear.
% \item Triangle inequality: \(D(x, z) \le D(x, y) + D(y, z)\).
% \[\frac{\bar{d}(x_i, z_i)}{i} \le \frac{\bar{d}(x_i, y_i) + \bar{d}(y_i, z_i)}{i} \le D(x, y) + D(y, z).\]
% Taking sup over \(i\), we get \(D(x, z) \le D(x, y) + D(y, z)\).
% \end{itemize}

% \paragraph*{(2) It defines the product topology}
% We need to show the topologies strictly match.
% \begin{itemize}
% \item[(a)] \textbf{Metric \(\implies\) Product}:
% For any \(U\) open in metric topology, show it is open in product topology. \(\forall x \in U, \exists \varepsilon > 0\) such that \(U_D(x, \varepsilon) \subset U\). We find a basic open set \(V\) in product topology s.t. \(x \in V \subset U_D(x, \varepsilon)\). Take \(N\) such that \(\frac{1}{N} < \varepsilon\). Let \(V = (x_1 - \varepsilon, x_1 + \varepsilon) \times \cdots \times (x_N - \varepsilon, x_N + \varepsilon) \times \mathbb{R} \times \mathbb{R} \times \cdots\), then \(V \subset U_D(x, \varepsilon)\).

% (Check: let \(y \in V\). If \(i \le N\), \(\frac{\bar{d}(x_i, y_i)}{i} \le \frac{\varepsilon}{1} = \varepsilon\). If \(i > N\), \(\frac{\bar{d}(x_i, y_i)}{i} \le \frac{1}{i} < \frac{1}{N} < \varepsilon\). Thus, \(D(x, y) \le \max(\cdots) < \varepsilon\)).

% \item[(b)] \textbf{Product \(\implies\) Metric}:
% Let \(U = \prod U_i\) be basic open in product topology (\(U_i = \mathbb{R}\) for \(i > n\)). Let \(x \in U\). Need: \(U_D(x, \varepsilon) \subset U\). Since \(U_i\) is open in \(\mathbb{R}\), \(\exists \varepsilon_i > 0\) s.t. \((x_i - \varepsilon_i, x_i + \varepsilon_i) \subset U_i\) (for \(i = 1 \cdots n\)). Let \(\varepsilon = \min_{i=1 \cdots n} \{ \frac{\varepsilon_i}{i} \}\). Claim: \(U_D(x, \varepsilon) \subset U\). Let \(y \in U_D(x, \varepsilon)\), then
% \[\frac{\bar{d}(x_i, y_i)}{i} \le D(x, y) < \varepsilon \le \frac{\varepsilon_i}{i} \implies \bar{d}(x_i, y_i) < \varepsilon_i \implies |x_i - y_i| < \varepsilon_i \implies y_i \in U_i.\]
% \end{itemize}
% \end{proof}

\subsection{Properties of Ordered Sets and Compactness}

\subsubsection{Least Upper Bound Property}
\begin{note}[Least Upper Bound Property]
Let \(X\) be a linearly ordered set with \(<\) topology. Basis: \((a, b) = \{x \in X \mid a < x < b\}\).

\(A \subset X\): \(A\) is \textbf{bounded from above} if \(\exists b \in X\) s.t. \(a \le b, \forall a \in A\).

\(b\) is an upper bound if \(b' < b \implies \exists a \in A\) s.t. \(b' < a\) (implies \(b\) is \textbf{least} upper bound).
\end{note}

\begin{definition}[Least Upper Bound Property]
\((X, \le)\) has the \textbf{least upper bound property} if every nonempty subset \(A \subset X\) bounded from above has \(\sup A \in X\).
\end{definition}

\begin{theorem}[Compactness of Intervals]
Let \(X\) be an ordered set with the least upper bound property. Then \(\forall a \le b \in X\), the interval \([a, b]\) is compact.
\end{theorem}

\begin{remark}
Proof is commented out!
\end{remark}

% \begin{proof}
% Let \(\mathcal{A}\) be an open covering of \([a, b]\).

% \textbf{Step 1}: For all \(x \in [a, b], x \neq b\), there exists \(y > x\) such that \([x, y]\) is covered by at most two intervals from \(\mathcal{A}\).
% \begin{itemize}
% \item If \(x\) has an immediate successor \(y \in X\), take it. Then \([x, y] = \{x, y\}\), which can be covered by 2 sets from \(\mathcal{A}\).
% \item If not: Take \(A \in \mathcal{A}\) such that \(x \in A\). Since \(A\) is open, there exists \(z > x\) such that \([x, z) \subset A\). Take any \(y\) such that \(x < y < z\). Then \([x, y] \subset A\) is covered by \(A\).
% \end{itemize}

% \textbf{Step 2}: Let \(C\) be the set of points \(y > a\) such that \([a, y]\) has a finite subcover. By Step 1 applied to \(a\), \(C \neq \emptyset\). Also \(C\) is bounded by \(b\). Let \(c = \sup C \le b\).

% \textbf{Step 3}: Show that \(c \in C\).

% Let \(A \in \mathcal{A}\) such that \(c \in A\). There exists \(d < c\) such that \((d, c] \subset A\). Since \(c = \sup C\), there exists \(z \in C\) such that \(d < z \le c\). \([a, z]\) has a finite subcover. Adding \(A\) covers \([a, c]\). Then
% \[c \in C.\]

% \textbf{Step 4}: Show that \(c = b\).

% Suppose \(c < b\). By Step 1, \(\exists y > c\) such that \([c, y]\) is covered by at most 2 sets. Then \([a, y] = [a, c] \cup [c, y]\) has a finite subcover, so \(y \in C\), contradicting \(c = \sup C\)! Thus, \(c = b\).
% \end{proof}

\begin{corollary}
\([a, b] \subset \mathbb{R}\) is compact.
\end{corollary}

\begin{corollary}
\([a_1, b_1] \times \cdots \times [a_n, b_n] \subset \mathbb{R}^n\) is compact (Tychonoff/Product of compacts).
\end{corollary}

\begin{theorem}[Heine-Borel]
A subset \(A \subset \mathbb{R}^n\) is compact \(\iff\) \(A\) is closed and bounded (in Euclidean metric).
\end{theorem}

\begin{remark}
Proof is commented out!
\end{remark}

% \begin{proof}
% \begin{itemize}
% \item[\(\Rightarrow\)] \(A\) compact \(\implies A\) closed (subset of Hausdorff \(\mathbb{R}^n\)).

% Take covering by balls \(U_N(0)\) of radii \(1, 2, \cdots\). They cover \(A \implies\) there is a finite subcover \(U_{N_1}, \cdots, U_{N_k}\).
% Then \(A \subset U_N\) where \(N = \max(N_i)\). Thus, \(A\) is bounded.

% \item[\(\Leftarrow\)] \(A\) bounded \(\implies A \subset [-M, M]^n\) (large box).

% The box is compact (product of closed intervals). \(A\) is a closed subset of a compact set \(\implies A\) is compact.
% \end{itemize}
% \end{proof}

\begin{theorem}[Extreme Value Theorem]
Let \(f: X \to Y\) be continuous. \(Y\) ordered, \(X\) compact.
Then \(\exists c, d \in X\) such that
\[f(c) = \min f \le f(x) \le \max f = f(d) \ \forall x \in X\]
\end{theorem}

\begin{remark}
Proof is commented out!
\end{remark}

% \begin{proof}
% \(f(X) \subset Y\) is compact. Show that a compact subset of an ordered set has a largest element.

% Let \(A = f(X)\) and \(M = \sup A\) (exists by LUB property if \(Y=\mathbb{R}\), or construct cover).

% Take \(\{(-\infty, a) \mid a \in A\}\). If there is no largest element, this is an open covering of \(A\). Finite subcovering \((-\infty, a_1) \cup \cdots \cup (-\infty, a_n)\). Let \(a_{\max} = \max(a_i)\), then \(A \subset (-\infty, a_{\max})\).

% Contradiction! (Since \(a_{\max}\) is not covered.)
% \end{proof}

\subsection{Distance to a Set}

\begin{definition}
Let \((X, d)\) be a metric space, \(A \subset X, x \in X\).
\[d(x, A) = \inf \{ d(x, a) \mid a \in A \}\]
\end{definition}

\begin{proposition}
\(d(x, A)\) is a continuous function (for \(A\) fixed).
\end{proposition}

\begin{remark}
Proof is commented out!
\end{remark}

% \begin{proof}
% \(d(x, a) \le d(x, y) + d(y, a)\).

% Taking inf over \(a \in A\):
% \[d(x, A) \le d(x, y) + d(y, A) \implies d(x, A) - d(y, A) \le d(x, y).\]

% Symmetrically,
% \[d(y, A) - d(x, A) \le d(x, y) \implies |d(x, A) - d(y, A)| \le d(x, y).\]

% This implies continuity (Lipshitz continuous).
% \end{proof}

\begin{remark}[Goal]
Show that if \(f: X \to Y\) is continuous, \(X, Y\) metric spaces, \(X\) compact \(\implies f\) uniformly continuous.
\end{remark}
