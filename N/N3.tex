\section{Notes 3 - 10.20}

\subsection{Interior Points and Open Sets}

\begin{definition}
Let \((X, d)\) be a metric space, \(M \subset X\):
\begin{itemize}
\item \textbf{Interior Point}: \(x \in M\) is an interior point if \(\exists \varepsilon > 0\) such that \(U_\varepsilon(x) \subset M\).
\item \textbf{Interior of \(M\)}: \(\text{Int } M = \{\text{interior points of } M\}\).
\item \textbf{Open Set (Alternative Definition)}: \(M\) is open if \(\text{Int } M = M\).
\end{itemize}
\end{definition}

\begin{proposition}[Properties of Open Sets]
\begin{enumerate}
\item The intersection of finitely many open sets \(U_1 \cap \cdots \cap U_k\) is open.
\item The union of any collection of open sets \(\bigcup U_\alpha\) is open.
\end{enumerate}
\end{proposition}

\begin{theorem}
\(U \subset \mathbb{R}\) is open if and only if \(U = \bigcup (a_i, b_i)\), where:
\begin{itemize}
\item The union is at most countable;
\item \(a_i \in \{-\infty\} \cup \mathbb{R}\), \(b_i \in \mathbb{R} \cup \{+\infty\}\);
\item \((a_i, b_i) \cap (a_j, b_j) = \varnothing\) for \(i \neq j\).
\end{itemize}
\end{theorem}

\begin{example}[Discrete Metric]
For \((X, d)\) with discrete metric \(d(x,y) = \begin{cases} 1 & x \neq y \\ 0 & x = y \end{cases}\):
\begin{itemize}
\item \(\{x\}\) is open (since \(U_{1/2}(x) = \{x\} \subset \{x\}\));
\item Every subset of \(X\) is open.
\end{itemize}
\end{example}

\begin{remark}[Remarks for \(\mathbb{R}^n\)]
\begin{itemize}
\item Sets defined by \textbf{strict inequalities} (e.g., \(\{x \in \mathbb{R}^n \mid x_i > 0\}\), \(\{ (x,y) \mid x^2 + y > 3x - 2 \}\)) are open.
\item Example: \(X = \mathbb{R}^n \cong \text{Mat}(n)\) (matrices). The set \(\{ A \in \text{Mat}(n) \mid \det A \neq 0 \}\) (non-degenerate matrices) is open. For \(A = (a_{ij})\), the neighborhood:
\[U_\varepsilon(A) = \left\{ B = (a_{ij} + \varepsilon_{ij}) \mid |\varepsilon_{ij}| < \varepsilon \right\} = \{ B \mid d(A, B) < \varepsilon \}\]

Choosing \(\varepsilon < \|\det A\|\) ensures \(\det B \neq 0\) (via expansion of \(\det B\)).
\end{itemize}
\end{remark}

\begin{note}[Estimate for Determinant Difference]
For \(A \in \text{Mat}(n)\) (\(\det A \neq 0\)) and \(B = A + E\) (\(\|E\| < \varepsilon\)):
\begin{itemize}
\item The determinant difference satisfies:
\[|\det B - \det A| < (2^n - 1) n! \|A\|^{n-1} \varepsilon\]
\item Choose \(\varepsilon < \frac{|\det A|}{(2^n - 1) n! \|A\|^{n-1}}\): this ensures \(\det B \neq 0\), so \(\{ A \mid \det A \neq 0 \}\) is open.
\end{itemize}
\end{note}

\subsection{Closed Sets and Closure}

\begin{definition}
Let \((X, d)\) be a metric space, \(M \subset X\):
\begin{itemize}
\item \textbf{Limit Point}: \(x \in X\) is a limit point of \(M\) if \(\forall \varepsilon > 0\), \(\cdot{U}_\varepsilon(x) \cap M \neq \varnothing\) (\(\cdot{U}_\varepsilon(x) = U_\varepsilon(x) \setminus \{x\}\)). Equivalently, \(U_\varepsilon(x)\) contains infinitely many points of \(M\).
\item \textbf{Isolated Point}: \(x \in M\) is isolated if \(\exists \varepsilon > 0\) such that \(U_\varepsilon(x) \cap M = \{x\}\).
\item \textbf{Closure}: \(\overline{M} = M \cup \{\text{limit points of } M\}\) (union of \(M\) and its limit points).
\item \textbf{Closed Set}: \(M\) is \textbf{closed} if \(\overline{M} = M\) (contains all its limit points).
\item \textbf{Property}: \(M\) is closed \(\iff X \setminus M\) is open (open sets have closed complements, and vice versa).
\item \textbf{Example}: Prove \(\overline{\overline{M}} = \overline{M}\) (closure of closure is closure).
\end{itemize}
\end{definition}

\begin{proposition}[Characterization of Closed Sets]
A subset \(M \subset X\) is \textbf{closed} if and only if
\[\forall y \in X \setminus M, \exists \varepsilon > 0 \text{ such that } U_\varepsilon(y) \cap M = \varnothing\]
(Any point not in \(M\) can be separated from \(M\).)
\end{proposition}

\begin{remark}
Proof is commented out!
\end{remark}

% \begin{proof}
% \begin{itemize}
% \item[\(\Rightarrow\)]
% Suppose \(M\) is closed. If \(y \in X \setminus M\) cannot be separated from \(M\), \(\forall \varepsilon > 0, U_\varepsilon(y) \cap M \neq \varnothing\), so \(y\) is a limit point of \(M\). Thus, \(y \in \overline{M} = M\) (contradiction).

% \item[\(\Leftarrow\)]
% Suppose every \(y \in X \setminus M\) can be separated from \(M\). Let \(x\) be a limit point of \(M\): if \(x \notin M\), \(\exists \varepsilon > 0\) with \(U_\varepsilon(x) \cap M = \varnothing\) (contradicts \(x\) being a limit point).

% Thus, \(x \in M\), so \(M\) is closed.
% \end{itemize}
% \end{proof}

\subsection{Continuous Maps Between Metric Spaces}

\begin{definition}[Continuous Maps]
Let \(f: (X_1, d_1) \to (X_2, d_2)\) be a function.
\begin{itemize}
\item \textbf{Definition 1}: \(f\) is \textbf{continuous} if
\[\forall x \in X_1, \forall \varepsilon > 0, \exists \delta > 0 \text{ such that } \forall x' \in X_1, d_1(x, x') < \delta \implies d_2(f(x), f(x')) < \varepsilon\]
\item \textbf{Definition 2}: \(f\) is \textbf{continuous} if for every open set \(U \subset X_2\), its preimage \(f^{-1}(U) = \{ x \in X_1 \mid f(x) \in U \}\) is open in \(X_1\).
\end{itemize}
\end{definition}

\begin{theorem}
Definition 1 \(\iff\) Definition 2.
\end{theorem}

\begin{remark}
Proof is commented out!
\end{remark}

% \begin{proof}[Proof Sketch (\(\Rightarrow\))]
% Let \(f\) be continuous (Definition 1). For open \(U \subset X_2\), take \(y \in f^{-1}(U)\) (so \(f(y) \in U\)). Since \(U\) is open, \(\exists \varepsilon > 0\) with \(U_\varepsilon(f(y)) \subset U\). By continuity, \(\exists \delta > 0\) such that \(d_1(y, x') < \delta \implies d_2(f(y), f(x')) < \varepsilon\), so \(U_\delta(y) \subset f^{-1}(U)\). Thus \(f^{-1}(U)\) is open.
% \end{proof}

% \begin{proof}[Full Proof of Equivalence]
% Let \(f: (X_1, d_1) \to (X_2, d_2)\). We prove Definition 1 (\(\varepsilon\)-\(\delta\)) \(\iff\) Definition 2 (open preimages).

% \noindent \textbf{\(\Rightarrow\)}
% To show \(f^{-1}(U)\) is open (for open \(U \subset X_2\)):
% \begin{enumerate}
% \item Take \(x \in f^{-1}(U)\), so \(f(x) = y \in U\).
% \item Since \(U\) is open, \(\exists \varepsilon > 0\) with \(U_\varepsilon(y) \subset U\).
% \item By Definition 1, \(\exists \delta > 0\) such that \(d_1(x, x') < \delta \implies d_2(f(x'), f(x)) < \varepsilon\).
% \item Thus \(U_\delta(x) \subset f^{-1}(U_\varepsilon(y)) \subset f^{-1}(U)\), so \(x\) is an interior point of \(f^{-1}(U)\).
% \item This holds for all \(x \in f^{-1}(U)\), so \(f^{-1}(U)\) is open.
% \end{enumerate}

% \noindent \textbf{\(\Leftarrow\)}
% To show the \(\varepsilon\)-\(\delta\) condition:
% \begin{enumerate}
% \item Take \(x \in X_1\), \(y = f(x)\), fix \(\varepsilon > 0\).
% \item \(U_\varepsilon(y) \subset X_2\) is open, so \(f^{-1}(U_\varepsilon(y))\) is open (by Definition 2).
% \item Since \(x \in f^{-1}(U_\varepsilon(y))\), \(\exists \delta > 0\) with \(U_\delta(x) \subset f^{-1}(U_\varepsilon(y))\).
% \item Thus \(d_1(x, x') < \delta \implies f(x') \in U_\varepsilon(y) \implies d_2(f(x'), f(x)) < \varepsilon\).
% \end{enumerate}
% \end{proof}

\begin{example}[Inverse of Continuous Bijection Need Not Be Continuous]
Let \(f: X \to Y\) be continuous and bijective. \(f^{-1}: Y \to X\) is not necessarily continuous:
\begin{enumerate}
\item[\bfseries Ex 1.]
\begin{itemize}
\item \(X = [0, 1]\) (discrete metric: \(d(x,x') = \begin{cases} 0 & x=x' \\ 1 & x \neq x' \end{cases}\)).
\item \(Y = [0, 1]\) (standard Euclidean metric).
\item \(f: X \to Y\), \(f(x) = x\):
\begin{itemize}
\item \(f\) is continuous (all subsets of \(X\) are open, so preimages of open sets in \(Y\) are open).
\item \(f^{-1}: Y \to X\), \(f^{-1}(y) = y\):
\begin{itemize}
\item \(\{ \frac{1}{2} \} \subset X\) is open, but \((f^{-1})^{-1}(\{ \frac{1}{2} \}) = \{ \frac{1}{2} \} \subset Y\) is not open (Euclidean metric). Thus \(f^{-1}\) is discontinuous.
\end{itemize}
\end{itemize}
\end{itemize}

\item[\bfseries Ex 2.]
\begin{itemize}
\item \(X = \{0\} \cup (1, 2]\).
\item \(Y = [0, 1]\).
\item \(f: X \to Y\): \(f(0) = 0\), \(f(x) = x-1\) for \(x > 0\):
\begin{itemize}
\item \(f\) is continuous and bijective.
\item \(f^{-1}(0) = \{0\} \subset X\) is open, but \((f^{-1})^{-1}(\{0\}) = \{0\} \subset Y\) is not open (Euclidean metric). Thus \(f^{-1}\) is discontinuous.
\end{itemize}
\end{itemize}
\end{enumerate}
\end{example}
