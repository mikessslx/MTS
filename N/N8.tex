\section{Notes 8 - 11.07}

\subsection{The Subspace Topology}

\begin{definition}[Subspace Topology]
Let \(X\) be a topological space with topology \(\mathcal{T}\). Let \(Y \subset X\) be an arbitrary subset.
Define the \textbf{subspace topology} \(\mathcal{T}_Y = \{ U \cap Y \mid U \in \mathcal{T} \}\).
\end{definition}

\begin{proposition}
\(\mathcal{T}_Y\) is a topology on \(Y\).
\end{proposition}

\begin{lemma}
If \(\mathcal{B}\) is a basis of \(\mathcal{T}\), then \(\mathcal{B}_Y = \{ B \cap Y \mid B \in \mathcal{B} \}\) is a basis of \(\mathcal{T}_Y\).
\end{lemma}

\begin{remark}
Proof is commented out!
\end{remark}

% \begin{proof}
% Let \(U \subset Y\) be open. Then \(U = U' \cap Y\) for some \(U'\) open in \(X\).

% Since \(\mathcal{B}\) is a basis of \(X\), \(U' = \bigcup B_\alpha\) with \(B_\alpha \in \mathcal{B}\).

% Then \(U = (\bigcup B_\alpha) \cap Y = \bigcup (B_\alpha \cap Y)\).

% These \(B_\alpha \cap Y\) are elements of \(\mathcal{B}_Y\). Thus \(\mathcal{B}_Y\) is a basis of \(Y\).
% \end{proof}

\begin{remark}[Warning]
Open in \(Y \nRightarrow\) open in \(X\).
\end{remark}

\begin{proposition}
If \(Y\) is \textbf{open} in \(X\), and \(U \subset Y\) is open in \(Y \implies U\) open in \(X\).
\end{proposition}

\begin{remark}
Proof is commented out!
\end{remark}

% \begin{proof}
% Since \(U \subset Y\) is open, we have \(U = Y \cap V\), where \(V\) is open in \(X\).

% \(Y\) is open in \(X\) \(\implies U\) is open in \(X\) (intersection of two open sets).
% \end{proof}

\begin{theorem}[Product of Subspaces]
Let \(A \subset X, B \subset Y\) be subsets. \(A \times B \subset X \times Y\). Then the product of subspace topologies on \(A\) and \(B\) is the subspace topology on \(A \times B \subset X \times Y\).
\end{theorem}

\begin{remark}
Proof is commented out!
\end{remark}

% \begin{proof}
% Let \(U \subset X, V \subset Y\) be open.

% The products \(U \times V\) form a basis of the product topology on \(X \times Y\).

% \((U \times V) \cap (A \times B)\) form a basis of subspace topology on \(A \times B\).

% But \((U \times V) \cap (A \times B) = (U \cap A) \times (V \cap B)\).

% So basis elements for both topologies are the same.
% \end{proof}

\subsection{Order Topology on Subsets}

\begin{example}
Let \(\mathbb{R} = X\) with standard order \(<\). Let \(Y = [0, 1) \cup \{2\}\).
Basis of topology on \(Y\) inherited from \(\mathbb{R}\) (subspace): \(\{(a, b) \cap Y\}\).
\begin{itemize}
\item \(\{2\}\) is open in subspace topology: \(\{2\} = Y \cap (1.9, 2.1)\).
\item \(\{2\}\) is \textbf{not} open in the order topology on \(Y\).
\end{itemize}

\textbf{Exercise}: Show that for the order topology, \([0, 1) \cup \{2\}\) is connected.
\end{example}

\begin{remark}[Warning]
Let \(Y \subset X\) subset; \(<\) same relation.
It may happen that the order topology on \((Y, <)\) is \textbf{different} from the subspace topology of the \((X, <)\) order topology!
\end{remark}

\begin{example}
\(X = Y = \mathbb{R}\), \(<\) order topology.
The product of order topology is the \textbf{standard topology} on \(\mathbb{R}^2\).
\(X \times Y = \{(x, y)\}\). Take the \textbf{lexicographic order}:
\((x, y) < (x', y')\) if
\begin{itemize}
\item either \(x < x'\);
\item or \(x = x', y < y'\).
\end{itemize}
Order topology on \((X \times Y, <)\)?

\(I \times I = [0, 1] \times [0, 1] \subset X \times Y\).
The order topology on \([0, 1] \times [0, 1]\) is \textbf{not} the subspace topology of the order topology on \(\mathbb{R} \times \mathbb{R}\)!
Example: \(\{\frac{1}{2}\} \times (\frac{1}{2}, 1]\) is \textbf{not} open in ord. top. on \(I \times I\), but is open in subspace top:
\(\{\frac{1}{2}\} \times (\frac{1}{2}, \frac{3}{2})\) is open in \(\mathbb{R} \times \mathbb{R}\).
\end{example}

\begin{definition}[Convex Set]
\(Y\) is \textbf{convex} if \(\forall a, b \in Y\), \((a, b) \subset Y\).
\end{definition}

\begin{theorem}
Let \(Y \subset X\) be a \textbf{convex} subset of \((X, <)\). Then the restriction of the order topology on \(X\) is the order topology on \(Y\).
\end{theorem}

\begin{remark}
Proof is commented out!
\end{remark}

% \begin{proof}
% Take \((a, +\infty), (-\infty, b) \subset X\). This is a basis of topology on \(X\).

% Take \(Y \subset X\).

% If \(a \in Y\), \(Y \cap (a, +\infty) = \{ y \in Y \mid a < y \}\).

% Otherwise \(a\) is either an upper bound or lower bound for \(Y\).

% \(Y \cap (a, +\infty)\) is either \(Y\) or \(\emptyset\) (depending on relation). Same for \((-\infty, b)\).
% \end{proof}

\subsection{Closed Sets and Closure}

\begin{definition}[Closed Set]
\(A \subset X\) is closed \(\iff X \setminus A\) is open.
\end{definition}

\begin{proposition}
\begin{enumerate}
\item \(\emptyset, X\) closed.
\item \(A_\alpha\) closed \(\implies \bigcap A_\alpha\) closed.
\item \(A_1, \cdots, A_n\) closed \(\implies \bigcup_{i=1}^n A_i\) closed.
\end{enumerate}
\end{proposition}

\begin{theorem}[Closed Sets in Subspace]
Let \(X\) be a topological space, \(Y \subset X\) have subspace topology. Then
\(A \subset Y\) closed in \(Y \iff A = Y \cap \tilde{A}\), where \(\tilde{A} \subset X\) is closed in \(X\).
\end{theorem}

\begin{remark}
Proof is commented out!
\end{remark}

% \begin{proof}
% \begin{itemize}
% \item[\(\Rightarrow\)]
% Let \(A \subset Y\) be closed in \(Y\), then \(Y \setminus A\) is open in \(Y\).

% So \(Y \setminus A = Y \cap U\), where \(U\) is open in \(X\).

% \(X \setminus U\) is closed in \(X\).

% \((X \setminus U) \cap Y = Y \setminus (U \cap Y) = Y \setminus (Y \setminus A) = A\). (Let \(\tilde{A} = X \setminus U\).)

% \item[\(\Leftarrow\)]
% Let \(A = Y \cap C\), where \(C\) is closed in \(X\).

% Then \(X \setminus C\) is open in \(X\), \((X \setminus C) \cap Y\) is open in \(Y\).

% But \((X \setminus C) \cap Y = (X \cap Y) \setminus (C \cap Y) = Y \setminus A\).

% So \(Y \setminus A\) is open in \(Y \implies A\) is closed in \(Y\).
% \end{itemize}
% \end{proof}

\begin{definition}[Neighborhood, Interior, Closure]
\begin{itemize}
\item If \(U \subset X\) open, \(x \in U\), then \(U\) is a \textbf{neighborhood} of \(x\).
\item \textbf{Interior} of \(A\): \(\text{Int } A = \bigcup \{ U \subset A, U \text{ open in } X \}\).
\item \textbf{Closure} of \(A\): \(\overline{A} = \bigcap \{ C \supset A, C \text{ closed in } X \}\).
\end{itemize}
Note: \(\text{Int } A \subset A \subset \overline{A}\).
\end{definition}

\begin{remark}[Warning]
If \(A \subset Y \subset X\), then the closure of \(A\) in \(Y\) and \(X\) can be different.

Example: \(A = (0, 1) \subset Y = [0, 1) \subset X = \mathbb{R}\).
\begin{itemize}
\item \(\overline{A}\) in \(\mathbb{R}\) is \([0, 1]\).
\item \(\overline{A}\) in \(Y\) is \([0, 1) = (\overline{A} \text{ in } \mathbb{R}) \cap Y\).
\end{itemize}
\end{remark}

\begin{theorem}
Let \(Y \subset X\) subspace, \(A \subset Y\) and \(\overline{A}\) be the closure of \(A\) in \(X\), then the closure of \(A\) in \(Y\) is
\[\overline{A} \cap Y.\]
\end{theorem}

\begin{remark}
Proof is commented out!
\end{remark}

% \begin{proof}
% Let \(B\) be the closure of \(A\) in \(Y\).

% \(\overline{A}\) closed in \(X\), so \(\overline{A} \cap Y\) closed in \(Y\).

% \(\overline{A} \cap Y \supset A \implies \overline{A} \cap Y \supset B\) (since \(B\) is smallest closed set containing \(A\)).

% Prove the opposite inclusion:

% \(B\) closed in \(Y \implies B = Y \cap C\), where \(C\) closed in \(X\).

% Hence \(C \supset A\) (closed sets containing \(A\)) \(\implies C \supset \overline{A}\).

% \(\overline{A} \cap Y \subset C \cap Y = B\).

% Therefore, \(B = \overline{A} \cap Y\).
% \end{proof}
