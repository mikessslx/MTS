\section{Notes 1 - 10.13}

\subsection{Metric Space}

\begin{definition}[Metric Space]
A metric space is a set \(X\) with a function \(d: X \times X \to \mathbb{R}_{\geq 0}\) (called a \textit{metric} or \textit{distance function}) satisfying the following axioms:
\begin{enumerate}
\item \textbf{Non-negativity}: \(d(x, y) \geq 0\), and \(d(x, y) = 0 \iff x = y\).
\item \textbf{Symmetry}: \(d(x, y) = d(y, x)\) for all \(x, y \in X\).
\item \textbf{Triangle Inequality}: \(d(x, y) + d(y, z) \geq d(x, z)\) for all \(x, y, z \in X\).
\end{enumerate}
\end{definition}

\begin{example}[Metric Space Examples]
\leavevmode
\begin{enumerate}
\item[\bfseries Ex 0.]
\(X = \mathbb{R}\), \(d(x, y) = |x - y|\) (standard absolute value distance).
        
\item[\bfseries Ex 1a.]
\(X\) is finite, \(d(x, y) = \begin{cases} 0 & x = y \\ 1 & x \neq y \end{cases}\) (discrete metric).
        
\item[\bfseries Ex 1b.]
\(X\) is finite, \(d(x, y) = \min\{\text{weight of a path } x \leftrightarrow y\}\) (generalization of the discrete metric, e.g., via a weighted graph).
        
\item[\bfseries Ex 2a.]
Euclidean metric in \(\mathbb{R}^n\): for \(x = (x_1, \cdots, x_n)\), \(y = (y_1, \cdots, y_n)\),
\[d(x, y) = \sqrt{(x_1 - y_1)^2 + \cdots + (x_n - y_n)^2}\]

\item[\bfseries Ex 2b.]
Manhattan metric in \(\mathbb{R}^n\):
\[d(x, y) = |x_1 - y_1| + \cdots + |x_n - y_n|\]

\item[\bfseries Ex 2c.]
\[d_\infty(x, y) = \max_{i=1, \cdots, n} |x_i - y_i|\]

\item[\bfseries Ex 3.]
\textbf{(Amazon Basin Metric).}
Let \(X = \mathbb{R}^2\), \(p = (x, y)\), \(p' = (x', y')\), then
\[d(p, p') = \begin{cases}
|x - x'| + |y| + |y'| & x \neq x' \\
|y - y'| & x = x'
\end{cases}\]

\item[\bfseries Ex 4a.]
Let \(X = C[0, 1]\) (continuous functions on \([0, 1]\)). The metric is
\[d(f, g) = \max_{x \in [0, 1]} |f(x) - g(x)|\]

\item[\bfseries Ex 4b.]
Let \(X = C[0, 1]\). The metric is
\[d(f, g) = \int_0^1 |f(x) - g(x)| dx\]
\end{enumerate}
\end{example}

\subsection{Normed Space}

\begin{definition}[Normed Space]
Let \(V\) be a vector space over \(\mathbb{R}\). A function \(\|\cdot\|: V \to \mathbb{R}_{\geq 0}\) is a norm if it satisfies:
\begin{enumerate}
\item \(\|v\| = 0 \iff v = 0\)
\item \(\|\lambda v\| = |\lambda| \|v\|\) for all \(v \in V\), \(\lambda \in \mathbb{R}\)
\item \(\|v + w\| \leq \|v\| + \|w\|\) (triangle inequality)
\end{enumerate}
\end{definition}

A norm defines a metric on \(V\) by \(d(x, y) \stackrel{\text{def}}{=} \|x - y\|\).

\begin{example}[Norm Examples]
\leavevmode
\begin{enumerate}
\item[\bfseries Ex 2a.] \textbf{(Euclidean Norm)}
For \(v = (x_1, \cdots, x_n) \in \mathbb{R}^n\):
\[\|v\| = \sqrt{x_1^2 + x_2^2 + \cdots + x_n^2}\]
        
\item[\bfseries Ex 2b.] \textbf{(Manhattan Norm)}
For \(v = (x_1, \cdots, x_n) \in \mathbb{R}^n\):
\[\|v\| = |x_1| + |x_2| + \cdots + |x_n|\]
        
\item[\bfseries Ex 2c.] \textbf{(Max Norm)}
For \(v = (x_1, \cdots, x_n) \in \mathbb{R}^n\):
\[\|v\| = \max_{i=1, \cdots, n} |x_i|\]
        
\item[\bfseries Ex 4a.] \textbf{(Supremum Norm)}
For \(f \in C[0, 1]\):
\[\|f\| = \max_{x \in [0, 1]} |f(x)|\]
        
\item[\bfseries Ex 4b.] \textbf{(L¹ Norm)}
For \(f \in C[0, 1]\):
\[\|f\| = \int_0^1 |f(x)| dx\]
\end{enumerate}
\end{example}

\subsubsection{\(l_p\)-Spaces (Sequences, \(1 \leq p \leq \infty\))}
\begin{itemize}
\item \textbf{a) \(l_\infty\)}: Space of bounded sequences. The norm is:
\[\|x\|_\infty = \sup_i |x_i|\]

\item \textbf{b) \(l_1\)}: Sequences satisfying \(\sum_{i=1}^\infty |x_i| < \infty\). The norm is:
\[\|x\|_1 = \sum_{i=1}^\infty |x_i|\]

\item \textbf{c) \(l_2\)}: Sequences satisfying \(\sum_{i=1}^\infty x_i^2 < \infty\). The norm is:
\[\|x\|_2 = \sqrt{\sum_{i=1}^\infty x_i^2}\]

\item \textbf{d) \(l_p, 1 < p < \infty\)}: The norm is:
\[\|x\|_p = \left( \sum_{i=1}^\infty |x_i|^p \right)^{\frac{1}{p}}\]
\end{itemize}

\paragraph*{Inclusion Property}
\(l_p \subset l_q\) if \(p < q\). For instance, \(x_n = \frac{1}{n}\) satisfies \(\sum \frac{1}{n} = \infty\) (not in \(l_1\)) but \(\sum \frac{1}{n^2} < \infty\) (in \(l_2\)).

\begin{theorem}
\(l_p\) spaces are normed spaces (satisfy the triangle inequality).
\end{theorem}

\subsection{Inner Product Space (Euclidean Space)}

\begin{definition}[Inner Product Space (Euclidean Space)]
A vector space \(V\) with an inner product \(\langle \cdot, \cdot \rangle: V \times V \to \mathbb{R}\) satisfying
\begin{enumerate}
\item \textbf{Symmetry}: \(\langle v, w \rangle = \langle w, v \rangle\).
\item \textbf{Linearity}: \(\langle \lambda v, w \rangle = \lambda \langle v, w \rangle\), \(\langle v + w, u \rangle = \langle v, u \rangle + \langle w, u \rangle\).
\item \textbf{Positive Definiteness}: \(\langle v, v \rangle \geq 0\) and \(\langle v, v \rangle = 0 \iff v = 0\).
\end{enumerate}
\end{definition}

An inner product defines a norm:
\[\|v\| = \sqrt{\langle v, v \rangle}\]

This norm satisfies the Cauchy-Schwarz inequality.

\begin{theorem}[Cauchy-Schwarz Inequality]
For all \(v, w \in V\) (a Euclidean space),
\[\langle v, w \rangle^2 \leq \|v\|^2 \|w\|^2 = \langle v, v \rangle \langle w, w \rangle\]

Equality holds if and only if \(v\) and \(w\) are proportional (\(v \sim w\)).
\end{theorem}

\begin{remark}
Proof is commented out!
\end{remark}

% \begin{proof}
% Consider \(w + tv \in V\) for \(t \in \mathbb{R}\). The inner product \(\langle tv + w, tv + w \rangle \geq 0\), so:
% \[0 \leq t^2 \langle v, v \rangle + 2t \langle v, w \rangle + \langle w, w \rangle\]

% This is a quadratic in \(t\); its discriminant \(D \leq 0\):
% \[D = (2\langle v, w \rangle)^2 - 4 \langle v, v \rangle \langle w, w \rangle \leq 0 \implies \langle v, w \rangle^2 \leq \langle v, v \rangle \langle w, w \rangle\]

% Equality occurs when \(D = 0\), that is, \(tv + w = 0\) for some \(t\), meaning \(v\) and \(w\) are proportional.
% \end{proof}

\begin{corollary}
\begin{enumerate}
\item For \(\|v\| = \sqrt{\langle v, v \rangle}\), the triangle inequality holds:
\[\|v + w\| \leq \|v\| + \|w\|\]
    
\item The \(l_2\)-norm satisfies the triangle inequality. \(l_2\) is a Euclidean space with inner product \(\langle x, y \rangle = \sum_{i=1}^\infty x_i y_i\) and norm \(\|x\|_2 = \sqrt{\langle x, x \rangle}\).
\end{enumerate}
\end{corollary}

\begin{remark}
The \(l_p\)-norm \(\|x\|_p\) does not come from an inner product when \(p \neq 2\).
\end{remark}
