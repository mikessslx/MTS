\section{Notes 21 - 12.23}

\subsection{Urysohn's Lemma (Recap)}

\begin{recap}[Urysohn's Lemma]
    Let \(X\) be a normal space, \(A, B \subset X\) closed, \(A \cap B = \emptyset\). Then there is a continuous function \(f: X \to [0, 1]\) such that
    \[f(x) = 0 \ \forall x \in A, \ f(x) = 1 \ \forall x \in B.\]

    (\(f\) \textit{separates} \(A\) and \(B\)).
\end{recap}

\begin{definition}[Completely Regular]
    If \(X\) satisfies \((T1)\) (points are closed) and any closed set \(B\) and point \(x_0 \notin B\) can be separated by a continuous function (that is, \(\exists f: X \to [0, 1]\) such that \(f(x_0) = 0, f(B) = 1\)), then \(X\) is \textbf{completely regular} (CR) or \((T_{3\frac{1}{2}})\).
\end{definition}

\begin{remark}
    Urysohn's Lemma implies \((T4) \implies (T_{3\frac{1}{2}})\).
    \begin{itemize}
        \item \((T3)\) (Regularity): Separate point \(x\) and closed \(A\) by disjoint \textit{open sets}.
        \item \((T_{3\frac{1}{2}})\) (Complete Regularity): Separate point \(x\) and closed \(A\) by a \textit{continuous function}.
        \item \((T4)\) (Normality): Separate closed \(A\) and closed \(B\) by disjoint \textit{open sets} (which implies separation by function via Urysohn).
    \end{itemize}
\end{remark}

\begin{proposition}
    \(X\) is completely regular \(\implies\) \(X\) is regular.
\end{proposition}

\begin{remark}
Proof is commented out!
\end{remark}

% \begin{proof}
%     Take \(x_0 \in X\), \(A \subset X\) closed, \(x_0 \notin A\).

%     Take \(f\) separating \(\{x_0\}\) and \(A\), that is, \(f: X \to [0, 1]\) with \(f(x_0) = 0, f(A) = 1\).

%     Let \(U = f^{-1}([0, \frac{1}{2})) \subset X\) and \(V = f^{-1}((\frac{1}{2}, 1]) \subset X\), both open.
    
%     Then \(x_0 \in U\) and \(A \subset V\).
%     Also \(U \cap V = \emptyset\), so \(X\) is regular.
% \end{proof}

\begin{theorem}
    \begin{enumerate}
        \item[(a)] Subspace of a CR space is CR.
        \item[(b)] Product of CR spaces is CR.
    \end{enumerate}
\end{theorem}

\begin{remark}
Proof is commented out!
\end{remark}

% \begin{proof}
%     \begin{enumerate}[(a)]
%         \item
%         Let \(X\) be CR, \(X \supset Y \ni x_0\). Let \(A \subset Y\) be closed in \(Y\) (with \(x_0 \notin A\)).

%         Let \(\overline{A}\) be the closure of \(A\) in \(X\).
%         Then we can find \(f: X \to [0, 1]\) continuous on \(X\) such that \(f(x_0) = 0\) and \(f(\overline{A}) = 1\).
%         Thus, \(f|_Y\) is the required function.

%         \item
%         Take \(A \subset \prod_{\alpha \in J} X_\alpha\) closed, \(b \notin A\), \(b = (b_\alpha)_{\alpha \in J}\).
%         \(X_\alpha\) CR \(\implies\) regular; \(\prod X_\alpha\) is regular.

%         Take \(U = \prod U_\alpha \ni b\) (open basis element) such that \(U \cap A = \emptyset\).
%         \(U_\alpha = X_\alpha\) if \(\alpha \neq \alpha_1, \cdots, \alpha_n\).
        
%         For each \(i=1, \cdots, n\), take \(f_i: X_{\alpha_i} \to [0, 1]\) such that \(f_i(b_{\alpha_i}) = 1\) and \(f_i(X_{\alpha_i} \setminus U_{\alpha_i}) = 0\).
        
%         Let \(\varphi_i(x) = f_i(\pi_{\alpha_i}(x))\) for \(x \in \prod X_\alpha\).
%         Define \(f = \varphi_1(x) \cdots \varphi_n(x) = \prod_{i=1}^n f_i(\pi_{\alpha_i}(x))\).
        
%         Then \(f(x) = 0 \ \forall x \in X \setminus U\).
%         So \(f(A) \equiv 0\), \(f(b) = \prod \varphi_i(b) = \prod f_i(b_{\alpha_i}) = 1\).
        
%         This \(f\) separates \(A\) and \(\{b\}\).
%     \end{enumerate}
% \end{proof}

\begin{remark}
    \begin{itemize}
        \item
        \((T_{3\frac{1}{2}}) \not\implies (T4)\). 
        Example: \(\mathbb{R}_l^2\) (Sorgenfrey plane).
        
        Reason: \(\mathbb{R}_l\) is normal \(\implies\) CR. Product of CR spaces is CR \(\implies \mathbb{R}_l^2\) is CR.
        
        But \(\mathbb{R}_l^2\) is \textbf{not} normal.
        
        \item
        Fact: There exist regular spaces that are not completely regular. (No proof given.)
    \end{itemize}
\end{remark}

\begin{proof}[Proof of Urysohn's Lemma for metric spaces]
    Let \((X, d)\) be a metric space and \(A, B \subset X\) be closed, \(A \cap B = \emptyset\).
    Define \(d_A(x) = \text{dist}(x, A) = \inf_{a \in A} d(x, a)\), which is continuous and
    \[d_A(x) = 0 \iff x \in \overline{A} = A.\]

    Similarly define \(d_B(x) = \text{dist}(x, B)\) and \(f(x) = \frac{d_A(x)}{d_A(x) + d_B(x)}\) (continuous).
    
    Since \(A \cap B = \emptyset\), for any \(x\), \(d_A(x)\) and \(d_B(x)\) cannot be both 0, so the denominator is non-zero.
    
    If \(x \in A\), \(d_A(x) = 0 \implies f(x) = 0\); If \(x \in B\), \(d_B(x) = 0 \implies f(x) = \frac{d_A(x)}{d_A(x)} = 1\).
\end{proof}

\subsection{Urysohn's Metrization Theorem}
\begin{theorem}[Urysohn's Metrization Theorem]
    Every regular space \(X\) with a countable basis is metrizable.
\end{theorem}

\begin{remark}
Proof is commented out!
\end{remark}

% \begin{proof}
%     Idea: Embed \(X\) into \(Y = [0, 1]^\omega \subset \mathbb{R}^\omega\).
%     \([0, 1]^\omega\) is called \textbf{Tychonoff cube}.

%     This can be \(\mathbb{R}^\omega\) with product topology (\textbf{coarser}) or \(\mathbb{R}^\omega\) with uniform metric topology (\textbf{finer}) defined by \(d(x, y) = \sup \bar{d}(x_i, y_i)\), where \(\bar{d}(a, b) = \min(|a-b|, 1)\).
    
%     \paragraph*{Step 1: Construct a family of functions}
%     Let \(\{B_n\}\) be a countable basis for \(X\).
%     Consider pairs \((m, n)\) such that \(\overline{B_n} \subset B_m\).
    
%     By regularity (and thus normality for second-countable spaces), there exists a continuous function \(g_{m, n}: X \to [0, 1]\) such that \(g_{m, n} = 1\) on \(\overline{B_n}\) and \(0\) outside \(B_m\).
    
%     This gives a countable family of functions \(\{f_n\}_{n \in \mathbb{N}}\) (reindexing the \(g_{m, n}\) via \(\mathbb{N} \times \mathbb{N} \to \mathbb{N}\)).
    
%     For any \(x_0 \in X\) and open \(U \ni x_0\), there exists \(B_m\) such that \(x_0 \in B_m \subset U\).
%     By regularity, there exists \(B_n\) such that \(x_0 \in B_n \subset \overline{B_n} \subset B_m\).
%     Then the function \(g_{m, n}\) satisfies \(g_{m, n}(x_0) = 1\) and vanishes outside \(U\).

%     \paragraph*{Step 2: Define the embedding}
%     Define \(F: X \to [0, 1]^\omega\) by \(F(x) = (f_1(x), f_2(x), \cdots)\).
%     \begin{enumerate}
%         \item[(a)] \(F\) is continuous because each component \(f_n\) is continuous (product topology).
%         \item[(b)] \(F\) is injective. If \(x \neq y\), since \(X\) is Hausdorff, there exists open \(U \ni x\) such that \(y \notin U\). By the separation property from Step 1, there exists some \(f_k\) such that \(f_k(x) > 0\) and \(f_k(y) = 0\). Thus \(F(x) \neq F(y)\).
%     \end{enumerate}
    
%     \paragraph*{Step 3: \(F\) is an embedding}
%     Let \(Z = F(X)\). \(F: X \to Z\) is a continuous bijection. We usually need to show \(F^{-1}\) is continuous, that is, \(F\) maps open sets to open sets in \(Z\).
    
%     Let \(U \subset X\) be open. Let \(z_0 \in F(U)\). Let \(x_0 = F^{-1}(z_0) \in U\).
%     We need to find an open set \(W \subset Z\) such that \(z_0 \in W \subset F(U)\).
    
%     Choose index \(N\) such that the corresponding function \(f_N\) separates \(x_0\) and \(X \setminus U\) (specifically, \(f_N(x_0) > 0\) and \(f_N(X \setminus U) = 0\)).
    
%     Let \(V = \pi_N^{-1}((0, \infty)) \cap Z = \{ (z_1, z_2, \cdots) \in Z \mid z_N > 0 \}\).
%     This is open in \(Z\).
%     Clearly \(z_0 \in V\) because \(\pi_N(z_0) = f_N(x_0) > 0\).
    
%     If \(z \in V\), let \(x = F^{-1}(z)\). Then \(f_N(x) = \pi_N(z) > 0\).
%     Since \(f_N\) vanishes on \(X \setminus U\), it must be that \(x \in U\).
%     Thus \(z = F(x) \in F(U)\).
%     So \(V \subset F(U)\), proving \(F(U)\) is open in \(Z\).
    
%     \paragraph*{Conclusion}
%     \(X\) is homeomorphic to \(F(X) \subset [0, 1]^\omega\). Since \([0, 1]^\omega\) is metrizable, \(X\) is metrizable.
% \end{proof}
