\section{Notes 17 - 12.12}

\subsection{Different Notions of Compactness}

\begin{definition}[Compact]
\(X\) is \textbf{compact} if any open covering admits a finite subcovering.
\end{definition}

\begin{definition}[LPC]
\(X\) is \textbf{limit point compact} (LPC) if every infinite \(A \subset X\) has a limit point. (Fréchet compactness).
\end{definition}

\begin{definition}[SC]
\(X\) is \textbf{sequentially compact} (SC) if every sequence \(x_1, x_2, \cdots\) has a convergent subsequence. (\(\exists n_1 < n_2 < \cdots\) s.t. \(x_{n_k}\) converges). (Bolzano-Weierstrass property).
\end{definition}

\begin{theorem}[Theorem 1]
If \(X\) is a \textbf{metric space}, then
\(X\) is compact \(\iff\) LPC \(\iff\) SC.
\end{theorem}

\begin{theorem}[Theorem 2]
For \(X\) an arbitrary \textbf{topological space}, \(X\) compact \(\implies X\) LPC. (The converse is \textbf{NOT} true).
\end{theorem}

\begin{proof}[Proof (Compact \(\implies\) LPC)]
Let \(X\) be compact, \(A \subset X\) infinite. Assume \(A\) has no limit points, then \(A\) is closed (contains all its limit points, which are none). So \(X \setminus A\) is open.

Since \(a \in A\) is \textbf{not} a limit point, \(\exists U_a\) open such that \(U_a \cap A = \{a\}\).

Take all \(\{U_a\}_{a \in A}\) and \(X \setminus A\). This is an open cover of \(X\).

Since \(X\) is compact, there exists a finite subcover,
\[X \subset (X \setminus A) \cup U_{a_1} \cup \cdots \cup U_{a_k}.\]

Since \(U_{a_i} \cap A = \{a_i\}\), we must have \(A \subset \{a_1, \cdots, a_k\} \implies A\) is finite, contradiction!
\end{proof}

\begin{example}[Counterexample: LPC but not Compact]
Let \(Y = \{p, q\}\) with anti-discrete topology \(\mathcal{T}_Y = \{\emptyset, Y\}\). Let \(X = \mathbb{N} \times Y\) with product topology (\(\mathbb{N}\) is discrete).

Open sets in \(X\) look like \(U \times Y\) where \(U \subset \mathbb{N}\). Every nonempty open set contains pairs \(\{(n, p), (n, q)\}\) for \(n \in U\). Then every nonempty set \(A \subset X\) has a limit point.

Suppose \(A \ni (n, p)\). Then \((n, q)\) is a limit point of \(A\): every open neighborhood of \((n, q)\) contains \((n, p)\) (because open sets come in ``columns'') \(\implies X\) is LPC.

But \(X\) is \textbf{not compact}: Cover by \(U_n = \{n\} \times Y\). No finite subcover. \(X\) is also \textbf{not sequentially compact}: The sequence \((1, p), (2, p), \cdots\) has no convergent subsequence.
\end{example}

\begin{proof}[Partial Proof of Theorem 1 (Metric Spaces)]
\begin{enumerate}
    \item \textbf{Compact \(\implies\) LPC}: Theorem 2.
    \item \textbf{LPC \(\implies\) SC}: (Proof works for 1st-countable spaces).
    Take a sequence \((x_n)\) in \(X\).
    \begin{itemize}
        \item
        If the set of values of \((x_n)\) is finite, then one value is assumed infinitely many times \(\implies\) constant subsequence \(\implies\) convergent.

        \item
        If \((x_n)\) has infinitely many values, let \(A\) be the set of values.
        
        By LPC, \(A\) has a limit point \(a \in X\). Since \(X\) is metric (or 1st-countable), we can find a subsequence converging to \(a\). (Take \(U_1(a) \supset U_2(a) \supset \cdots\) and pick \(x_{n_k} \in U_k(a)\) with increasing indices.)
    \end{itemize}
    \item \textbf{SC \(\implies\) Compact}: Harder.
\end{enumerate}
\end{proof}

\subsection{Locally Compact Spaces}

\begin{definition}[Locally Compact Space]
\(X\) is \textbf{locally compact at \(x\)} if \(\exists\) open \(U \ni x\), and a \textbf{compact subspace} \(C\) of \(X\) such that \(x \in U \subset C\).
\(X\) is \textbf{locally compact} if it is locally compact at every \(x \in X\).
\end{definition}

\begin{example}
    \leavevmode
    \begin{enumerate}
        \item \(\mathbb{R}\) is locally compact. \(\forall x \in \mathbb{R}\), \(x \in (a, b) \subset [a, b]\), and \([a, b]\) is compact.
        \item \(\mathbb{R}^n\) is locally compact.
        \item \(\mathbb{Q} \subset \mathbb{R}\) i \textbf{not} locally compact.
        \item \(\mathbb{R}^\omega\) with product topology is \textbf{not} locally compact. If \(U\) is open, \(U\) contains a basis element \((a_1, b_1) \times \cdots \times (a_n, b_n) \times \mathbb{R} \times \mathbb{R} \times \cdots\). Then \(U\) is not contained in any compact set \(C\). (If \(U \subset C\), then \(\overline{U} \subset C \implies \overline{U}\) is compact. But \(\overline{U}\) contains factors of \(\mathbb{R}\), and hence not compact.)
    \end{enumerate}
\end{example}

\begin{note}
Compact Hausdorff spaces are ``nice''.
\end{note}

\subsection{One-Point Compactification}

\begin{theorem}[One-Point Compactification]
Let \(X\) be a \textbf{Hausdorff} space. Then \(X\) is locally compact
\(\iff \exists Y\) such that
\begin{enumerate}
    \item \(X \subset Y\).
    \item \(Y \setminus X = \{p\}\) is a single point.
    \item \(Y\) is a \textbf{compact Hausdorff} space.
\end{enumerate}
This \(Y\) is unique in the following sense: if \(Y, Y'\) are two such spaces, there exists a homeomorphism \(h: Y \to Y'\) fixing \(X\).
\end{theorem}

\begin{remark}
Proof is commented out!
\end{remark}

% \begin{proof}
% \textbf{Uniqueness}: Let \(Y = X \cup \{p\}, Y' = X \cup \{q\}\). Define \(h: Y \to Y'\) by \(h(x) = x\) for \(x \in X\), \(h(p) = q\). Show \(h\) is continuous. Take \(U \subset Y'\) open.
% \begin{itemize}
%     \item If \(U \subset X\), \(h^{-1}(U) = U\) is open in \(X \implies\) open in \(Y\).
%     \item If \(q \in U\), let \(C = Y' \setminus U\). \(C\) is closed in \(Y'\), \(C \subset X\).
%     Since \(Y'\) compact, \(C\) is compact.
%     \(h^{-1}(U) = Y \setminus C\).
%     By construction of topology, complements of compact sets \(\subset X\) are open neighborhoods of infinity.
% \end{itemize}

% \textbf{Construction}: Let \(Y = X \cup \{\infty\}\). Define topology on \(Y\):
% \begin{enumerate}
%     \item \(U \subset X\) open \(\implies U\) open in \(Y\).
%     \item \(Y \setminus C\) open in \(Y\) if \(C \subset X\) is compact.
% \end{enumerate}

% Need to prove this defines a topology and \(Y\) is compact Hausdorff. (Continued next time).
% \end{proof}
