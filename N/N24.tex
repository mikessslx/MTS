\section{Notes 24 - 01.05}

\subsection{Stone-\v{C}ech Compactification}

Complete regular space \(X \hookrightarrow Y\) (compact Hausdorff).

``Maximality'': for every bounded continuous function on \(X\), it can be extended to a continuous function on \(Y\).

Idea: take all bounded continuous \(f_\alpha: X \to [-M_\alpha, M_\alpha] = I_\alpha\),
\[F: X \to \prod_{\alpha \in J} I_\alpha \ \text{compact (Tychonoff's thm)}, \ Y = \overline{F(X)}.\]

\subsubsection{Uniqueness}

\begin{lemma}
Let \(A \subset X\), \(f: A \to Z\) continuous, \(Z\) Hausdorff. Then there is at most one extension \(g: \overline{A} \to Z\) such that \(g|_A = f\).
\end{lemma}

\begin{remark}
Proof is commented out!
\end{remark}

% \begin{proof}
% Let \(g, g': \overline{A} \to Z\) and take \(x \in \overline{A} \setminus A\): \(g(x) \neq g'(x)\).

% Let \(U, U'\) be neighborhoods of \(g(x), g'(x) \in Z\).
% Take \(V \ni x\) open s.t. \(g(V) \subset U\), \(g'(V) \subset U'\).

% Then \(V \cap A \neq \varnothing\). If \(y \in V \cap A\), then \(g(y) \in U\), \(g'(y) \in U'\).

% But \(g(y) = g'(y)\) since \(y \in A\). But \(U \cap U' = \varnothing\), contradiction!
% \end{proof}

\begin{theorem}
Let \(X\) be completely regular. \(Y\) compactification of \(X\). \(Y\) satisfies the Stone-\v{C}ech extension property. Then any continuous map \(f: X \to C\), \(C\) compact Hausdorff, uniquely extends to a continuous map \(Y \to C\).
\end{theorem}

\begin{remark}
Proof is commented out!
\end{remark}

% \begin{proof}
% Take \(C \hookrightarrow [0, 1]^J\). Then \(f = (f_\alpha)_{\alpha \in J}\), \(f_\alpha: X \to [0, 1]\).

% Then \(f_\alpha\) extends to \(g_\alpha: Y \to [0, 1]\), \(g(y) = (g_\alpha(y))_{\alpha \in J}\) continuous, \(g: Y \to \overline{C} = C\):
% \[g(Y) = g(\overline{X}) \subset \overline{g(X)} = \overline{f(X)} \subset \overline{C} = C.\]
% \end{proof}

\begin{theorem}
Let \(Y_1, Y_2\) be compactifications of \(X\) satisfying the Stone-\v{C}ech condition, \(X\) is completely regular. Then \(Y_1 \cong Y_2\) are homeomorphic.
\end{theorem}

\begin{remark}
Proof is commented out!
\end{remark}

% \begin{proof}
% \(j_1: X \hookrightarrow Y_1\) maps into a compact set, extends to \(f_1: Y_2 \to Y_1\) so it satisfies the previous Thm.
% \(j_2: X \hookrightarrow Y_2\) extends to \(f_2: Y_1 \to Y_2\).

% \(f_1 \circ f_2: Y_1 \to Y_1\), \((f_1 \circ f_2)(x) = x\) continuous map.

% \(f_1 \circ f_2\) is a cont. extension of \(Id_X\).
% But \(Id_{Y_1}\) is another ext. of \(Id_X\).

% So \(f_1 \circ f_2 = Id_{Y_1}\) acc. to Lemma.
% Similarly \(f_2 \circ f_1 = Id_{Y_2}\). So \(f_1 = f_2^{-1}\).
% \end{proof}

\subsection{Complete Metric Spaces}

\begin{definition}
Let \((X, d)\) be a metric space. Then \((x_n)\) is a \underline{Cauchy sequence} if
\[\forall \varepsilon > 0 \ \exists N \in \mathbb{N}: \forall n, m > N \ d(x_n, x_m) < \varepsilon.\]

Convergent \(\implies\) Cauchy: if \(x_n \to x\), then
\[\forall \varepsilon \exists N: \forall n > N \ d(x_n, x) < \frac{\varepsilon}{2}, \forall m > N \ d(x_m, x) < \frac{\varepsilon}{2} \implies d(x_n, x_m) < \varepsilon.\]
\end{definition}

\begin{definition}
\((X, d)\) is \underline{complete} if every Cauchy seq. converges.
\end{definition}

\begin{example}
\leavevmode
\begin{enumerate}
\item[\bfseries Ex 1.] \((0, 1)\) not complete: \(x_n = \frac{1}{n}\) is Cauchy, but not conv (in \((0, 1)\)).
\item[\bfseries Ex 2.] \(\mathbb{Q}\) not complete: \(x_0 = 3, x_1 = 3.1, x_2 = 3.14, x_3 = 3.141, x_4 = 3.1415 \cdots\)
\(x_n = \frac{\lfloor 10^n \pi \rfloor}{10^n}\).
\end{enumerate}
\end{example}

\begin{remark}
\leavevmode
\begin{enumerate}
\item[\bfseries Rem 1.] If \(A \subset X\), \(A\) closed, \(X\) complete \(\implies A\) complete.
\item[\bfseries Rem 2.] Take \(\tilde{d}(x, y) = \min(d(x, y), 1)\). If \(X\) is complete w.r.t. \(d\), it is complete w.r.t. \(\tilde{d}\).
\end{enumerate}
\end{remark}

\begin{lemma}
\(X\) is complete iff every Cauchy sequence has a convergent subsequence.
\end{lemma}

\begin{remark}
Proof is commented out!
\end{remark}

% \begin{proof}
% \begin{itemize}
% \item[\(\Rightarrow\)] obvious.
% \item[\(\Leftarrow\)]
% \((x_n)\) Cauchy, suppose \(x_{n_k} \xrightarrow{n \to \infty} x\).
% Need to show: \(x_n \to x\).

% \(\forall \varepsilon > 0\), take \(N\): \(\forall n, m > N\), \(d(x_n, x_m) < \frac{\varepsilon}{2}\); \(\forall n_i\) from subseq., \(d(x_{n_i}, x) < \frac{\varepsilon}{2}\).

% \(\forall n, n_i > N\), \(d(x_n, x) \leq d(x_n, x_{n_i}) + d(x_{n_i}, x) < \frac{\varepsilon}{2} + \frac{\varepsilon}{2} = \varepsilon\).
% \end{itemize}
% \end{proof}

\begin{theorem}
\(\mathbb{R}^k\) is complete in \(d_1, d_2, d_\infty\) (standard metrics).
\end{theorem}

\begin{remark}
Proof is commented out!
\end{remark}

% \begin{proof}
% Show this for \(d_\infty(x, y) = \max_{i=1 \cdots k} |x_i - y_i|\).
% Take \(x_n\) Cauchy \(\implies\) bounded.

% So all \(x_n\) lie in \([-M, M]^k\).
% \([-M, M]^k\) compact \(\iff\) seq. compact (metric space!).

% So \(x_n\) has a convergent subsequence \(\implies\) convergent by the previous Lemma.

% All these metrics are equiv. \(\implies\) holds for \(d_1, d_2\).
% \end{proof}

\begin{theorem}
There is a metric on \(\mathbb{R}^\omega\) s.t. \(\mathbb{R}^\omega\) is complete w.r.t. this metric.
\end{theorem}

\begin{lemma}
Let \(X = \prod_{\alpha \in J} X_\alpha, x_n \in X\).
Then
\[x_n \xrightarrow{n \to \infty} x \iff \forall \alpha \in J \ \pi_\alpha(x_n) \xrightarrow{n \to \infty} \pi_\alpha(x).\]
\end{lemma}

\begin{remark}
Proof is commented out!
\end{remark}

% \begin{proof}
% \begin{itemize}
% \item[\(\Rightarrow\)]
% \(\pi_\alpha: X \to X_\alpha\) are continuous.

% \item[\(\Leftarrow\)]
% Suppose \(\pi_\alpha(x_n) \to \pi_\alpha(x) \forall \alpha \in J\).
% Let \(U = \prod U_\alpha \ni x\) be a n'hood containing \(x\).

% Then \(U_\alpha = X_\alpha\) unless \(\alpha \in \{\alpha_1, \cdots, \alpha_m\}\).
% For \(\alpha \in \{\alpha_1, \cdots, \alpha_m\}\),
% \[\exists N_\alpha: \pi_\alpha(x_n) \in U_\alpha, \ \forall n > N_\alpha.\]

% Take \(N = \max(N_{\alpha_1}, \cdots, N_{\alpha_m})\), then \(\forall v > N\), \(\pi_\alpha(x_n) \in U_\alpha\).
% So \(x_n \in U = \prod U_\alpha\).
% \end{itemize}
% \end{proof}

\begin{remark}
Proof of Thm is commented out!
\end{remark}

% \begin{proof}[Proof of Thm]
% \(\tilde{d}(a, b) = \min(1, |a - b|)\).
% Take \(D(x, y) = \sup_{i \in \mathbb{N}} \left( \frac{\tilde{d}(x_i, y_i)}{i} \right)\).

% Show that \((\mathbb{R}^\omega, D)\) is complete.
% Let \((x_n) \in \mathbb{R}^\omega\) be Cauchy,
% \[\tilde{d}(\pi_i(x_n), \pi_i(y_n)) \leq i D(x, y).\]

% For a fixed \(i\), \(\pi_i(x_n)\) is Cauchy in \(\mathbb{R}\), so \(\pi_i(x_n) \to a_i\).

% Then \(x_n \to (a_1, a_2 \cdots) \in \mathbb{R}^\omega\) by Lemma.
% \end{proof}

\subsection{Uniform Metric}
\begin{definition}
Let \((Y, d)\) be a metric space. \(\tilde{d} = \min(|x - y|, 1)\).
Take functions \(J \to Y\), \(J\) set: this is \(Y^J\).
There is a metric \(\bar{\rho}\) on \(Y^J\) defined by:
if \(x = (x_\alpha)_{\alpha \in J}, y = (y_\alpha)_{\alpha \in J}\), then
\[\bar{\rho}(x, y) = \sup_{\alpha \in J} \{ \tilde{d}(x_\alpha, y_\alpha) \}.\]

This is a metric (Exercise).
It is called \underline{uniform metric} on \(Y^J = \text{Fun}(J, Y)\).

Ex: \(J = [0, 1], Y = \mathbb{R}\). \(Y^J = \{ \text{functions } [0, 1] \to \mathbb{R} \}\).
\(d(f, g) = \sup (\min(|f(x) - g(x)|, 1))\).
\end{definition}

\begin{theorem}
If \(Y\) is complete w.r.t. \(d\), then \((Y^J, \bar{\rho})\) is complete w.r.t.
\[\bar{\rho} = \sup_{\alpha \in J} \{ \tilde{d}(x_\alpha, y_\alpha) \}.\]
\end{theorem}

\begin{remark}
Proof is commented out!
\end{remark}

% \begin{proof}
% \((Y, d)\) complete \(\implies (Y, \tilde{d})\) complete.

% Take \(f_n: J \to Y\), \(f_n\) Cauchy seq. w.r.t. \(\bar{\rho}\).
% \(\forall \varepsilon > 0, \ \exists N: \forall m, n > N \ \bar{\rho}(f_n, f_m) < \varepsilon\).

% \(\tilde{d}(f_n(\alpha), f_m(\alpha)) \leq \bar{\rho}(f_n, f_m) < \varepsilon \implies f_n(\alpha)\) is Cauchy.

% Suppose \(f_n(\alpha) \xrightarrow{n \to \infty} y_{\alpha}\), \(y = (y_\alpha)_{\alpha \in J} \in Y^J\).
% Claim: \(f_n \to y\) in \(\bar{\rho}\).

% For \(\varepsilon > 0\): choose \(N\) s.t. \(\bar{\rho}(f_n, f_m) < \frac{\varepsilon}{2}\) \(\forall n, m > N\).

% Fix \(n\), fix \(\alpha\), take \(m\) large: \(\tilde{d}(f_n(\alpha), y_\alpha) < \frac{\varepsilon}{2}\).

% This holds for \(\forall \alpha \in J\) if \(n > N\). So \(\bar{\rho}(f_n, y) \leq \frac{\varepsilon}{2} < \varepsilon\).
% \end{proof}
