%%%%%%%%%%%%%%%%% DO NOT CHANGE HERE %%%%%%%%%%%%%%%%%%%% {
    \documentclass[12pt,letterpaper]{report}
    \usepackage{fullpage}
    \usepackage[top=2cm, bottom=4.5cm, left=2.5cm, right=2.5cm]{geometry}
    \usepackage{amsmath,amsthm,amsfonts,amssymb,amscd}
    \usepackage{lastpage}
    \usepackage{enumerate}
    \usepackage{fancyhdr}
    \usepackage{mathrsfs}
    \usepackage{xcolor}
    \usepackage{graphicx}
    \usepackage{listings}
    \usepackage{hyperref}
    
    \hypersetup{%
      colorlinks=true,
      linkcolor=blue,
      linkbordercolor={0 0 1}
    }
    
    \theoremstyle{plain}
    \newtheorem{theorem}{Theorem}[section]
    \newtheorem{lemma}[theorem]{Lemma}
    \newtheorem{proposition}[theorem]{Proposition}
    \newtheorem{corollary}[theorem]{Corollary}

    \theoremstyle{definition}
    \newtheorem{definition}[theorem]{Definition}
    \newtheorem{example}{Example}[section]
    \newtheorem{xca}[theorem]{Exercise}

    \theoremstyle{remark}
    \newtheorem*{remark}{Remark}
    \newtheorem*{note}{Note}
    \newtheorem*{recap}{Recap}
    \newtheorem*{hint}{Hint}
    
    \setlength{\parindent}{0.0in}
    \setlength{\parskip}{0.05in}
%%%%%%%%%%%%%%%%%%%%%%%%%%%%%%%%%%%%%%%%%%%%%%%%%%%%%%%%%% }
    
%%%%%%%%%%%%%%%%%%%%%%%% CHANGE HERE %%%%%%%%%%%%%%%%%%%% {
    \newcommand\course{MTS}
    \newcommand\semester{Winter 2025}
    \newcommand\NetIDa{Xiang Li}
    \newcommand\NetIDb{\href{https://github.com/Mikessslx}{\textcolor{black}{GitHub}}}
    %%%%%%%%%%%%%%%%%%%%%%%%%%%%%%%%%%%%%%%%%%%%%%%%%%%%%%%%%% }
    
%%%%%%%%%%%%%%%%% DO NOT CHANGE HERE %%%%%%%%%%%%%%%%%%%% {
    \pagestyle{fancyplain}
    \headheight 35pt
    \lhead{\NetIDa}
    \lhead{\NetIDa\\\NetIDb} 
    \chead{\textbf{\Large \rightmark}}
    \rhead{\course \\ \semester}
    \lfoot{}
    \cfoot{\thepage}
    \rfoot{}
    \headsep 1.5em
    \renewcommand{\sectionmark}[1]{\markright{#1}}
%%%%%%%%%%%%%%%%%%%%%%%%%%%%%%%%%%%%%%%%%%%%%%%%%%%%%%%%%% }

\begin{document}

% 工作流:
% 1. 复述附上图片的白板内容,并转为英文 latex,写入 L.tex。格式参考已有笔记的格式
% 2. 好的,继续来

\begin{titlepage}
    \centering
    \pagestyle{empty}
    
    \vspace*{3cm}

    {\Huge\bfseries Metric and Topological Spaces}\\[2cm]

    {\Large Notes}\\[0.5cm]
    {\Large \semester}\\[2cm]

    {\large Professor: Evgeny SMIRNOV}\\[2cm]
    
    {\Large \NetIDa}\\[0.5cm]
    {\Large \NetIDb}\\[2cm]
    
    \vfill

    {\large GTIIT, \today}
\end{titlepage}

% Week 1
\section{Notes 1 - 10.13}

\subsection{Metric Space}

\begin{definition}[Metric Space]
A metric space is a set \(X\) with a function \(d: X \times X \to \mathbb{R}_{\geq 0}\) (called a \textit{metric} or \textit{distance function}) satisfying the following axioms:
\begin{enumerate}
\item \textbf{Non-negativity}: \(d(x, y) \geq 0\), and \(d(x, y) = 0 \iff x = y\).
\item \textbf{Symmetry}: \(d(x, y) = d(y, x)\) for all \(x, y \in X\).
\item \textbf{Triangle Inequality}: \(d(x, y) + d(y, z) \geq d(x, z)\) for all \(x, y, z \in X\).
\end{enumerate}
\end{definition}

\begin{example}[Metric Space Examples]
\leavevmode
\begin{enumerate}
\item[\bfseries Ex 0.]
\(X = \mathbb{R}\), \(d(x, y) = |x - y|\) (standard absolute value distance).
        
\item[\bfseries Ex 1a.]
\(X\) is finite, \(d(x, y) = \begin{cases} 0 & x = y \\ 1 & x \neq y \end{cases}\) (discrete metric).
        
\item[\bfseries Ex 1b.]
\(X\) is finite, \(d(x, y) = \min\{\text{weight of a path } x \leftrightarrow y\}\) (generalization of the discrete metric, e.g., via a weighted graph).
        
\item[\bfseries Ex 2a.]
Euclidean metric in \(\mathbb{R}^n\): for \(x = (x_1, \cdots, x_n)\), \(y = (y_1, \cdots, y_n)\),
\[d(x, y) = \sqrt{(x_1 - y_1)^2 + \cdots + (x_n - y_n)^2}\]

\item[\bfseries Ex 2b.]
Manhattan metric in \(\mathbb{R}^n\):
\[d(x, y) = |x_1 - y_1| + \cdots + |x_n - y_n|\]

\item[\bfseries Ex 2c.]
\[d_\infty(x, y) = \max_{i=1, \cdots, n} |x_i - y_i|\]

\item[\bfseries Ex 3.]
\textbf{(Amazon Basin Metric).}
Let \(X = \mathbb{R}^2\), \(p = (x, y)\), \(p' = (x', y')\), then
\[d(p, p') = \begin{cases}
|x - x'| + |y| + |y'| & x \neq x' \\
|y - y'| & x = x'
\end{cases}\]

\item[\bfseries Ex 4a.]
Let \(X = C[0, 1]\) (continuous functions on \([0, 1]\)). The metric is
\[d(f, g) = \max_{x \in [0, 1]} |f(x) - g(x)|\]

\item[\bfseries Ex 4b.]
Let \(X = C[0, 1]\). The metric is
\[d(f, g) = \int_0^1 |f(x) - g(x)| dx\]
\end{enumerate}
\end{example}

\subsection{Normed Space}

\begin{definition}[Normed Space]
Let \(V\) be a vector space over \(\mathbb{R}\). A function \(\|\cdot\|: V \to \mathbb{R}_{\geq 0}\) is a norm if it satisfies:
\begin{enumerate}
\item \(\|v\| = 0 \iff v = 0\)
\item \(\|\lambda v\| = |\lambda| \|v\|\) for all \(v \in V\), \(\lambda \in \mathbb{R}\)
\item \(\|v + w\| \leq \|v\| + \|w\|\) (triangle inequality)
\end{enumerate}
\end{definition}

A norm defines a metric on \(V\) by \(d(x, y) \stackrel{\text{def}}{=} \|x - y\|\).

\begin{example}[Norm Examples]
\leavevmode
\begin{enumerate}
\item[\bfseries Ex 2a.] \textbf{(Euclidean Norm)}
For \(v = (x_1, \cdots, x_n) \in \mathbb{R}^n\):
\[\|v\| = \sqrt{x_1^2 + x_2^2 + \cdots + x_n^2}\]
        
\item[\bfseries Ex 2b.] \textbf{(Manhattan Norm)}
For \(v = (x_1, \cdots, x_n) \in \mathbb{R}^n\):
\[\|v\| = |x_1| + |x_2| + \cdots + |x_n|\]
        
\item[\bfseries Ex 2c.] \textbf{(Max Norm)}
For \(v = (x_1, \cdots, x_n) \in \mathbb{R}^n\):
\[\|v\| = \max_{i=1, \cdots, n} |x_i|\]
        
\item[\bfseries Ex 4a.] \textbf{(Supremum Norm)}
For \(f \in C[0, 1]\):
\[\|f\| = \max_{x \in [0, 1]} |f(x)|\]
        
\item[\bfseries Ex 4b.] \textbf{(L¹ Norm)}
For \(f \in C[0, 1]\):
\[\|f\| = \int_0^1 |f(x)| dx\]
\end{enumerate}
\end{example}

\subsubsection{\(l_p\)-Spaces (Sequences, \(1 \leq p \leq \infty\))}
\begin{itemize}
\item \textbf{a) \(l_\infty\)}: Space of bounded sequences. The norm is:
\[\|x\|_\infty = \sup_i |x_i|\]

\item \textbf{b) \(l_1\)}: Sequences satisfying \(\sum_{i=1}^\infty |x_i| < \infty\). The norm is:
\[\|x\|_1 = \sum_{i=1}^\infty |x_i|\]

\item \textbf{c) \(l_2\)}: Sequences satisfying \(\sum_{i=1}^\infty x_i^2 < \infty\). The norm is:
\[\|x\|_2 = \sqrt{\sum_{i=1}^\infty x_i^2}\]

\item \textbf{d) \(l_p, 1 < p < \infty\)}: The norm is:
\[\|x\|_p = \left( \sum_{i=1}^\infty |x_i|^p \right)^{\frac{1}{p}}\]
\end{itemize}

\paragraph*{Inclusion Property}
\(l_p \subset l_q\) if \(p < q\). For instance, \(x_n = \frac{1}{n}\) satisfies \(\sum \frac{1}{n} = \infty\) (not in \(l_1\)) but \(\sum \frac{1}{n^2} < \infty\) (in \(l_2\)).

\begin{theorem}
\(l_p\) spaces are normed spaces (satisfy the triangle inequality).
\end{theorem}

\subsection{Inner Product Space (Euclidean Space)}

\begin{definition}[Inner Product Space (Euclidean Space)]
A vector space \(V\) with an inner product \(\langle \cdot, \cdot \rangle: V \times V \to \mathbb{R}\) satisfying
\begin{enumerate}
\item \textbf{Symmetry}: \(\langle v, w \rangle = \langle w, v \rangle\).
\item \textbf{Linearity}: \(\langle \lambda v, w \rangle = \lambda \langle v, w \rangle\), \(\langle v + w, u \rangle = \langle v, u \rangle + \langle w, u \rangle\).
\item \textbf{Positive Definiteness}: \(\langle v, v \rangle \geq 0\) and \(\langle v, v \rangle = 0 \iff v = 0\).
\end{enumerate}
\end{definition}

An inner product defines a norm:
\[\|v\| = \sqrt{\langle v, v \rangle}\]

This norm satisfies the Cauchy-Schwarz inequality.

\begin{theorem}[Cauchy-Schwarz Inequality]
For all \(v, w \in V\) (a Euclidean space),
\[\langle v, w \rangle^2 \leq \|v\|^2 \|w\|^2 = \langle v, v \rangle \langle w, w \rangle\]

Equality holds if and only if \(v\) and \(w\) are proportional (\(v \sim w\)).
\end{theorem}

\begin{remark}
Proof is commented out!
\end{remark}

% \begin{proof}
% Consider \(w + tv \in V\) for \(t \in \mathbb{R}\). The inner product \(\langle tv + w, tv + w \rangle \geq 0\), so:
% \[0 \leq t^2 \langle v, v \rangle + 2t \langle v, w \rangle + \langle w, w \rangle\]

% This is a quadratic in \(t\); its discriminant \(D \leq 0\):
% \[D = (2\langle v, w \rangle)^2 - 4 \langle v, v \rangle \langle w, w \rangle \leq 0 \implies \langle v, w \rangle^2 \leq \langle v, v \rangle \langle w, w \rangle\]

% Equality occurs when \(D = 0\), that is, \(tv + w = 0\) for some \(t\), meaning \(v\) and \(w\) are proportional.
% \end{proof}

\begin{corollary}
\begin{enumerate}
\item For \(\|v\| = \sqrt{\langle v, v \rangle}\), the triangle inequality holds:
\[\|v + w\| \leq \|v\| + \|w\|\]
    
\item The \(l_2\)-norm satisfies the triangle inequality. \(l_2\) is a Euclidean space with inner product \(\langle x, y \rangle = \sum_{i=1}^\infty x_i y_i\) and norm \(\|x\|_2 = \sqrt{\langle x, x \rangle}\).
\end{enumerate}
\end{corollary}

\begin{remark}
The \(l_p\)-norm \(\|x\|_p\) does not come from an inner product when \(p \neq 2\).
\end{remark}

\newpage

\section{Notes 2 - 10.17}

\subsection{Recap}

\begin{recap}[Norm]
A norm \(\|\cdot\|: V \to \mathbb{R}\) satisfies:
\begin{enumerate}
\item \(\|x\| \geq 0\), and \(\|x\| = 0 \iff x = 0\)
\item \(\|\lambda x\| = |\lambda| \cdot \|x\|\) for all \(\lambda \in \mathbb{R}, x \in V\)
\item \(\|x + y\| \leq \|x\| + \|y\|\) (triangle inequality)
\end{enumerate}
\end{recap}

A norm defines a metric: \(d(x, y) = \|x - y\|\).

\begin{remark}[Relationship to Inner Product Spaces]
A positive-definite inner product \(\langle \cdot, \cdot \rangle\) defines a norm: \(\|x\| = \sqrt{\langle x, x \rangle}\). The triangle inequality follows from Cauchy-Schwarz. The hierarchy is:
\[\text{Inner Product Space} \implies \text{Normed Space} \implies \text{Metric Space}\]
\end{remark}

\begin{definition}[\(l_p\)-Spaces (Revisited)]
\begin{itemize}
\item \(l_p = \left\{ (x_1, x_2, \cdots) \mid x_i \in \mathbb{R}, \sum_{i=1}^\infty |x_i|^p \text{ converges} \right\}\) for \(1 \leq p < \infty\)
\item \(l_\infty = \left\{ (x_1, x_2, \cdots) \mid x_i \in \mathbb{R}, \text{ bounded} \right\}\)
\item Norms:
\[\|x\|_p = \left( \sum_{i=1}^\infty |x_i|^p \right)^{1/p} \ (1 \leq p < \infty), \ \|x\|_\infty = \max_i |x_i|\]
\item \(l_2\) is an inner product space with \(\langle x, y \rangle = \sum_{i=1}^\infty x_i y_i\) and \(\|x\|_2 = \sqrt{\langle x, x \rangle}\), satisfying the triangle inequality.
\end{itemize}
\end{definition}

\subsection{Important Inequalities}

\begin{theorem}[Minkowski Inequality]
For \(1 \leq p < \infty\) and \(x, y \in l_p\),
\[\|x + y\|_p \leq \|x\|_p + \|y\|_p\]
\end{theorem}

For \(p > 1\), let \(q > 1\) be its conjugate (\(\frac{1}{p} + \frac{1}{q} = 1\)).

\begin{theorem}[Hölder's Inequality]
For sequences \(a_1, \cdots, a_n\), \(b_1, \cdots, b_n \geq 0\), and \(p, q > 1\) with \(\frac{1}{p} + \frac{1}{q} = 1\),
\[\sum_{k=1}^n a_k b_k \leq \left( \sum_{k=1}^n a_k^p \right)^{\frac{1}{p}} \left( \sum_{k=1}^n b_k^q \right)^{\frac{1}{q}}\]
\end{theorem}

\begin{remark}
Proof is commented out!
\end{remark}

% \begin{proof}[Proof of Minkowski Inequality for Finite Sequences]
% Let \(x = (x_1, \cdots, x_n)\), \(y = (y_1, \cdots, y_n)\), and \(1 \leq p < \infty\). We aim to show:
% \[\left( \sum_{k=1}^n |x_k + y_k|^p \right)^{\frac{1}{p}} \leq \left( \sum_{k=1}^n |x_k|^p \right)^{\frac{1}{p}} + \left( \sum_{k=1}^n |y_k|^p \right)^{\frac{1}{p}}\]

% \paragraph*{Step 1: Decompose \(|x_k + y_k|^p\)}
% \[|x_k + y_k|^p = |x_k + y_k|^{p-1} |x_k + y_k| \leq |x_k + y_k|^{p-1} (|x_k| + |y_k|)\]

% \paragraph*{Step 2: Apply Hölder's Inequality to each term}
% \begin{itemize}
% \item For \(\sum |x_k + y_k|^{p-1} |x_k|\): Let \(a_k = |x_k|\), \(b_k = |x_k + y_k|^{p-1}\), \(q = \frac{p}{p-1}\) (so \(\frac{1}{p} + \frac{1}{q} = 1\)). Then
% \[\sum |x_k + y_k|^{p-1} |x_k| \leq \left( \sum |x_k|^p \right)^{\frac{1}{p}} \left( \sum |x_k + y_k|^p \right)^{\frac{1}{q}}\]
% \item For \(\sum |x_k + y_k|^{p-1} |y_k|\): Similarly,
% \[\sum |x_k + y_k|^{p-1} |y_k| \leq \left( \sum |y_k|^p \right)^{\frac{1}{p}} \left( \sum |x_k + y_k|^p \right)^{\frac{1}{q}}\]
% \end{itemize}

% \paragraph*{Step 3: Combine the inequalities}
% Let \(A = \left( \sum |x_k + y_k|^p \right)^{\frac{1}{p}}\). Summing the two inequalities gives:
% \[A^p \leq \left( \sum |x_k|^p \right)^{\frac{1}{p}} A^{\frac{p}{q}} + \left( \sum |y_k|^p \right)^{\frac{1}{p}} A^{\frac{p}{q}}\]

% Since \(\frac{p}{q} = p - 1\), divide both sides by \(A^{p-1}\) (for \(A \neq 0\)):
% \[A \leq \left( \sum |x_k|^p \right)^{\frac{1}{p}} + \left( \sum |y_k|^p \right)^{\frac{1}{p}}\]
% \end{proof}

\begin{theorem}[Young's Inequality]
For \(a, b \geq 0\) and \(p, q > 1\) with \(\frac{1}{p} + \frac{1}{q} = 1\),
\[ab \leq \frac{a^p}{p} + \frac{b^q}{q}\]
\end{theorem}

\begin{remark}
Proof is commented out!
\end{remark}

% \begin{proof}
% Since \(\ln x\) is concave, by Jensen's inequality:
% \[\ln a + \ln b = \frac{1}{p} \ln a^p + \frac{1}{q} \ln b^q \leq \ln\left( \frac{a^p}{p} + \frac{b^q}{q} \right)\]

% Exponentiating both sides gives the inequality. A special case is \(ab \leq \frac{a^2 + b^2}{2}\) (when \(p = q = 2\)).
% \end{proof}

\textbf{Hölder's Inequality (Alternative Proof)}
For sequences \(a_1, \cdots, a_n\), \(b_1, \cdots, b_n \geq 0\) and \(p, q > 1\) with \(\frac{1}{p} + \frac{1}{q} = 1\),
\[\sum_{k=1}^n a_k b_k \leq \left( \sum_{k=1}^n a_k^p \right)^{\frac{1}{p}} \left( \sum_{k=1}^n b_k^q \right)^{\frac{1}{q}}\]

\begin{remark}
Proof is commented out!
\end{remark}

% \begin{proof}[Proof (from Young's Inequality)]
% Let \(A = \left( \sum_{k=1}^n a_k^p \right)^{\frac{1}{p}}\), \(B = \left( \sum_{k=1}^n b_k^q \right)^{\frac{1}{q}}\). Define \(u_k = \frac{a_k}{A}\), \(v_k = \frac{b_k}{B}\), so \(\sum_{k=1}^n u_k^p = 1\) and \(\sum_{k=1}^n v_k^q = 1\). By Young's Inequality:
% \[u_k v_k \leq \frac{u_k^p}{p} + \frac{v_k^q}{q}\]

% Summing over \(k\):
% \[\sum_{k=1}^n u_k v_k \leq \frac{1}{p} \sum_{k=1}^n u_k^p + \frac{1}{q} \sum_{k=1}^n v_k^q = \frac{1}{p} + \frac{1}{q} = 1\]

% Substituting back \(u_k, v_k\):
% \[\sum_{k=1}^n \frac{a_k b_k}{AB} \leq 1 \implies \sum_{k=1}^n a_k b_k \leq AB = \left( \sum_{k=1}^n a_k^p \right)^{\frac{1}{p}} \left( \sum_{k=1}^n b_k^q \right)^{\frac{1}{q}}\]
% \end{proof}

% Tutorial
\subsection{Metric Space and Neighborhoods}

Let \(X\) be a metric space.

\begin{definition}[Neighborhoods]
\textbf{Neighborhood}: For \(x \in X\) and \(\varepsilon > 0\), \(U_\varepsilon(x) = \{ y \in X \mid d(x, y) < \varepsilon \}\).

\textbf{Punctured Neighborhood}: \(\cdot{U}_\varepsilon(x) = \{ y \in X \mid 0 < d(x, y) < \varepsilon \} = U_\varepsilon(x) \setminus \{ x \}\).
\end{definition}

\begin{definition}
A subset \(M \subset X\) is open if \(\forall x \in M\), \(\exists \varepsilon > 0\) such that \(U_\varepsilon(x) \subset M\). \(\varnothing\) and \(X\) are open by definition.
\end{definition}

\begin{theorem}[Theorems on Open Sets]
\begin{enumerate}
\item The intersection of finitely many open sets \(U_1 \cap \cdots \cap U_k\) is open.
\item The union of any collection of open sets \(\bigcup U_\alpha\) is open.
\end{enumerate}
\end{theorem}

\begin{remark}
Proof is commented out!
\end{remark}

% \begin{proof}
% \begin{enumerate}
% \item
% Let \(V = \bigcap_{i=1}^k U_i\). If \(V = \emptyset\), it is open.

% Otherwise, let \(x \in V\), then \(x \in U_i\) for all \(i = 1, \cdots, k\).

% Since each \(U_i\) is open, there exists \(\varepsilon_i > 0\) such that \(U_{\varepsilon_i}(x) \subset U_i\).

% Take \(\varepsilon = \min\{\varepsilon_1, \cdots, \varepsilon_k\} > 0\), then \(U_\varepsilon(x) \subset U_{\varepsilon_i}(x) \subset U_i\) for all \(i\).

% Thus, \(U_\varepsilon(x) \subset V \implies V\) is open.

% \item
% Let \(U = \bigcup_\alpha U_\alpha\). For any \(x \in U\), there exists some \(\alpha\) such that \(x \in U_\alpha\).

% Since \(U_\alpha\) is open, there exists \(\varepsilon > 0\) such that \(U_\varepsilon(x) \subset U_\alpha\).

% Since \(U_\alpha \subset U\), then \(U_\varepsilon(x) \subset U \implies U\) is open.
% \end{enumerate}
% \end{proof}

\begin{example}[Questions and Examples]
\leavevmode
\begin{enumerate}
\item[\bfseries Qu 1.] Is \(U_R(y) \subset U_r(x)\) possible for \(r < R\)?
    
Example: Let \(X\) be a disk of radius 1 (\(U_1(0) = X\)), then \(B_{\frac{3}{2}}\left( \frac{3}{4} \right) \subset B_1(0)\), so yes.

\item[\bfseries Qu 2.] Draw balls centered at 0 in \(\mathbb{R}^2\) for norms \(\|\cdot\|_1\), \(\|\cdot\|_2\), \(\|\cdot\|_\infty\).

\item[\bfseries Qu 3.] For the ``Amazon metric'' on \(\mathbb{R}^2\):
\[d((x_1,y_1),(x_2,y_2)) = \begin{cases} |y_1 - y_2| & x_1 = x_2 \\ |y_1| + |x_1 - x_2| + |y_2| & x_1 \neq x_2 \end{cases}\]

Describe balls centered at (0,0) and (0,1).
\end{enumerate}
\end{example}

\newpage

% Week 2
\section{Notes 3 - 10.20}

\subsection{Interior Points and Open Sets}

\begin{definition}
Let \((X, d)\) be a metric space, \(M \subset X\):
\begin{itemize}
\item \textbf{Interior Point}: \(x \in M\) is an interior point if \(\exists \varepsilon > 0\) such that \(U_\varepsilon(x) \subset M\).
\item \textbf{Interior of \(M\)}: \(\text{Int } M = \{\text{interior points of } M\}\).
\item \textbf{Open Set (Alternative Definition)}: \(M\) is open if \(\text{Int } M = M\).
\end{itemize}
\end{definition}

\begin{proposition}[Properties of Open Sets]
\begin{enumerate}
\item The intersection of finitely many open sets \(U_1 \cap \cdots \cap U_k\) is open.
\item The union of any collection of open sets \(\bigcup U_\alpha\) is open.
\end{enumerate}
\end{proposition}

\begin{theorem}
\(U \subset \mathbb{R}\) is open if and only if \(U = \bigcup (a_i, b_i)\), where:
\begin{itemize}
\item The union is at most countable;
\item \(a_i \in \{-\infty\} \cup \mathbb{R}\), \(b_i \in \mathbb{R} \cup \{+\infty\}\);
\item \((a_i, b_i) \cap (a_j, b_j) = \varnothing\) for \(i \neq j\).
\end{itemize}
\end{theorem}

\begin{example}[Discrete Metric]
For \((X, d)\) with discrete metric \(d(x,y) = \begin{cases} 1 & x \neq y \\ 0 & x = y \end{cases}\):
\begin{itemize}
\item \(\{x\}\) is open (since \(U_{1/2}(x) = \{x\} \subset \{x\}\));
\item Every subset of \(X\) is open.
\end{itemize}
\end{example}

\begin{remark}[Remarks for \(\mathbb{R}^n\)]
\begin{itemize}
\item Sets defined by \textbf{strict inequalities} (e.g., \(\{x \in \mathbb{R}^n \mid x_i > 0\}\), \(\{ (x,y) \mid x^2 + y > 3x - 2 \}\)) are open.
\item Example: \(X = \mathbb{R}^n \cong \text{Mat}(n)\) (matrices). The set \(\{ A \in \text{Mat}(n) \mid \det A \neq 0 \}\) (non-degenerate matrices) is open. For \(A = (a_{ij})\), the neighborhood:
\[U_\varepsilon(A) = \left\{ B = (a_{ij} + \varepsilon_{ij}) \mid |\varepsilon_{ij}| < \varepsilon \right\} = \{ B \mid d(A, B) < \varepsilon \}\]

Choosing \(\varepsilon < \|\det A\|\) ensures \(\det B \neq 0\) (via expansion of \(\det B\)).
\end{itemize}
\end{remark}

\begin{note}[Estimate for Determinant Difference]
For \(A \in \text{Mat}(n)\) (\(\det A \neq 0\)) and \(B = A + E\) (\(\|E\| < \varepsilon\)):
\begin{itemize}
\item The determinant difference satisfies:
\[|\det B - \det A| < (2^n - 1) n! \|A\|^{n-1} \varepsilon\]
\item Choose \(\varepsilon < \frac{|\det A|}{(2^n - 1) n! \|A\|^{n-1}}\): this ensures \(\det B \neq 0\), so \(\{ A \mid \det A \neq 0 \}\) is open.
\end{itemize}
\end{note}

\subsection{Closed Sets and Closure}

\begin{definition}
Let \((X, d)\) be a metric space, \(M \subset X\):
\begin{itemize}
\item \textbf{Limit Point}: \(x \in X\) is a limit point of \(M\) if \(\forall \varepsilon > 0\), \(\cdot{U}_\varepsilon(x) \cap M \neq \varnothing\) (\(\cdot{U}_\varepsilon(x) = U_\varepsilon(x) \setminus \{x\}\)). Equivalently, \(U_\varepsilon(x)\) contains infinitely many points of \(M\).
\item \textbf{Isolated Point}: \(x \in M\) is isolated if \(\exists \varepsilon > 0\) such that \(U_\varepsilon(x) \cap M = \{x\}\).
\item \textbf{Closure}: \(\overline{M} = M \cup \{\text{limit points of } M\}\) (union of \(M\) and its limit points).
\item \textbf{Closed Set}: \(M\) is \textbf{closed} if \(\overline{M} = M\) (contains all its limit points).
\item \textbf{Property}: \(M\) is closed \(\iff X \setminus M\) is open (open sets have closed complements, and vice versa).
\item \textbf{Example}: Prove \(\overline{\overline{M}} = \overline{M}\) (closure of closure is closure).
\end{itemize}
\end{definition}

\begin{proposition}[Characterization of Closed Sets]
A subset \(M \subset X\) is \textbf{closed} if and only if
\[\forall y \in X \setminus M, \exists \varepsilon > 0 \text{ such that } U_\varepsilon(y) \cap M = \varnothing\]
(Any point not in \(M\) can be separated from \(M\).)
\end{proposition}

\begin{remark}
Proof is commented out!
\end{remark}

% \begin{proof}
% \begin{itemize}
% \item[\(\Rightarrow\)]
% Suppose \(M\) is closed. If \(y \in X \setminus M\) cannot be separated from \(M\), \(\forall \varepsilon > 0, U_\varepsilon(y) \cap M \neq \varnothing\), so \(y\) is a limit point of \(M\). Thus, \(y \in \overline{M} = M\) (contradiction).

% \item[\(\Leftarrow\)]
% Suppose every \(y \in X \setminus M\) can be separated from \(M\). Let \(x\) be a limit point of \(M\): if \(x \notin M\), \(\exists \varepsilon > 0\) with \(U_\varepsilon(x) \cap M = \varnothing\) (contradicts \(x\) being a limit point).

% Thus, \(x \in M\), so \(M\) is closed.
% \end{itemize}
% \end{proof}

\subsection{Continuous Maps Between Metric Spaces}

\begin{definition}[Continuous Maps]
Let \(f: (X_1, d_1) \to (X_2, d_2)\) be a function.
\begin{itemize}
\item \textbf{Definition 1}: \(f\) is \textbf{continuous} if
\[\forall x \in X_1, \forall \varepsilon > 0, \exists \delta > 0 \text{ such that } \forall x' \in X_1, d_1(x, x') < \delta \implies d_2(f(x), f(x')) < \varepsilon\]
\item \textbf{Definition 2}: \(f\) is \textbf{continuous} if for every open set \(U \subset X_2\), its preimage \(f^{-1}(U) = \{ x \in X_1 \mid f(x) \in U \}\) is open in \(X_1\).
\end{itemize}
\end{definition}

\begin{theorem}
Definition 1 \(\iff\) Definition 2.
\end{theorem}

\begin{remark}
Proof is commented out!
\end{remark}

% \begin{proof}[Proof Sketch (\(\Rightarrow\))]
% Let \(f\) be continuous (Definition 1). For open \(U \subset X_2\), take \(y \in f^{-1}(U)\) (so \(f(y) \in U\)). Since \(U\) is open, \(\exists \varepsilon > 0\) with \(U_\varepsilon(f(y)) \subset U\). By continuity, \(\exists \delta > 0\) such that \(d_1(y, x') < \delta \implies d_2(f(y), f(x')) < \varepsilon\), so \(U_\delta(y) \subset f^{-1}(U)\). Thus \(f^{-1}(U)\) is open.
% \end{proof}

% \begin{proof}[Full Proof of Equivalence]
% Let \(f: (X_1, d_1) \to (X_2, d_2)\). We prove Definition 1 (\(\varepsilon\)-\(\delta\)) \(\iff\) Definition 2 (open preimages).

% \noindent \textbf{\(\Rightarrow\)}
% To show \(f^{-1}(U)\) is open (for open \(U \subset X_2\)):
% \begin{enumerate}
% \item Take \(x \in f^{-1}(U)\), so \(f(x) = y \in U\).
% \item Since \(U\) is open, \(\exists \varepsilon > 0\) with \(U_\varepsilon(y) \subset U\).
% \item By Definition 1, \(\exists \delta > 0\) such that \(d_1(x, x') < \delta \implies d_2(f(x'), f(x)) < \varepsilon\).
% \item Thus \(U_\delta(x) \subset f^{-1}(U_\varepsilon(y)) \subset f^{-1}(U)\), so \(x\) is an interior point of \(f^{-1}(U)\).
% \item This holds for all \(x \in f^{-1}(U)\), so \(f^{-1}(U)\) is open.
% \end{enumerate}

% \noindent \textbf{\(\Leftarrow\)}
% To show the \(\varepsilon\)-\(\delta\) condition:
% \begin{enumerate}
% \item Take \(x \in X_1\), \(y = f(x)\), fix \(\varepsilon > 0\).
% \item \(U_\varepsilon(y) \subset X_2\) is open, so \(f^{-1}(U_\varepsilon(y))\) is open (by Definition 2).
% \item Since \(x \in f^{-1}(U_\varepsilon(y))\), \(\exists \delta > 0\) with \(U_\delta(x) \subset f^{-1}(U_\varepsilon(y))\).
% \item Thus \(d_1(x, x') < \delta \implies f(x') \in U_\varepsilon(y) \implies d_2(f(x'), f(x)) < \varepsilon\).
% \end{enumerate}
% \end{proof}

\begin{example}[Inverse of Continuous Bijection Need Not Be Continuous]
Let \(f: X \to Y\) be continuous and bijective. \(f^{-1}: Y \to X\) is not necessarily continuous:
\begin{enumerate}
\item[\bfseries Ex 1.]
\begin{itemize}
\item \(X = [0, 1]\) (discrete metric: \(d(x,x') = \begin{cases} 0 & x=x' \\ 1 & x \neq x' \end{cases}\)).
\item \(Y = [0, 1]\) (standard Euclidean metric).
\item \(f: X \to Y\), \(f(x) = x\):
\begin{itemize}
\item \(f\) is continuous (all subsets of \(X\) are open, so preimages of open sets in \(Y\) are open).
\item \(f^{-1}: Y \to X\), \(f^{-1}(y) = y\):
\begin{itemize}
\item \(\{ \frac{1}{2} \} \subset X\) is open, but \((f^{-1})^{-1}(\{ \frac{1}{2} \}) = \{ \frac{1}{2} \} \subset Y\) is not open (Euclidean metric). Thus \(f^{-1}\) is discontinuous.
\end{itemize}
\end{itemize}
\end{itemize}

\item[\bfseries Ex 2.]
\begin{itemize}
\item \(X = \{0\} \cup (1, 2]\).
\item \(Y = [0, 1]\).
\item \(f: X \to Y\): \(f(0) = 0\), \(f(x) = x-1\) for \(x > 0\):
\begin{itemize}
\item \(f\) is continuous and bijective.
\item \(f^{-1}(0) = \{0\} \subset X\) is open, but \((f^{-1})^{-1}(\{0\}) = \{0\} \subset Y\) is not open (Euclidean metric). Thus \(f^{-1}\) is discontinuous.
\end{itemize}
\end{itemize}
\end{enumerate}
\end{example}

\newpage

\section{Notes 4 - 10.24}

\subsection{Homeomorphisms and Isometries}

\begin{recap}[Continuous Map]
\(f: X \to Y\) is continuous if \(f^{-1}(U) \subset X\) is open for every open \(U \subset Y\).
\end{recap}

\begin{note}
\textbf{Key Note}: A continuous bijection \(f: X \to Y\) does \textbf{not} ensure \(f^{-1}\) is continuous.
\end{note}

\begin{definition}[Homeomorphism]
A homeomorphism is a continuous bijection \(f: X \to Y\) where \(f^{-1}\) is also continuous. \(X \cong Y\) (topologically equivalent) if there exists a homeomorphism \(f: X \to Y\).
\end{definition}

\begin{definition}[Isometry]
\((X, d)\) and \((Y, d')\) are isometric if there exists a bijection \(f: X \to Y\) (an isometry) such that
\[d'(f(x), f(y)) = d(x, y) \ \forall x, y \in X\]
\end{definition}

\begin{remark}[Relationship]
Every isometry is a homeomorphism (isometries are continuous, bijective, and \(f^{-1}\) is also an isometry/continuous). However, not all homeomorphisms are isometries (homeomorphisms preserve topology, not necessarily metric distances).
\end{remark}

\begin{example}
Let \(D = \{ (x,y) \mid x^2 + y^2 \leq 1 \}\) (closed disk, Euclidean metric) and \(S = \{ (x,y) \mid \max(|x|, |y|) \leq 1 \}\) (closed square, \(\|\cdot\|_\infty\) metric). \(D\) and \(S\) are homeomorphic but not isometric: For \(p_1, p_2 \in D\) with \(d(p_1, p_2) = \sqrt{2}\), their images in \(S\) satisfy \(d'(f(p_1), f(p_2)) \leq 2\), so distances are not preserved.
\end{example}

\subsection{Properties of Homeomorphisms}

\begin{proposition}[Composition]
If \(f: X \to Y\) and \(g: Y \to Z\) are continuous, then \(g \circ f: X \to Z\) (where \((g \circ f)(x) = g(f(x))\)) is continuous.
\end{proposition}

\begin{remark}
Proof is commented out!
\end{remark}

% \begin{proof}
% \begin{enumerate}
% \item Let \(U \subset Z\) be open.
% \item Since \(g\) is continuous, \(g^{-1}(U) \subset Y\) is open.
% \item Since \(f\) is continuous, \(f^{-1}(g^{-1}(U)) = (g \circ f)^{-1}(U) \subset X\) is open.
% \item Thus \(g \circ f\) is continuous (open preimage definition).
% \end{enumerate}
% \end{proof}

\begin{corollary}
If \(f: X \to Y\) and \(g: Y \to Z\) are homeomorphisms, then \(g \circ f: X \to Z\) is a homeomorphism.
\((g \circ f)^{-1} = f^{-1} \circ g^{-1}\) is continuous because composition of continuous maps is continuous.
\end{corollary}

\begin{proposition}[Equivalence Relation]
The relation \(\cong\) (is homeomorphic to) is an equivalence relation on metric spaces.
\end{proposition}

\begin{remark}
Proof is commented out!
\end{remark}

% \begin{proof}
% \begin{enumerate}
% \item \textbf{Reflexive}: \(X \cong X\) (identity map \(\text{Id}_X: X \to X\) is a homeomorphism).
% \item \textbf{Symmetric}: If \(X \cong Y\) (via homeomorphism \(f: X \to Y\)), then \(Y \cong X\) (via \(f^{-1}: Y \to X\), a homeomorphism).
% \item \textbf{Transitive}: If \(X \cong Y\) (via \(f\)) and \(Y \cong Z\) (via \(g\)), then \(X \cong Z\) (via \(g \circ f\), a homeomorphism: bijective, continuous, inverse \(f^{-1} \circ g^{-1}\) is continuous).
% \end{enumerate}
% \end{proof}

\subsection{Path-Connected Spaces and IVT}

\begin{definition}[Path-Connected]
\(X\) is said to be \textbf{path-connected} if any two points \(x, y \in X\) can be joined by a path: \(\exists \gamma: [0,1] \to X\), with \(\gamma(0) = x\), \(\gamma(1) = y\), and \(\gamma\) continuous.
\end{definition}

\begin{proposition}
If \(X\) is path-connected and \(f: X \to Y\) is a homeomorphism, then \(Y\) is path-connected.
\end{proposition}

\begin{remark}
Proof is commented out!
\end{remark}

% \begin{proof}
% For \(f(x), f(y) \in Y\), let \(\gamma: [0,1] \to X\) be a path from \(x\) to \(y\). Then \(f \circ \gamma: [0,1] \to Y\) is continuous, with \((f \circ \gamma)(0) = f(x)\) and \((f \circ \gamma)(1) = f(y)\). Thus \(Y\) is path-connected.
% \end{proof}

\begin{example}[\(\mathbb{R} \setminus \{0\}\) is Not Path-Connected]
\begin{proof}[Proof by Contradiction]
Suppose \(\mathbb{R} \setminus \{0\}\) is path-connected. Then \(\exists\) continuous \(f: [0,1] \to \mathbb{R} \setminus \{0\}\) with \(f(0) = -1\), \(f(1) = 1\).

By the \textbf{Intermediate Value Theorem (IVT)}: For continuous \(f: [a,b] \to \mathbb{R}\), if \(f(a) < f(b)\), then \(\forall y \in [f(a), f(b)]\), \(\exists c \in [a,b]\) such that \(f(c) = y\).

Applying IVT to \(f: [0,1] \to \mathbb{R}\), \(\exists t \in [0,1]\) with \(f(t) = 0\). But \(f(t) \in \mathbb{R} \setminus \{0\}\) (contradiction). Thus \(\mathbb{R} \setminus \{0\}\) is not path-connected.
\end{proof}
\end{example}

\begin{proposition}[IVT for Path-Connected Spaces]
For a path-connected space \(X\), if \(f: X \to \mathbb{R}\) is continuous, then \(f(X)\) is an interval: If \(\exists x_1, x_2 \in X\) with \(f(x_1) = y_1\), \(f(x_2) = y_2\), then \(\forall y_3 \in [y_1, y_2]\), \(\exists x_3 \in X\) such that \(f(x_3) = y_3\).
\end{proposition}

\begin{remark}
Proof is commented out!
\end{remark}

% \begin{proof}
% \begin{enumerate}
% \item Since \(X\) is path-connected, \(\exists\) continuous \(\gamma: [0,1] \to X\) with \(\gamma(0) = x_1\), \(\gamma(1) = x_2\).
% \item \(f \circ \gamma: [0,1] \to \mathbb{R}\) is continuous (composition of continuous maps).
% \item By IVT (for \([0,1]\)), \(\forall y_3 \in [y_1, y_2]\), \(\exists t_0 \in [0,1]\) with \((f \circ \gamma)(t_0) = y_3\).
% \item Let \(x_3 = \gamma(t_0) \in X\): then \(f(x_3) = y_3\).
% \end{enumerate}
% \end{proof}

\subsection{Connected Spaces}

\begin{definition}[Connected Space]
\(X\) is said to be \textbf{connected} if
\begin{enumerate}
\item \(X\) cannot be represented as \(X = U_1 \cup U_2\) with \(U_1, U_2 \neq \varnothing\), \(U_1, U_2\) open, and \(U_1 \cap U_2 = \varnothing\).
\item \(X\) cannot be represented as \(X = V_1 \cup V_2\) with \(V_1, V_2 \neq \varnothing\), \(V_1, V_2\) closed, and \(V_1 \cap V_2 = \varnothing\).
\item There is no proper nonempty subset \(U \subset X\) which is both open and closed (clopen).
\end{enumerate}
\end{definition}

\begin{proposition}
\(X\) is connected \(\iff\)
\begin{itemize}
\item there is no continuous surjective map \(X \to \{0, 1\}\).
\item \(X\) satisfies the Intermediate Value Theorem (IVT).
\end{itemize}
\end{proposition}

\begin{remark}
\begin{itemize}
\item \(X\) path connected \(\implies X\) connected.
\item The converse is \textbf{not} true.
\end{itemize}
\end{remark}

\begin{example}[Connected but Not Path-Connected Space]
Let \(X \subset \mathbb{R}^2\):
\[X = \left\{ \left(x, \sin\frac{1}{x}\right) \mid x > 0 \right\} \cup \left( \{0\} \times [-1,1] \right)\]

\begin{itemize}
\item \(X\) is \textbf{connected}: The curve accumulates on \(\{0\} \times [-1,1]\), so \(X\) cannot split into disjoint non-empty open sets.
\item \(X\) is \textbf{not path-connected}: No continuous path exists between \(\{0\} \times [-1,1]\) and \(\left(x, \sin\frac{1}{x}\right)\) (due to rapid oscillation of \(\sin\frac{1}{x}\)).
\end{itemize}
\end{example}

\newpage

% Week 3
\section{Notes 5 - 10.27}

\subsection{Connectedness (Recap and Equivalences)}

\begin{recap}[Path-Connectedness]
A space \(X\) is path-connected if \(\forall x,y \in X\), \(\exists\) continuous \(\gamma: [0,1] \to X\) with \(\gamma(0)=x\), \(\gamma(1)=y\).
\end{recap}

\begin{theorem}[Def-Thm]
The following are equivalent:
\begin{enumerate}
    \item \textbf{IVT}: \(\forall f: X \to \mathbb{R}\) continuous, s.t. \(f(x)=a, f(y)=b, a < b\), \(\forall c \in [a,b], \exists z \in X\) s.t. \(f(z)=c\).
    \item There is no continuous map \(f: X \to \{0,1\}\), \(f(x)=0, f(y)=1\) for some \(x, y \in X\) (surjective).
    \item 
    \begin{enumerate}
        \item[(3a)] \(X\) cannot be presented as a disjoint union of two open nonempty sets. \(X = U_1 \cup U_2, U_1, U_2 \text{ open}, U_1 \cap U_2 = \emptyset \implies U_1 = \emptyset \text{ or } U_2 = \emptyset\).
        \item[(3b)] If \(X = V_1 \cup V_2\), \(V_1, V_2\) closed, \(V_1 \cap V_2 = \emptyset\), then \(V_1 = \emptyset\) or \(V_2 = \emptyset\).
        \item[(3c)] If \(U \subset X\) is both \underline{open and closed}, then \(U = \emptyset\) or \(U = X\) (``clopen set'').
    \end{enumerate}
\end{enumerate}
\end{theorem}

\begin{remark}[Path-Connected \(\implies\) Connected]
All path-connected spaces are connected, but connected spaces need not be path-connected (e.g., the topologist's sine curve).
\end{remark}

\begin{example}[Topologist's Sine Curve]
Let \(X = \left\{ \left(x, \sin\frac{1}{x}\right) \mid x > 0 \right\} \cup \left( \{0\} \times [-1,1] \right) \subset \mathbb{R}^2\).

\begin{proof}[\textbf{\(X\) is Not Path-Connected}]
Let \(p_1 = (0,0)\) and \(p_2 = (1, \sin 1)\).

Suppose there exists a continuous path \(\gamma: [0,1] \to X\) such that \(\gamma(0)=p_1\) and \(\gamma(1)=p_2\).

Let \(\gamma(t) = (x(t), y(t))\), where \(x, y: [0,1] \to \mathbb{R}\) are continuous functions.

Let \(U = \{ \tau \in [0,1] \mid x(\tau) = 0 \} = x^{-1}(\{0\})\).

Then \(U = x^{-1}(0)\) is closed (the preimage of a closed set \(\{0\}\)).

Thus, \(\sup U \in U\) and \(1 \notin U \implies \sup U = t_0 < 1\).

For \(t > t_0\), we have \(x(t) > 0\), so \(\varphi(t) = \frac{1}{x(t)}\) is well-defined. (As \(t \to t_0^+\), \(\varphi(t) \to +\infty\).)

The \(y\)-coordinate for \(t > t_0\) is \(y(t) = \sin \frac{1}{x(t)} = \sin \varphi(t)\).

Now, consider any interval \((t_0, t_0 + \varepsilon)\). Since \(\varphi(t) \to \infty\), this interval contains infinitely many points where \(\sin \varphi(t) = 1\) and where \(\sin \varphi(t) = -1\). This implies \(y(t)\) does not converge to \(y(t_0) = 0\) as \(t \to t_0^+\), contradicting the continuity of \(\gamma\).
\end{proof}

\textbf{\(X\) is Connected}:
Let \(X = Y \cup (\{0\} \times [-1,1])\) and \(Y = \left\{ \left(x, \sin\frac{1}{x}\right) \mid x > 0 \right\}\). Note that \(Y\) is connected (and even path-connected).
Furthermore, \(X = \overline{Y}\).
\end{example}

\begin{proposition}[Fact]
If \(A\) is a connected, then \(\overline{A}\) is also connected (in \(\mathbb{R}^2\) or other metric spaces).
\end{proposition}

\begin{remark}
Proof is commented out!
\end{remark}

% \begin{proof}
% Assume that \(\overline{A} = U_1 \cup U_2\), where \(U_1, U_2\) are open in \(\overline{A}\) and \(U_1 \cap U_2 = \varnothing\).

% Consider the sets \(A_1 = U_1 \cap A\) and \(A_2 = U_2 \cap A\).

% We have \(A = A_1 \cup A_2\), \(A_1, A_2\) are open in \(A\) and \(A_1 \cap A_2 = \varnothing\).

% Since \(A\) is connected, one of these sets must be empty.

% Let \(A_1 = U_1 \cap A = \varnothing\), then \(U_1 \subset \overline{A} \setminus A\). However, \(U_1\) is a non-empty open subset of \(\overline{A}\). 

% By definition, every non-empty open set in \(\overline{A}\) must intersect \(A\), which contradicts \(U_1 \cap A = \varnothing\)!
% \end{proof}

\subsection{Connected Components}

\begin{definition}[Path-Connected Components]
For any \(x, y \in X\), say that \(x \sim y\)
if \(\exists \gamma: [0,1] \to X, \gamma(0)=x, \gamma(1)=y\).

This is an \textbf{equivalence relation}:
\begin{itemize}
    \item \textbf{Reflexive}: \(x \sim x\) (\(\gamma(t)=x \forall t\)).
    \item \textbf{Symmetric}: If \(x \sim y\), \(y \sim x\) (\(\tilde{\gamma}(t) = \gamma(1-t)\)).
    \item \textbf{Transitive}: If \(x \sim y, y \sim z\), \(x \sim z\) (concatenate paths):
    \[\gamma(t) = \begin{cases}
        \gamma_1(2t) & t \leq \frac{1}{2} \\
        \gamma_2(2t-1) & t \geq \frac{1}{2}
    \end{cases}\]
    (\(\gamma_1: x \to y\), \(\gamma_2: y \to z\)).
\end{itemize}

The \textbf{connected components} (path-connected components) of \(X\) are the equivalence classes of \(\sim\).
\end{definition}

\begin{definition}[General Connected Components]
Connected components (for general connectedness) are maximal connected subsets of \(X\) (each is connected, and no larger connected subset contains it).
\end{definition}

\begin{remark}[Invariance Under Homeomorphism]
The number of connected components is preserved under homeomorphisms (\(X \cong Y \implies X,Y\) have the same number of components).
\end{remark}

\begin{example}[Examples of Components]
    \leavevmode
    \begin{enumerate}
        \item[\bfseries Ex 1.] \((0,1) \cup (2,3) \not\cong (0,1)\)

        \item[\bfseries Ex 2.] \([0,1] \not\cong [0,1] \times [0,1]\)
    
        Removing a non-boundary point makes \([0,1]\) disconnected (\([0,1] \setminus \{pt\}\) is disconnected), but \([0,1] \times [0,1] \setminus \{pt\}\) is connected for any points.
    
        \item[\bfseries Ex 3.] \([0,1] \not\cong (0,1) \not\cong [0,1)\)
    \end{enumerate}
\end{example}

\subsection{Equivalent Norms on \(\mathbb{R}^n\)}

\begin{example}[Norms on \(\mathbb{R}^2\)]
For \(v = (x,y) \in \mathbb{R}^2\), common norms:
\[\|v\|_1 = |x| + |y|, \ \|v\|_2 = \sqrt{x^2 + y^2}, \ \|v\|_\infty = \max(|x|, |y|)\]

These induce diamond-shaped, circular, and square-shaped balls, respectively.
\end{example}

\begin{proposition}[Same Topology]
A subset \(U \subset \mathbb{R}^2\) is open in one of these metrics \(\iff\) it is open in the others (they define the \textbf{same topology}).

\paragraph*{Reason (Ball Inclusions)}
For \(\varepsilon > 0\):
\[U_\varepsilon^{(2)} \subset U_\varepsilon^{(1)} \subset U_\varepsilon^{(\infty)}, \ U_{\varepsilon/\sqrt{2}}^{(2)} \subset U_\varepsilon^{(1)} \subset U_{\varepsilon/2}^{(2)}\]

This ensures open sets (unions of balls) are identical across metrics.
\end{proposition}

\begin{definition}[Equivalent Norms]
Two norms \(\|\cdot\|_a\) and \(\|\cdot\|_b\) on a vector space \(V\) are \textbf{equivalent} if \(\exists C_1, C_2 > 0\) such that
\[C_1 \|v\|_a \leq \|v\|_b \leq C_2 \|v\|_a \ \forall v \in V\]
\end{definition}

\begin{xca}[Exercises]
\leavevmode
\begin{enumerate}
    \item[(1)] This is an equiv. relation.
    \item[(2)] Equiv. norms define the same topology.
    \item[(3)] \(\|\cdot\|_1, \|\cdot\|_2, \|\cdot\|_\infty\) on \(\mathbb{R}^2\) equivalent.
    \item[(4*)] All norms on \(\mathbb{R}^n\) are equivalent.
\end{enumerate}
\end{xca}

\newpage

\section{Notes 6 - 10.31}

\subsection{Topological Spaces in General}

\begin{definition}[Topological Space]
A \textbf{topological space} is a pair \((X, \mathcal{U})\), where \(\mathcal{U} \subseteq 2^X\) (all subsets of \(X\)) and
\begin{enumerate}
    \item \(\varnothing, X \in \mathcal{U}\)
    \item If \(U_\alpha \in \mathcal{U}\), then \(\bigcup_\alpha U_\alpha \in \mathcal{U}\) (arbitrary union)
    \item If \(U_1, \cdots, U_k \in \mathcal{U}\), then \(\bigcap_{i=1}^k U_i \in \mathcal{U}\) (finite intersection)
\end{enumerate}
\end{definition}

\(\mathcal{U}\) is a \textbf{topology} on \(X\), and \(U \in \mathcal{U}\) are \textbf{open sets}.

\begin{example}[Examples of Topologies]
    \leavevmode
    \begin{enumerate}
        \item[\bfseries Ex 1.] \textbf{(Metric Spaces)}
        Any metric space \((X, d)\) is a topological space, where \(\mathcal{U} = \{ U \mid \forall x \in U, \exists \varepsilon > 0, U_\varepsilon(x) \subset U \}\).

        \item[\bfseries Ex 2.] \textbf{(Discrete Topology)}
        \(\mathcal{U} = 2^X\) (every subset of \(X\) is open).

        \item[\bfseries Ex 3.] \textbf{(Antidiscrete Topology)}
        \(\mathcal{U} = \{\varnothing, X\}\) (only \(\varnothing, X\) are open; not metric-induced).

        \item[\bfseries Ex 4.] \textbf{(Finite Complement Topology)}
        For infinite \(X\) (e.g., \(\mathbb{R}\)), \(\mathcal{U} = \{ U \mid X \setminus U \text{ is finite} \} \cup \{\varnothing\}\).
    \end{enumerate}
\end{example}

\subsection{Hausdorff Spaces}

\begin{proposition}[Metric Spaces are Hausdorff]
For distinct \(x,y \in X\) (metric space), \(\exists\) open \(U, V\) with \(x \in U\), \(y \in V\), \(U \cap V = \varnothing\).
\end{proposition}

\begin{remark}
Proof is commented out!
\end{remark}

% \begin{proof}
% Let \(\varepsilon = \frac{d(x,y)}{2}\), take \(U = U_\varepsilon(x)\), \(V = U_\varepsilon(y)\). If \(z \in U \cap V\), \(d(x,y) \leq d(x,z) + d(z,y) < \varepsilon + \varepsilon = d(x,y)\) (contradiction), then \(U \cap V = \varnothing\).
% \end{proof}

\begin{definition}[Hausdorff Space]
\((X, \mathcal{U})\) is \textbf{Hausdorff} if \(\forall x \neq y \in X\), \(\exists U, V \in \mathcal{U}\) with \(x \in U\), \(y \in V\), \(U \cap V = \varnothing\).
\end{definition}

\begin{remark}
Finite complement topology is \textit{not} Hausdorff: For \(U, V\) containing \(x \neq y\), \(X \setminus U, X \setminus V\) are finite, so \(U \cap V = X \setminus ((X \setminus U) \cup (X \setminus V))\) is non-empty.
\end{remark}

\subsection{Closed Sets and Zariski Topology}

\begin{definition}[Topological Space by Closed Sets]
A topology can be defined by satisfying closed sets: (\(Y \subset X\) is closed if \(X \setminus Y\) is open)
\begin{enumerate}
    \item \(\varnothing, X\) are closed;
    \item Arbitrary intersections: If \(\{V_\alpha\}\) are closed, \(\bigcap V_\alpha\) is closed;
    \item Finite unions: If \(V_1, \cdots, V_n\) are closed, \(\bigcup_{i=1}^n V_i\) is closed.
\end{enumerate}
\end{definition}

\begin{example}[Infinite Intersection of Open Sets]
Is \(\bigcap_{i=1}^\infty U_i\) open if each \(U_i\) is open?

Let \(U_n = \left(-\frac{1}{n}, \frac{1}{n}\right) \subset \mathbb{R}\). Then \(\bigcap_{n=1}^\infty U_n = \{0\}\), which is \textit{not open}.
\end{example}

\begin{definition}[Zariski Topology]
For \(X = \mathbb{C}^n\) (or \(\mathbb{R}^n\), \(K\) a field), \(Y \subset X\) is \textbf{closed} if \(Y\) is the solution set of polynomials:
\[Y = \left\{ (x_1, \cdots, x_n) \mid f_\alpha(x_1, \cdots, x_n) = 0 \ \forall \alpha \right\}\]
where \(f_\alpha\) are polynomials.
\end{definition}

\begin{example}[Zariski Closed Sets]
\begin{itemize}
    \item A single point \((a_1, \cdots, a_n) \in \mathbb{C}^n\) (closed: \(x_1=a_1, \cdots, x_n=a_n\)).
    \item \(\{ (x,y) \mid x^2 + y^2 = 3 \}\), \(\{ (x,y) \mid xy = \frac{1}{4} \}\) (closed: single polynomial equations).
\end{itemize}
\end{example}

\begin{remark}[Link to Finite Complement Topology]
For \(X = \mathbb{C}^1\) (or \(\mathbb{R}^1\)), the Zariski topology \textit{equals the finite complement topology}: closed sets are finite sets (solutions to non-constant polynomials) or \(X\). The Zariski topology is \textbf{not Hausdorff}.
\end{remark}

\subsection{Basis for a Topology}

\begin{definition}[Basis for a Topology]
Let \((X, \mathcal{U})\) be a topological space. A subset \(\mathcal{B} \subseteq \mathcal{U}\) is a basis of \(\mathcal{U}\) if
\begin{enumerate}
    \item \(\forall x \in X, \exists U \in \mathcal{B}\) with \(x \in U\);
    \item \(\forall x \in X, \forall U_1, U_2 \in \mathcal{B}\) (with \(x \in U_1, x \in U_2\)), \(\exists U_3 \in \mathcal{B}\) with \(x \in U_3 \subseteq U_1 \cap U_2\).
\end{enumerate}
\end{definition}

\begin{example}[Basis for Metric Spaces]
For a metric space (e.g., \(\mathbb{R}^n\)), the set of open balls \(\mathcal{B} = \{ U_\varepsilon(x) \mid x \in X, \varepsilon > 0 \}\) is a basis:
\begin{enumerate}
    \item Obvious (\(x \in U_\varepsilon(x)\) for any \(\varepsilon > 0\));
    \item \(\forall x \in U_{\varepsilon_1}(x_1) \cap U_{\varepsilon_2}(x_2)\), take \(\varepsilon = \min(\varepsilon_1, \varepsilon_2)\), \(U_\varepsilon(x) \subseteq U_{\varepsilon_1}(x_1) \cap U_{\varepsilon_2}(x_2)\) and \(U_\varepsilon(x) \in \mathcal{B}\).
\end{enumerate}
\end{example}

\begin{theorem}[Open Sets Are Unions of Basis Elements]
If \(\mathcal{B}\) is a basis of \(\mathcal{U}\), then any \(U \in \mathcal{U}\) can be written as \(U = \bigcup_{U_\alpha \in \mathcal{B}} U_\alpha\).
\end{theorem}

\begin{remark}
Proof is commented out!
\end{remark}

% \begin{proof}
% \(\forall x \in U, \exists \tilde{U}_x \in \mathcal{B}\) such that \(x \in \tilde{U}_x \subset U\). Take \(U_x \subset U \cap \tilde{U}_x\). Since \(\bigcup U_x \subset U\), then
% \[\bigcup U_x = U.\]
% \end{proof}

\begin{example}
\begin{enumerate}
    \item Balls in \(d_1, d_2, d_\infty\)-metric in \(\mathbb{R}^n\) form a basis of the standard topology.
    \item \(\{ U_\varepsilon(x) \mid \varepsilon \in \mathbb{Q}, \varepsilon > 0, x \in \mathbb{Q}^n \}\) is a countable set that is a basis of standard topology on \(\mathbb{R}^n\).
\end{enumerate}
\end{example}

\begin{xca}
Find a metric space not admitting a countable basis of topology.
\end{xca}

\newpage

% Week 4
\section{Notes 7 - 11.03}

\subsection{Basis of Topology}

\begin{definition}[Basis of Topology]
Let \(X\) be a set. A collection \(\mathcal{B} \subset 2^X\) is a basis for a topology on \(X\) if
\begin{enumerate}
\item \(\forall x \in X\), \(\exists B \in \mathcal{B}\) such that \(x \in B\).
\item If \(x \in B_1 \cap B_2\) with \(B_1, B_2 \in \mathcal{B}\), then \(\exists B_3 \in \mathcal{B}\) such that \(x \in B_3 \subset B_1 \cap B_2\).
\end{enumerate}

Then \(\mathcal{B}\) is a \textbf{basis of topology} on \(X\). The topology generated by \(\mathcal{B}\) is the collection \(\mathcal{U}\) of all unions of elements from \(\mathcal{B}\) (including \(\emptyset\)).
\end{definition}

\begin{proposition}
This collection \(\mathcal{U}\) is a topology on \(X\).
\end{proposition}

\begin{remark}
Proof is commented out!
\end{remark}

% \begin{proof}
% \begin{enumerate}
% \item \(\emptyset \in \mathcal{U}\); \(X \in \mathcal{U}\) (by property 1, \(X = \bigcup_{B \in \mathcal{B}} B\)).
% \item \textbf{Union}: Obvious (union of unions is a union).
% \item \textbf{Intersection}: Need to show \(U_1, U_2 \in \mathcal{U} \implies U_1 \cap U_2 \in \mathcal{U}\). Let \(x \in U_1 \cap U_2\).
% \begin{itemize}
% \item \(x \in U_1 \implies \exists B_1 \in \mathcal{B}\) such that \(x \in B_1 \subset U_1\).
% \item \(x \in U_2 \implies \exists B_2 \in \mathcal{B}\) such that \(x \in B_2 \subset U_2\).
% \end{itemize}

% By property (2), \(\exists B_3(x) \subset B_1 \cap B_2 \subset U_1 \cap U_2\) with \(x \in B_3(x)\).

% Thus \(U_1 \cap U_2 = \bigcup_{x \in U_1 \cap U_2} B_3(x)\), so it is in \(\mathcal{U}\).
% \end{enumerate}
% \end{proof}

\subsection{Examples and Lemma}

\begin{example}[Examples of Topologies]
\leavevmode
\begin{enumerate}
\item[\bfseries Ex 1.]
Any metric space \(X\) is a topological space, with \(\mathcal{U}\) defined as:
\(U \in \mathcal{U}\) if \(\forall x \in U, \exists \varepsilon > 0: U_\varepsilon(x) \subset U\).
(Open sets are defined ``as usual for metric spaces'').

\item[\bfseries Ex 2a.]
\(X\) arbitrary, \(\mathcal{U} = 2^X\) (discrete topology). Every subset is open.

\item[\bfseries Ex 2b.]
\(X\) arbitrary, \(\mathcal{U} = \{\varnothing, X\}\) (``antidiscrete topology'').

\item[\bfseries Ex 3.]
\(X\) infinite set (say, \(\mathbb{R}\)).
\(U \in \mathcal{U}\) if \(X \setminus U\) is finite, or \(U = \varnothing\).
Check axioms:
\begin{enumerate}
\item \(\varnothing\) is open (def). \(X \setminus X = \varnothing\) (finite) \(\implies X\) open.
\item Union: \(X \setminus (\bigcup U_\alpha) = \bigcap (X \setminus U_\alpha)\). Intersection of finite sets is finite.
\item Intersection: \(X \setminus (U_1 \cap U_2) = (X \setminus U_1) \cup (X \setminus U_2)\). Union of finite sets is finite.
\end{enumerate}
\end{enumerate}
\end{example}

\begin{remark}[Warning]
No ``minimality'' is required from \(\mathcal{B}\).
\end{remark}

\begin{example}[Standard Topology on \(\mathbb{R}^n\)]
Basis generating the standard topology on \(\mathbb{R}^n\):
\[\{ U_\varepsilon(x) \mid x \in \mathbb{Q}^n, \varepsilon \in \mathbb{Q}, \varepsilon > 0 \} = \mathcal{B}\]
This \(\mathcal{B}\) is countable \(\implies \mathbb{R}^n\) is second-countable.
\end{example}

\begin{lemma}
Let \(X\) be a topological space and \(\mathcal{C}\) be a collection of open sets such that \(\forall\) open \(U \subset X\), \(\forall x \in U\), we have \(x \in C \subset U\) for some \(C \in \mathcal{C}\).
Then \(\mathcal{C}\) is a basis of \(X\).
\end{lemma}

\begin{remark}
Proof is commented out!
\end{remark}

% \begin{proof}
% Need to check \(\mathcal{C}\) is a basis.
% \begin{enumerate}
% \item Take \(U=X\) (open). Lemma says \(\forall x \in X, \exists C \in \mathcal{C}, x \in C \subset X\).
% \item Take \(C_1, C_2 \in \mathcal{C}\) (open \(\implies C_1 \cap C_2\) open), then
% \[\forall x \in C_1 \cap C_2, \exists C_3 \in \mathcal{C} \text{ s.t. } x \in C_3 \subset C_1 \cap C_2.\]
% \end{enumerate}
% \end{proof}

\subsection{Comparing Topologies}

Consider topologies \(\mathcal{T}, \mathcal{U}\) on \(X\).

\begin{definition}[Comparing Topologies]
If \(\mathcal{T} \subset \mathcal{U}\), then
\begin{itemize}
\item \(\mathcal{U}\) is \textbf{finer} than \(\mathcal{T}\).
\item \(\mathcal{T}\) is \textbf{coarser} than \(\mathcal{U}\).
\end{itemize}
(Every set open in \(\mathcal{T}\) is open in \(\mathcal{U}\), but possibly not vice versa).
\end{definition}

\begin{note}[Extremes]
\begin{itemize}
\item Discrete topology \(\mathcal{U} = 2^X\) is the \textbf{finest} topology.
\item Anti-discrete topology \(\mathcal{U} = \{\emptyset, X\}\) is the \textbf{coarsest} one.
\end{itemize}
\end{note}

\begin{definition}[Topology Generated by \(\mathcal{C}\)]
Let \(\mathcal{T}\) be the topology generated by \(\mathcal{C}\) and define the initial topology on \(X\) by \(\mathcal{U}\), then
\[\mathcal{U} \subseteq \mathcal{T}.\]

Any element from \(\mathcal{U}\) can be obtained as a union of elements from \(\mathcal{C}\).
Indeed, for any \(U \in \mathcal{U}, x \in U\), we have \(x \in C_x \subset U\). So \(\bigcup C_x = U\).
We need the opposite: \(\mathcal{T} \subset \mathcal{U}\).
But \(\mathcal{C} \subset \mathcal{U}\), and \(\mathcal{U}\) is closed under unions (and intersections, by the Lemma/Basis property), so \(\mathcal{T} \subseteq \mathcal{U}\).
\end{definition}

\begin{lemma}
Let \(\mathcal{B}\) and \(\mathcal{C}\) be bases for \(\mathcal{T}\) and \(\mathcal{U}\). Then TFAE:
\begin{enumerate}
\item \(\mathcal{U}\) is finer than \(\mathcal{T}\).
\item \(\forall x \in X, B \in \mathcal{B}\), if \(x \in B\), then \(\exists C \in \mathcal{C}\) s.t. \(x \in C \subset B\).
\end{enumerate}
\end{lemma}

\begin{remark}
Proof is commented out!
\end{remark}

% \begin{proof}
% \begin{itemize}
% \item \((1) \implies (2)\)

% Let \(x \in X\) and \(B \in \mathcal{B}\) with \(x \in B\).
% Since \(\mathcal{B} \subset \mathcal{T}\) and \(\mathcal{T} \subset \mathcal{U}\), we have \(B \in \mathcal{U}\).
% Since \(\mathcal{C}\) is a basis for \(\mathcal{U}\), \(\exists C \in \mathcal{C}\) such that \(x \in C \subset B\).

% \item \((2) \implies (1)\)

% Let \(U \in \mathcal{T}\). Show that \(U \in \mathcal{U}\).
% \begin{itemize}
% \item \(\forall x \in U\), since \(\mathcal{B}\) is a basis for \(\mathcal{T}\), \(\exists B \in \mathcal{B}\) such that \(x \in B \subset U\).
% \item By condition (2), \(\exists C_x \in \mathcal{C}\) such that \(x \in C_x \subset B \subset U\).
% \end{itemize}

% Since \(\mathcal{C}\) is a basis for \(\mathcal{U}\), each \(C_x\) is open in \(\mathcal{U}\).
% Thus \(U = \bigcup_{x \in U} C_x\) is in \(\mathcal{U}\).
% \end{itemize}
% \end{proof}

\begin{example}[Lower Limit Topology in \(\mathbb{R}\)]
\begin{itemize}
\item Standard topology: Basis \(\{(a, b) \mid a < b\}\).
\item Lower limit topology (\(\mathbb{R}_\ell\)): Basis \(\{[a, b) \mid a < b\}\).
\end{itemize}
Then \(\mathbb{R}_\ell\) is finer than \(\mathbb{R}\). \([a, b)\) is open in \(\mathbb{R}_\ell\), but not open in \(\mathbb{R}\): take \(x=a\). There is no \((a', b') \subset [a, b)\), \(x \in (a', b') \implies \mathbb{R}_\ell \neq \mathbb{R} \implies \mathbb{R}_\ell\) strictly finer than \(\mathbb{R}\).

Ex. \(\mathbb{R}_\ell\) does not have a countable basis.
\end{example}

\begin{definition}[Order Topology]
\(X\) completely ordered set.
\[\mathcal{B} = \{ (a, b) \} \cup \{ [a_0, b) \} \cup \{ (a, b_0] \}\]
where \((a, b) = \{ x \in X \mid a < x < b \}\), \(a_0 = \min X\) (if exists), \(b_0 = \max X\) (if exists).
\end{definition}

\subsection{Product Topology}
\((X, \mathcal{U}), (Y, \mathcal{V})\) topological spaces.
\(X \times Y = \{(x, y) \mid x \in X, y \in Y\}\).

\begin{note}[Idea]
Take \(\mathcal{U} \times \mathcal{V}\) to be a \textbf{basis} of topology on \(X \times Y\). Closed under intersection:
\[(U_1 \times V_1) \cap (U_2 \times V_2) = (U_1 \cap U_2) \times (V_1 \cap V_2).\]
\end{note}

\begin{definition}[Product Topology]
The topology on \(X \times Y\) defined by basis \(\mathcal{U} \times \mathcal{V}\) is called the \textbf{product topology} on \(X \times Y\).
\end{definition}

\begin{theorem}
Let \(\mathcal{B} \subset \mathcal{U}\) be a basis of \(\mathcal{U}\), \(\mathcal{C} \subset \mathcal{V}\) be a basis of \(\mathcal{V}\).
\[\mathcal{D} = \mathcal{B} \times \mathcal{C} = \{ B \times C \subset X \times Y \mid B \in \mathcal{B}, C \in \mathcal{C} \}\]
is a basis of the product topology on \(X \times Y\).
\end{theorem}

\begin{remark}
Proof is commented out!
\end{remark}

% \begin{proof}
% Need to show: for every open \(W \subset X \times Y\), we have
% \[(x,y) \in B \times C \in \mathcal{D}.\]

% But \(\exists\) basis element \(U \times V \subset W\).
% Now
% \begin{align*}
% \exists B &: x \in B \subset U, B \in \mathcal{B}; \\
% \exists C &: y \in C \subset V, C \in \mathcal{C}.
% \end{align*}

% So \((x,y) \in B \times C \subset U \times V \subset W\), \(\mathcal{B} \times \mathcal{C} = \mathcal{D}\) is a basis.
% \end{proof}

\begin{example}
\(X \times Y\). Projections \(\pi_1, \pi_2\).

If \(U\) is open in \(X\), then \(\pi_1^{-1}(U) = \{ (x,y) \mid x \in U \} = U \times Y\) open in \(X \times Y\).

\(\pi_1\) is a continuous map! Same with \(\pi_2\).

\(U \times V = \pi_1^{-1}(U) \cap \pi_2^{-1}(V)\) open.

So \(S = \{ \pi_1^{-1}(U) \mid U \subset X \text{ open} \} \cup \{ \pi_2^{-1}(V) \mid V \subset Y \text{ open} \}\) is a basis (subbasis) for \(\mathcal{U} \times \mathcal{V}\).

``The product topology is the \textbf{coarsest} topology s.t. \(\pi_1, \pi_2\) are continuous.''
\end{example}

\newpage

\section{Notes 8 - 11.07}

\subsection{The Subspace Topology}

\begin{definition}[Subspace Topology]
Let \(X\) be a topological space with topology \(\mathcal{T}\). Let \(Y \subset X\) be an arbitrary subset.
Define the \textbf{subspace topology} \(\mathcal{T}_Y = \{ U \cap Y \mid U \in \mathcal{T} \}\).
\end{definition}

\begin{proposition}
\(\mathcal{T}_Y\) is a topology on \(Y\).
\end{proposition}

\begin{lemma}
If \(\mathcal{B}\) is a basis of \(\mathcal{T}\), then \(\mathcal{B}_Y = \{ B \cap Y \mid B \in \mathcal{B} \}\) is a basis of \(\mathcal{T}_Y\).
\end{lemma}

\begin{remark}
Proof is commented out!
\end{remark}

% \begin{proof}
% Let \(U \subset Y\) be open. Then \(U = U' \cap Y\) for some \(U'\) open in \(X\).

% Since \(\mathcal{B}\) is a basis of \(X\), \(U' = \bigcup B_\alpha\) with \(B_\alpha \in \mathcal{B}\).

% Then \(U = (\bigcup B_\alpha) \cap Y = \bigcup (B_\alpha \cap Y)\).

% These \(B_\alpha \cap Y\) are elements of \(\mathcal{B}_Y\). Thus \(\mathcal{B}_Y\) is a basis of \(Y\).
% \end{proof}

\begin{remark}[Warning]
Open in \(Y \nRightarrow\) open in \(X\).
\end{remark}

\begin{proposition}
If \(Y\) is \textbf{open} in \(X\), and \(U \subset Y\) is open in \(Y \implies U\) open in \(X\).
\end{proposition}

\begin{remark}
Proof is commented out!
\end{remark}

% \begin{proof}
% Since \(U \subset Y\) is open, we have \(U = Y \cap V\), where \(V\) is open in \(X\).

% \(Y\) is open in \(X\) \(\implies U\) is open in \(X\) (intersection of two open sets).
% \end{proof}

\begin{theorem}[Product of Subspaces]
Let \(A \subset X, B \subset Y\) be subsets. \(A \times B \subset X \times Y\). Then the product of subspace topologies on \(A\) and \(B\) is the subspace topology on \(A \times B \subset X \times Y\).
\end{theorem}

\begin{remark}
Proof is commented out!
\end{remark}

% \begin{proof}
% Let \(U \subset X, V \subset Y\) be open.

% The products \(U \times V\) form a basis of the product topology on \(X \times Y\).

% \((U \times V) \cap (A \times B)\) form a basis of subspace topology on \(A \times B\).

% But \((U \times V) \cap (A \times B) = (U \cap A) \times (V \cap B)\).

% So basis elements for both topologies are the same.
% \end{proof}

\subsection{Order Topology on Subsets}

\begin{example}
Let \(\mathbb{R} = X\) with standard order \(<\). Let \(Y = [0, 1) \cup \{2\}\).
Basis of topology on \(Y\) inherited from \(\mathbb{R}\) (subspace): \(\{(a, b) \cap Y\}\).
\begin{itemize}
\item \(\{2\}\) is open in subspace topology: \(\{2\} = Y \cap (1.9, 2.1)\).
\item \(\{2\}\) is \textbf{not} open in the order topology on \(Y\).
\end{itemize}

\textbf{Exercise}: Show that for the order topology, \([0, 1) \cup \{2\}\) is connected.
\end{example}

\begin{remark}[Warning]
Let \(Y \subset X\) subset; \(<\) same relation.
It may happen that the order topology on \((Y, <)\) is \textbf{different} from the subspace topology of the \((X, <)\) order topology!
\end{remark}

\begin{example}
\(X = Y = \mathbb{R}\), \(<\) order topology.
The product of order topology is the \textbf{standard topology} on \(\mathbb{R}^2\).
\(X \times Y = \{(x, y)\}\). Take the \textbf{lexicographic order}:
\((x, y) < (x', y')\) if
\begin{itemize}
\item either \(x < x'\);
\item or \(x = x', y < y'\).
\end{itemize}
Order topology on \((X \times Y, <)\)?

\(I \times I = [0, 1] \times [0, 1] \subset X \times Y\).
The order topology on \([0, 1] \times [0, 1]\) is \textbf{not} the subspace topology of the order topology on \(\mathbb{R} \times \mathbb{R}\)!
Example: \(\{\frac{1}{2}\} \times (\frac{1}{2}, 1]\) is \textbf{not} open in ord. top. on \(I \times I\), but is open in subspace top:
\(\{\frac{1}{2}\} \times (\frac{1}{2}, \frac{3}{2})\) is open in \(\mathbb{R} \times \mathbb{R}\).
\end{example}

\begin{definition}[Convex Set]
\(Y\) is \textbf{convex} if \(\forall a, b \in Y\), \((a, b) \subset Y\).
\end{definition}

\begin{theorem}
Let \(Y \subset X\) be a \textbf{convex} subset of \((X, <)\). Then the restriction of the order topology on \(X\) is the order topology on \(Y\).
\end{theorem}

\begin{remark}
Proof is commented out!
\end{remark}

% \begin{proof}
% Take \((a, +\infty), (-\infty, b) \subset X\). This is a basis of topology on \(X\).

% Take \(Y \subset X\).

% If \(a \in Y\), \(Y \cap (a, +\infty) = \{ y \in Y \mid a < y \}\).

% Otherwise \(a\) is either an upper bound or lower bound for \(Y\).

% \(Y \cap (a, +\infty)\) is either \(Y\) or \(\emptyset\) (depending on relation). Same for \((-\infty, b)\).
% \end{proof}

\subsection{Closed Sets and Closure}

\begin{definition}[Closed Set]
\(A \subset X\) is closed \(\iff X \setminus A\) is open.
\end{definition}

\begin{proposition}
\begin{enumerate}
\item \(\emptyset, X\) closed.
\item \(A_\alpha\) closed \(\implies \bigcap A_\alpha\) closed.
\item \(A_1, \cdots, A_n\) closed \(\implies \bigcup_{i=1}^n A_i\) closed.
\end{enumerate}
\end{proposition}

\begin{theorem}[Closed Sets in Subspace]
Let \(X\) be a topological space, \(Y \subset X\) have subspace topology. Then
\(A \subset Y\) closed in \(Y \iff A = Y \cap \tilde{A}\), where \(\tilde{A} \subset X\) is closed in \(X\).
\end{theorem}

\begin{remark}
Proof is commented out!
\end{remark}

% \begin{proof}
% \begin{itemize}
% \item[\(\Rightarrow\)]
% Let \(A \subset Y\) be closed in \(Y\), then \(Y \setminus A\) is open in \(Y\).

% So \(Y \setminus A = Y \cap U\), where \(U\) is open in \(X\).

% \(X \setminus U\) is closed in \(X\).

% \((X \setminus U) \cap Y = Y \setminus (U \cap Y) = Y \setminus (Y \setminus A) = A\). (Let \(\tilde{A} = X \setminus U\).)

% \item[\(\Leftarrow\)]
% Let \(A = Y \cap C\), where \(C\) is closed in \(X\).

% Then \(X \setminus C\) is open in \(X\), \((X \setminus C) \cap Y\) is open in \(Y\).

% But \((X \setminus C) \cap Y = (X \cap Y) \setminus (C \cap Y) = Y \setminus A\).

% So \(Y \setminus A\) is open in \(Y \implies A\) is closed in \(Y\).
% \end{itemize}
% \end{proof}

\begin{definition}[Neighborhood, Interior, Closure]
\begin{itemize}
\item If \(U \subset X\) open, \(x \in U\), then \(U\) is a \textbf{neighborhood} of \(x\).
\item \textbf{Interior} of \(A\): \(\text{Int } A = \bigcup \{ U \subset A, U \text{ open in } X \}\).
\item \textbf{Closure} of \(A\): \(\overline{A} = \bigcap \{ C \supset A, C \text{ closed in } X \}\).
\end{itemize}
Note: \(\text{Int } A \subset A \subset \overline{A}\).
\end{definition}

\begin{remark}[Warning]
If \(A \subset Y \subset X\), then the closure of \(A\) in \(Y\) and \(X\) can be different.

Example: \(A = (0, 1) \subset Y = [0, 1) \subset X = \mathbb{R}\).
\begin{itemize}
\item \(\overline{A}\) in \(\mathbb{R}\) is \([0, 1]\).
\item \(\overline{A}\) in \(Y\) is \([0, 1) = (\overline{A} \text{ in } \mathbb{R}) \cap Y\).
\end{itemize}
\end{remark}

\begin{theorem}
Let \(Y \subset X\) subspace, \(A \subset Y\) and \(\overline{A}\) be the closure of \(A\) in \(X\), then the closure of \(A\) in \(Y\) is
\[\overline{A} \cap Y.\]
\end{theorem}

\begin{remark}
Proof is commented out!
\end{remark}

% \begin{proof}
% Let \(B\) be the closure of \(A\) in \(Y\).

% \(\overline{A}\) closed in \(X\), so \(\overline{A} \cap Y\) closed in \(Y\).

% \(\overline{A} \cap Y \supset A \implies \overline{A} \cap Y \supset B\) (since \(B\) is smallest closed set containing \(A\)).

% Prove the opposite inclusion:

% \(B\) closed in \(Y \implies B = Y \cap C\), where \(C\) closed in \(X\).

% Hence \(C \supset A\) (closed sets containing \(A\)) \(\implies C \supset \overline{A}\).

% \(\overline{A} \cap Y \subset C \cap Y = B\).

% Therefore, \(B = \overline{A} \cap Y\).
% \end{proof}

\newpage

% Week 5
\section{Notes 9 - 11.10}

\subsection{Closure and Limit Points}

\begin{recap}[Closure]
Let \(A \subset X\). We defined \(\overline{A} = \bigcap C_\alpha\) where \(C_\alpha \supset A\) are closed sets.
\end{recap}

\begin{theorem}
\begin{enumerate}
\item[(a)] \(x \in \overline{A} \iff\) every open set \(U \subset X\) such that \(x \in U\) intersects \(A\) (that is, \(U \cap A \neq \emptyset\)).
\item[(b)] For a basis \(\mathcal{B}\) of \(\mathcal{T}\), we have \(x \in \overline{A} \iff\) for every \(B \in \mathcal{B}\) with \(x \in B\), we have
\[B \cap A \neq \emptyset.\]
\end{enumerate}
\end{theorem}

\begin{remark}
Proof is commented out!
\end{remark}

% \begin{proof}
% (a) is same as: \(x \notin \overline{A} \iff \exists U\) open, \(x \in U\) such that \(U \cap A = \emptyset\).
% \begin{itemize}
% \item[\(\Rightarrow\)]
% Take \(x \notin \overline{A}\), then \(U = X \setminus \overline{A}\) is open (since \(\overline{A}\) is closed) and contains \(x\).

% Since \(\overline{A} \supset A\), \(U \cap A = \emptyset\).

% \item[\(\Leftarrow\)]
% Take \(U \ni x\), \(U \cap A = \emptyset\), \(U\) open. Then \(X \setminus U\) is closed and \(X \setminus U \supset A\).

% Since \(\overline{A}\) is the intersection of all closed sets containing \(A\), \(X \setminus U \supset \overline{A}\) and \(x \notin \overline{A}\).
% \end{itemize}

% (b) Exercise.
% \end{proof}

\begin{definition}[Limit Points]
Another way to describe closures.
For \(A \subset X\), \(x \in X\) is a \textbf{limit point} of \(A\) if every open set \(U \ni x\) also contains some \(y \in A\) with \(y \neq x\).
Define the set of limit points by \(A'\).
\end{definition}

\begin{theorem}
\(\overline{A} = A \cup A'\).
\end{theorem}

\begin{remark}
Proof is commented out!
\end{remark}

% \begin{proof}
% \begin{itemize}
% \item If \(x \in A'\), every neighborhood of \(x\) intersects \(A\). Then \(x \in \overline{A}\) by the previous theorem. Thus, \(A' \subset \overline{A}\), \(A \subset \overline{A} \implies A \cup A' \subset \overline{A}\).
% \item Reverse: take \(x \in \overline{A}\). Then \(x \in A \subset A \cup A'\). Otherwise, if \(x \notin A\), every open neighborhood of \(x\) intersects \(A\) (at some \(y\)). Since \(x \notin A\), \(y \neq x\), then \(x \in A'\).
% \end{itemize}
% \end{proof}

\begin{corollary}
\(A\) is closed \(\iff A\) contains all its limit points (\(A' \subset A\)).
\end{corollary}

\subsection{Convergence of Sequences}

\begin{definition}[Convergence of Sequences]
For a sequence \(x_n = (x_1, x_2, \cdots)\), \(x_i \in X\), \(a \in X\) is a \textbf{limit} of \(x_n\) (\(x_n \stackrel{n \to \infty}{\longrightarrow} a\) or \(a = \lim_{n \to \infty} x_n\)) if for every open \(U \ni a\), there exists \(N \in \mathbb{N}\) such that \(x_n \in U\) for all \(n > N\).
\end{definition}

\begin{remark}
This is equivalent to: every neighborhood of \(a\) contains \textbf{ALMOST} all elements of \(x_n\) (all except finitely many).
\end{remark}

\begin{example}[Examples of Convergence]
\leavevmode
\begin{enumerate}
\item[\bfseries Ex 1.] \textbf{Discrete Topology}
\(X\) arbitrary with discrete topology.
\(x_n \to a \iff x_n = a\) for all \(n > N\) (\(n \gg 0\), sufficiently large).

\item[\bfseries Ex 2.] \textbf{Rationals}
\(X = \mathbb{Q}\), topology defined by usual metric.
Sequence \(3, 3.1, 3.14, 3.141, \cdots\) (approximating \(\pi\)). \textbf{DOES NOT converge!} (since \(\pi \notin \mathbb{Q}\)).

\item[\bfseries Ex 3.] \textbf{Lower Limit Topology (\(\mathbb{R}_\ell\))}
\(\mathbb{R}_\ell\) is \(\mathbb{R}\) with lower limit topology: \([a, b)\) are basis open sets. What sequences \(x_n \in \mathbb{R}\) are convergent?

A sequence converges in \(\mathbb{R}_\ell \iff\) it converges in \(\mathbb{R}\) \textbf{AND} satisfies an extra condition:
\[x_n \ge a \ \text{for } n \gg 0\]
Example: \(\frac{(-1)^n}{n}\) is not convergent to 0 in \(\mathbb{R}_\ell\).
For every \(\varepsilon\), \([0, \varepsilon)\) does not satisfy the condition for limits (negative terms are outside).
\end{enumerate}
\end{example}

\subsection{Comparing Topologies and Convergence}

Let \(\mathcal{T}, \mathcal{T}'\) be topologies on \(X\).
\(\mathcal{T}'\) is \textbf{finer} than \(\mathcal{T}\) if \(\mathcal{T}' \supset \mathcal{T}\) (\(\mathcal{T}\) is coarser than \(\mathcal{T}'\)).

\begin{proposition}[Convergence in Finer Topology]
\(x_n\) converges in \(\mathcal{T}'\) (finer) \(\implies x_n\) converges in \(\mathcal{T}\) (coarser).
\end{proposition}

(Since open sets in \(\mathcal{T}\) are also open in \(\mathcal{T}'\), the condition is easier to satisfy in \(\mathcal{T}\)).

\begin{example}[\(\mathbb{R}_\ell\) vs \(\mathbb{R}\)]
We know \(\mathbb{R}_\ell\) is finer than \(\mathbb{R}\).
\begin{itemize}
\item \(\mathbb{R}\) is \textbf{NOT} finer than \(\mathbb{R}_\ell\):
    
For \([a, b) \ni x\) (basis in \(\mathbb{R}_\ell\)), does there exist \((a', b') \ni x\) such that \((a', b') \subset [a, b)\)?

If \(x=a\), then \((a', b')\) must contain points less than \(a\), which are not in \([a, b)\). Impossible.
    
\item \(\mathbb{R}_\ell\) \textbf{IS} finer than \(\mathbb{R}\): For any \((a, b) \ni x\), \([x, b)\) is a basis element in \(\mathbb{R}_\ell\) and \([x, b) \subset (a, b)\).
\end{itemize}

Implication: Convergence in \(\mathbb{R}_\ell \implies\) Convergence in \(\mathbb{R}\).

(As seen in Ex 3, convergence in \(\mathbb{R}_\ell\) is strictly stronger).
\end{example}

\subsection{More Limit Examples}

\begin{example}[More Limit Examples]
\leavevmode
\begin{enumerate}
\item[\bfseries Ex 4.] \textbf{Non-Hausdorff Space (\(X = \mathbb{N}\))}
Let \(U_m = \{0, 1, 2, \cdots, m\}\).
Topology \(\mathcal{T} = \{ U_m \mid m \in \mathbb{N} \} \cup \{ \mathbb{N}, \emptyset \}\).
Convergence \(x_n \to a\):
\begin{align*}
&\iff \forall U_m, m \ge a, x_n \in U_m \text{ for } n \gg 0 ;\\
&\iff \forall m \ge a: x_n \le m \text{ for } n \gg 0.
\end{align*}

Equivalently: \(x_n \le a\) for almost all \(n\). Thus, \(a\) is an \textbf{essential upper bound} of \(x_n\).
\begin{itemize}
\item Convergent \(\iff\) Bounded.
\item \textbf{Limit is not unique!} If \(x_n \to a\), then \(x_n \to b\) for any \(b \ge a\).
\end{itemize}

\item[\bfseries Ex 5.] \textbf{Finite Complement Topology on \(\mathbb{R}\)}
\(\mathcal{T} = \{ \emptyset, \mathbb{R} \} \cup \{ \mathbb{R} \setminus \{p_1, \cdots, p_k\} \}\).
\begin{enumerate}
\item
If \(x_n = a\) for \(n \gg 0\): \(x_n \to a\).

\item
If \(x_n\) assumes two values infinitely many times (e.g., \(x_n = (-1)^n\)): \textbf{NO LIMIT}.

(Any open set excluding one value will fail to contain the tail).
            
\item
If \(x_n\) assumes every value finitely many times (all terms distinct): \(x_n \to a, \forall a \in \mathbb{R}\).

(Proof: Take any \(a\). Any open \(U \ni a\) is \(\mathbb{R} \setminus F\) where \(F\) is finite. Since \(x_n\) takes values in \(F\) only finitely many times, eventually \(x_n \notin F\), so \(x_n \in U\).)
\end{enumerate}
\end{enumerate}
\end{example}

\newpage

\section{Notes 10 - 11.14}

\subsection{Hausdorff Spaces (\(T_2\))}

\begin{definition}[Hausdorff Space]
A topological space \(X\) is \textbf{Hausdorff}, or (\(T_2\)) if for any distinct points \(x \neq y\) in \(X\), there exist open sets \(U \ni x\) and \(V \ni y\) such that \(U \cap V = \varnothing\).
\end{definition}

\begin{proposition}
If \(X\) is a metric space, then \(X\) is Hausdorff.
\end{proposition}

\begin{proof}
Let \(x \neq y\) and \(d = d(x, y) > 0\), then \(U_{d/2}(x) \cap U_{d/2}(y) = \varnothing\).
\end{proof}

\begin{proposition}
If \(X\) is Hausdorff, then \(\{x_0\} \subset X\) is closed for every \(x_0 \in X\).
\end{proposition}

\begin{remark}
Proof is commented out!
\end{remark}

% \begin{proof}
% We show that \(X \setminus \{x_0\}\) is open.
% For any \(x \in X \setminus \{x_0\}\) (\(x \neq x_0\)), since \(X\) is Hausdorff, there exist disjoint open sets \(U_x \ni x_0\) and \(U \ni x\).

% Since \(x_0 \in U_x\) and \(U \cap U_x = \varnothing\), it follows that \(x_0 \notin U\). Thus, \(U \subset X \setminus \{x_0\}\).

% Since this holds for any \(x\), \(X \setminus \{x_0\}\) is open \(\implies \{x_0\}\) is closed.
% \end{proof}

\begin{xca}
If for all \(x \in X\), \(\{x\}\) is closed, is it true that \(X\) is Hausdorff?
\end{xca}
\begin{proof}[Solution]
No. Consider the \textbf{Finite Complement Topology}. \(\{x\}\) are closed (since their complement is open), but the space is not Hausdorff (any two non-empty open sets intersect).
\end{proof}

\subsection{Fréchet Spaces (\(T_1\))}

\begin{definition}[Fréchet Space]
If every point \(x \in X\) is closed (\(\{x\}\) is a closed set), \(X\) is a \textbf{Fréchet space}, or (\(T_1\)) space.
\end{definition}

\begin{remark}[Relationship between \(T_1\) and \(T_2\)]
\[(T_2) \implies (T_1)\]
(Hausdorff spaces are \(T_1\), but the converse is not true).
\end{remark}

\begin{proposition}
\(X\) is (\(T_1\)) iff \(\forall x \neq y \in X, \exists \text{ open } U \ni x \text{ such that } y \notin U\).
\end{proposition}

Ex. Prove this.

\begin{xca}
Let \(X\) satisfy the following: for every \(x, y \in X, x \neq y\), either \(\exists U \ni x, y \notin U\) or \(\exists U \ni y, x \notin U\). Is it true that \(X\) is (\(T_1\))?
\end{xca}

\begin{theorem}
Let \(X\) be a (\(T_1\)) space and \(A \subset X\). Then \(x \in A'\) (limit point) if and only if every open neighborhood \(U \ni x\) contains infinitely many points from \(A\).
\end{theorem}

\begin{remark}
Proof is commented out!
\end{remark}

% \begin{proof}
% \begin{itemize}
%     \item[\(\Leftarrow\)] Obvious.
%     \item[\(\Rightarrow\)]
%     Suppose \(x \in A'\). Then for any open \(U \ni x\), we have \(U \cap (A \setminus \{x\}) \neq \varnothing\).

%     Suppose for contradiction that \(U \cap A\) is finite, say \(U \cap A = \{y_1, \cdots, y_n\}\).

%     Then \(V = U \setminus \{y_1, \cdots, y_n\}\) is open (finite sets are closed in \(T_1\)).

%     Also \(x \in V\) (assuming \(x\) was not one of \(y_i\), or taking \(V = U \setminus (\{y_1, \cdots, y_n\} \setminus \{x\})\)).

%     Then \((V \setminus \{x\}) \cap A = \emptyset\), which implies \(x\) is not a limit point. Contradiction!
% \end{itemize}
% \end{proof}

\begin{theorem}[Uniqueness of Limits]
A sequence \((x_1, x_2, \cdots)\) in a Hausdorff space \(X\) has at most one limit.
\end{theorem}

\begin{remark}
Proof is commented out!
\end{remark}

% \begin{proof}
% Suppose \(a_1, a_2\) are distinct limits. Since \(X\) is Hausdorff, \(\exists U_1 \ni a_1, U_2 \ni a_2\) with \(U_1 \cap U_2 = \varnothing\).

% Then \(U_1\) contains all \(x_n\) starting from \(N_1\).

% \(x_n \in U_2\) if \(n > N_2\). So for \(n > \max(N_1, N_2)\), \(x_n \in U_1 \cap U_2 = \varnothing\). Contradiction!
% \end{proof}

\begin{theorem}[Properties of Hausdorff Spaces]
\begin{enumerate}
    \item[(i)] \((X, <)\) with the order topology is Hausdorff.
    \item[(ii)] If \(X\) is Hausdorff, \(Y \subset X\) with subspace topology \(\implies Y\) Hausdorff.
    \item[(iii)] If \(X_1, X_2\) Hausdorff, then \(X_1 \times X_2\) with product topology is Hausdorff.
\end{enumerate}
\end{theorem}

\subsection{Continuity}

\begin{definition}[Continuity]
A function \(f: X \to Y\) is \textbf{continuous} if \(\forall V \subset Y\) open, \(f^{-1}(V) \subset X\) is open.
\end{definition}

\begin{remark}
If topology on \(Y\) is given by basis \(\mathcal{B}\), \(f\) continuous \(\iff f^{-1}(B)\) is open \(\forall B \in \mathcal{B}\).
\end{remark}

\begin{example}
\(\mathbb{R}_{st} \leftrightarrow \mathbb{R}_{\ell}\).
\(f(x)=x\) (Id as maps of sets).
\begin{itemize}
    \item \(\mathbb{R}_{st} \xrightarrow{f} \mathbb{R}_{\ell}\): not continuous. \(f^{-1}([0, 1)) = [0, 1)\) not open in \(\mathbb{R}_{st}\).
    \item \(\mathbb{R}_{\ell} \xrightarrow{g} \mathbb{R}_{st}\): continuous. For \(\mathbb{R}_{st}\), \((a, b)\) is open. \(g^{-1}((a, b)) = (a, b)\) is open in \(\mathbb{R}_{\ell}\) (since \([x, b) \subset (a, b)\) type sets form basis, or simply \(\mathbb{R}_{\ell}\) finer).
\end{itemize}
\end{example}

\begin{remark}[Observation]
    If \(\mathcal{T}, \mathcal{T}'\) topologies on \(X\) and \(\mathcal{T} \subsetneq \mathcal{T}'\), then
    \begin{itemize}
        \item \(f = Id: (X, \mathcal{T}) \to (X, \mathcal{T}')\) is \textbf{not} continuous.
        \item \(g = Id: (X, \mathcal{T}') \to (X, \mathcal{T})\) is continuous.
    \end{itemize}
\end{remark}

\begin{theorem}[TFAE]
\begin{enumerate}
    \item \(f: X \to Y\) continuous.
    \item \(\forall A \subset X\), \(f(\overline{A}) \subset \overline{f(A)}\).
    \item For every closed \(C \subset Y\), \(f^{-1}(C)\) closed in \(X\).
    \item \(\forall x \in X\), \(\forall V \ni f(x), \exists U \ni x: f(U) \subset V\).
\end{enumerate}
\[(1) \implies (2) \implies (3) \implies (1), \ (1) \iff (4)\]
\end{theorem}

\begin{remark}
Proof is commented out!
\end{remark}

% \begin{proof}
% \begin{itemize}
%     \item \((1) \implies (2)\): \(f: X \to Y, A \subset X\). Let \(x \in \overline{A}\); need to show \(f(x) \in \overline{f(A)}\). Take \(V \ni f(x)\) open. \(x \in f^{-1}(V)\) open in \(X\). So \(\exists y \in A \cap f^{-1}(V)\). Then \(f(y) \in f(A) \cap V\). So \(V \cap f(A) \neq \varnothing\).
%     \item \((2) \implies (3)\): Let \(C \subset Y\) closed, \(A = f^{-1}(C)\). Need to show \(A = \overline{A}\).
%     \(f(A) = f(f^{-1}(C)) \subset C\). So if \(x \in \overline{A}\), \(f(x) \in f(\overline{A}) \subset \overline{f(A)} \subset \overline{C} = C\). So \(x \in f^{-1}(C)\).
%     \item \((3) \implies (1)\): Obvious (take complement).
%     \item \((1) \implies (4)\): Take \(f^{-1}(V) =: U\) open.
%     \item \((4) \implies (1)\): Let \(V \subset Y\) open. Need: \(f^{-1}(V)\) open. \(x \in f^{-1}(V) \implies f(x) \in V\). So \(\exists U_x \subset X\) s.t. \(f(U_x) \subset V\). Take \(U = \bigcup U_x\) open. \(f^{-1}(V) \subset U \implies f^{-1}(V) = U\).
% \end{itemize}
% \end{proof}

\subsection{Constructing Continuous Functions}

\begin{theorem}[Constructing Continuous Functions]
\(X, Y, Z\) topological spaces.
\begin{enumerate}
    \item[(a)] \(f: X \to Y, f(x) = y \in Y\) is continuous (constant).
    \item[(b)] \(A \subset X, j: A \hookrightarrow X\) continuous (subspace embedding).
    \item[(c)] \(f: X \to Y, g: Y \to Z\) cont \(\implies gf: X \to Z\) cont.
    \item[(d)] Restriction of domain: \(f: X \to Y\) cont, \(A \subset X\), then \(f|_A: A \to Y\) cont.
    \item[(e)] \(f: X \to Y, Z \subset Y, f(X) \subset Z\). Then \(f: X \to Z\) continuous (restriction of range).
    \item[(f)] \(f: X \to Y, Y \subset Z\), then \(f: X \to Z\) continuous.
    \item[(g)] \(f: X \to Y\) is cont if \(X = \bigcup U_\alpha\), \(f|_{U_\alpha} \to Y\) cont. (Locality).
\end{enumerate}
\end{theorem}

\begin{proof}[Proof of (g)]
\(V \subset Y \implies f^{-1}(V) = \bigcup_\alpha (f^{-1}(V) \cap U_\alpha)\).
\end{proof}

\begin{lemma}[Pasting Lemma]
Let \(X = A \cup B\).
\(f: A \to Y\) cont, \(g: B \to Y\) cont.
\(f(x) = g(x) \forall x \in A \cap B\).
Then \(f\) and \(g\) are ``glued together'': \(\exists h: X \to Y\) cont, \(h|_A = f, h|_B = g\). (\(A, B\) both closed or open.)
\end{lemma}

\newpage

% Week 6
\section{Notes 11 - 11.17}

\subsection{Product Topology on \(X \times Y\)}

\begin{theorem}
Let \(f: A \to X \times Y\), \(f(a) = (f_1(a), f_2(a))\). Then \(f\) is continuous \(\iff\) both \(f_1, f_2\) are continuous.
\([f_1: A \to X, f_2: A \to Y]\).
\end{theorem}

\begin{remark}
Proof is commented out!
\end{remark}

% \begin{proof}
% \begin{itemize}
%     \item[\(\Rightarrow\)]
%     The projection maps \(\pi_1, \pi_2\) are continuous:

%     Let \(U \subset X\) be open: \(\pi_1^{-1}(U) = U \times Y\); same about \(\pi_2\).

%     Then \(f_1 = \pi_1 \circ f\), \(f_2 = \pi_2 \circ f\) are continuous as compositions.

%     \item[\(\Leftarrow\)]
%     Take \(U \times V \subset X \times Y\) open. \(f^{-1}(U \times V) = f_1^{-1}(U) \cap f_2^{-1}(V)\) is open.
% \end{itemize}
% \end{proof}

\subsection{Infinite Products}

Take (countably) infinitely many spaces \(X_1, X_2, \cdots\).

Maps \([0, 1] \to \mathbb{R}\): how to turn this into a topological space? \(\mathbb{R}^{[0,1]}\).

\subsubsection{Tuples}
\begin{definition}[Tuples]
Let \(J\) be an arbitrary set. A \(J\)-tuple of elements from \(X\) is a function \(x: J \to X\), \(\alpha \in J \leadsto x(\alpha) = x_\alpha \in X\).
Usually denoted \((x_\alpha)_{\alpha \in J}\).
\end{definition}

\subsubsection{The Cartesian Product}
\begin{definition}[Cartesian Product]
Let \((A_\alpha)_{\alpha \in J}\) be an indexed family of sets.
The \textbf{Cartesian product} of this family \((A_\alpha)\) is denoted by \(\prod_{\alpha \in J} A_\alpha\) and defined as the set of all \(J\)-tuples of elements in \(X\) (where \(X = \bigcup_{\alpha \in J} A_\alpha\)) such that \(x_\alpha \in A_\alpha\) \(\forall \alpha \in J\).
That is, the set of all functions \(x: J \to \bigcup A_\alpha\) s.t. \(x_\alpha \in A_\alpha, \forall \alpha \in J\).

When \(A_\alpha = X\), \(\prod_{\alpha \in J} A_\alpha = X^J\).
\end{definition}

\begin{example}
\(X = [0, 2] \times [1, 5] = \{ (x, y) \mid 0 \le x \le 2, 1 \le y \le 5 \}\).
\end{example}

\subsection{Topologies on Infinite Products}

Two ways to introduce topology on \(\prod_{\alpha \in J} X_\alpha\).

\subsubsection{1. Box Topology}
Take \(U_1 \subset X_1, U_2 \subset X_2, \cdots\) (open sets).
The topology is given by the basis \(\prod_{\alpha \in J} U_\alpha\).
This defines the \textbf{box topology}.

\begin{definition}[Box Topology]
Let \((X_\alpha)_{\alpha \in J}\) be an indexed family of topological spaces.
The \textbf{box topology} on \(\prod_{\alpha \in J} X_\alpha\) is given by the basis \(\prod_{\alpha \in J} U_\alpha\), where \(U_\alpha \subset X_\alpha\) is open. Basis element is
\[(\prod U_\alpha) \cap (\prod V_\alpha) = \prod (U_\alpha \cap V_\alpha).\]
\end{definition}

\subsubsection{2. Product Topology}
Take as a basis products \(U_1 \times U_2 \times \cdots\) where \textbf{only finitely many} of \(U_i \neq X_i\).
This defines the \textbf{Product topology}.

\subsubsection{Subbasis}
\begin{definition}[Subbasis]
A collection \(\mathcal{S}\) of subsets of space \(X\) is a \textbf{subbasis} if \(\bigcup_{S \in \mathcal{S}} S = X\).
\end{definition}

From Subbasis to Basis: take all finite intersections of elements from \(\mathcal{S}\).

Topology: all unions of these finite intersections.

\begin{note}[Diagram]
\(\mathcal{S} \xrightarrow{\text{finite } \cap} \mathcal{B} \xrightarrow{\text{arbitrary } \cup} \mathcal{T}\).
\end{note}

\begin{example}
    \leavevmode
    \begin{itemize}
        \item \(\mathcal{S} = \{X\}\) (trivial).
        \item \(\mathcal{S} = \{ \text{horizontal strips}, \text{vertical strips} \}\). Basis \(\mathcal{B} = \{ \text{rectangles} \}\).
    \end{itemize}
\end{example}

\begin{definition}[Product Topology]
Let \(\pi_\beta: \prod_{\alpha \in J} X_\alpha \to X_\beta\) be the projection map, \(x \mapsto x_\beta\).

Let \(\mathcal{S}_\beta = \{ \pi_\beta^{-1}(U_\beta) \mid U_\beta \subset X_\beta \text{ open} \}\) (``cylinders''), then \(\mathcal{S} = \bigcup_{\beta \in J} \mathcal{S}_\beta\) is a subbasis.

The topology defined by it is the \textbf{product topology}.

Basis \(\mathcal{B}\) given by finite intersections of elements from \(\mathcal{S}\).

\(\mathcal{B} \ni B = \prod U_\alpha\), where \(U_\alpha \subset X_\alpha\) open, and \(U_\alpha \neq X_\alpha\) for \(\alpha \in \{\beta_1, \cdots, \beta_n\}\) (finite set of indices).
That is, \(U_\alpha = X_\alpha\) for \textbf{almost all} elements (all except a finite set).
\end{definition}

\begin{remark}[Comparison]
If \(|J| < \infty\), the box and product topologies are the same.
If \(J\) is infinite, \textbf{Box topology} is \textbf{finer} than \textbf{Product topology}.
\end{remark}

\subsection{Properties of Product Spaces}

\begin{theorem}[Subspace Topology]
Let \(A_\alpha \subset X_\alpha\) be subspaces.
Then \(\prod A_\alpha \subset \prod X_\alpha\) is a subspace:
\begin{itemize}
    \item in box topology.
    \item in product topology.
\end{itemize}
\end{theorem}

\begin{theorem}[Hausdorff]
If \(X_\alpha\) are Hausdorff, then \(\prod X_\alpha\) is Hausdorff in both topologies.
\end{theorem}

\begin{theorem}[Closure]
Let \(A_\alpha \subset X_\alpha\) be subsets. In each topology (box or product),
\[\overline{\prod A_\alpha} = \prod \overline{A_\alpha}\]
\end{theorem}

\begin{remark}
Proof is commented out!
\end{remark}

% \begin{proof}
% \begin{itemize}
%     \item[\(\Leftarrow\)]
%     Take \((x_\alpha) = x \in \prod \overline{A_\alpha}\); show that \(x \in \overline{\prod A_\alpha}\).

%     Let \(\prod U_\alpha \ni x\) (basis element), then
%     \[x_\alpha \in \overline{A_\alpha} \implies \exists y_\alpha \in A_\alpha \cap U_\alpha.\]

%     So \(y = (y_\alpha) \in \prod A_\alpha\) and \(y \in U = \prod U_\alpha\).

%     So \(x \in \overline{\prod A_\alpha}\).

%     \item[\(\Rightarrow\)]
%     Take \(x = (x_\alpha) \in \overline{\prod A_\alpha}\); show that \(\forall \beta, x_\beta \in \overline{A_\beta}\).

%     Take \(V_\beta \ni x_\beta\) open in \(X_\beta\); its preimage \(\pi_\beta^{-1}(V_\beta)\) is open in \(\prod X_\alpha\).

%     So \(\exists y = (y_\alpha) \in \prod A_\alpha \cap \pi_\beta^{-1}(V_\beta)\), \(y_\beta \in V_\beta \cap A_\beta \implies x_\beta \in \overline{A_\beta}\).
% \end{itemize}
% \end{proof}

\subsection{Continuity into Product Spaces}

\begin{theorem}
Let \(f: A \to \prod X_\alpha\) be given by \(f(a) = (f_\alpha(a))_{\alpha \in J}\).
Then \(f\) is continuous \textbf{in the product topology} iff all \(f_\alpha\) are continuous.
\end{theorem}

\begin{remark}
Proof is commented out!
\end{remark}

% \begin{proof}
% \begin{itemize}
%     \item[\(\Rightarrow\)]
%     \(f\) continuous. \(\pi_\alpha\) is continuous \(\implies f_\alpha: A \to X_\alpha\), \(f_\alpha = \pi_\alpha \circ f\) is continuous.

%     \item[\(\Leftarrow\)]
%     Let all \(f_\alpha\) be continuous.

%     Enough to prove: \(f^{-1}(B)\) is open for each element \(B\) of \textbf{subbasis}.

%     But if \(B = \pi_\beta^{-1}(U_\beta)\), \(U_\beta \subset X_\beta\) open.
%     \(f^{-1}(B) = f^{-1}(\pi_\beta^{-1}(U_\beta)) = f_\beta^{-1}(U_\beta)\) is open in \(A\) (since \(f_\beta\) is continuous).
% \end{itemize}
% \end{proof}

\begin{example}[Counterexample for Box Topology]
\(\mathbb{R}^\omega = \prod_{n \in \mathbb{N}} X_n\), \(X_n = \mathbb{R}\).
\(f: \mathbb{R} \to \mathbb{R}^\omega\), \(t \mapsto (t, t, t, \cdots)\).

This map is continuous in the product topology (since each component \(f_n(t) = t\) is continuous) and \textbf{not continuous} in the box topology.

Take \(B = (-1, 1) \times (-\frac{1}{2}, \frac{1}{2}) \times (-\frac{1}{3}, \frac{1}{3}) \times \cdots\). \(B\) is open in box topology.

\(f^{-1}(B) = \{ t \mid t \in (-\frac{1}{n}, \frac{1}{n}) \forall n \} = \{0\}\). \(\{0\}\) is \textbf{not open} in \(\mathbb{R}\).

Thus \(f^{-1}(B)\) is not open, so \(f\) is not continuous. (\(f_n(-\delta, \delta) \not\subset (-\frac{1}{n}, \frac{1}{n})\) for large \(n\)).
\end{example}

\newpage

\section{Notes 12 - 11.21}

\subsection{Metrizable Spaces}

\begin{definition}[Metrizable Space]
A topological space \((X, \mathcal{T})\) is \textbf{metrizable} if there is a metric \(d\) on \(X\) inducing the topology \(\mathcal{T}\).
(that is, \(U \in \mathcal{T} \iff \forall x \in U, \exists \varepsilon > 0\) s.t. \(U_\varepsilon(x) \subset U\)).
Here \(U_\varepsilon(x) = \{ y \in X \mid d(x, y) < \varepsilon \}\).
\end{definition}

\begin{remark}[Basis]
Take \(U_\varepsilon(x)\) for all \(\varepsilon > 0, x \in X\) as a basis of topology on \(X\).
\end{remark}

\begin{xca}
Why is this a basis?
\end{xca}
\begin{proof}[Solution]
\(U_{\varepsilon_1}(x) \cap U_{\varepsilon_2}(x) = U_{\min(\varepsilon_1, \varepsilon_2)}(x)\). (Intersection condition holds).
\end{proof}

\subsection{Metrics on \(\mathbb{R}^n\)}

\begin{definition}[Standard Metrics on \(\mathbb{R}^n\)]
For \(x = (x_1, \cdots, x_n), y = (y_1, \cdots, y_n) \in \mathbb{R}^n\):
\begin{enumerate}
    \item \(d_1(x, y) = \sum_{i=1}^n |x_i - y_i|\)
    \item \(d_2(x, y) = \left( \sum_{i=1}^n (x_i - y_i)^2 \right)^{1/2}\)
    \item \(d_\infty(x, y) = \max_{i=1, \cdots, n} |x_i - y_i|\)
\end{enumerate}
\end{definition}

\begin{proposition}
The metrics \(d_1, d_2, d_\infty\) define the same topology on \(\mathbb{R}^n\).
This is the product (= box) topology on \(\mathbb{R}^n = \mathbb{R} \times \cdots \times \mathbb{R}\) (\(n\) times).
\end{proposition}

\begin{lemma}[Comparing Metrics]
Let \(d, d'\) be two metrics on \(X\) defining topologies \(\mathcal{T}, \mathcal{T}'\).
\(\mathcal{T}' \supset \mathcal{T}\) (\(\mathcal{T}'\) is finer) iff
\(\forall x \in X, \varepsilon > 0, \exists \delta > 0\) such that \(U_\delta^{(d')}(x) \subset U_\varepsilon^{(d)}(x)\).
\end{lemma}

\begin{remark}
Proof is commented out!
\end{remark}

% \begin{proof}
% Look at the shapes of balls:
% \begin{itemize}
%     \item \(d_\infty\) balls are squares.
%     \item \(d_2\) balls are circles.
%     \item \(d_1\) balls are diamonds.
% \end{itemize}

% We can fit a circle inside a square, and a square inside a circle (scaled).
% \(d_\infty \sim d_2\), \(d_1 \sim d_\infty\), \(d_1 \sim d_2\).
% Thus they induce the same topology.
% \end{proof}

\subsection{Bounding a Metric}

\begin{definition}[Bounded Metric]
Let \(d\) be a metric on \(X\).
Let \(\overline{d}: X \times X \to \mathbb{R}\),
\[\overline{d}(x, y) = \min(d(x, y), 1) = \begin{cases} d(x, y) & \text{if } d(x, y) \le 1 \\ 1 & \text{otherwise} \end{cases}\]
\end{definition}

\begin{theorem}
\(\overline{d}\) is a metric on \(X\) defining the \textbf{same topology} as \(d\).
\end{theorem}

\begin{remark}
Proof is commented out!
\end{remark}

% \begin{proof}
% Need to show: \(\overline{d}(x, z) \le \overline{d}(x, y) + \overline{d}(y, z)\).

% If \(d(x, y) \ge 1\) or \(d(y, z) \ge 1\), then RHS \(\ge 1\). Since LHS \(\le 1\), inequality holds.

% If both \(< 1\), then \(\overline{d} = d\), and triangle inequality holds for \(d\).

% \textbf{Same topology}:
% For \(\varepsilon < 1\), we have \(U_\varepsilon^{(d)}(x) = U_\varepsilon^{(\overline{d})}(x)\).

% Since small balls generate the topology, the topologies are identical.
% \end{proof}

\subsection{Metrics on Infinite Products}

\begin{example}
How to introduce a metric on \(\mathbb{R}^\omega = \mathbb{R}^\mathbb{N} = \text{Maps}(\mathbb{N} \to \mathbb{R}) \ni (x_1, x_2, \cdots)\).
\begin{enumerate}
    \item \(d(x, y) = \sqrt{\sum (x_i - y_i)^2}\). \textbf{Not a metric} (sum can be infinite).
    \item \(d(x, y) = \sup |x_i - y_i|\). (Can be \(\infty\)).
    \item \(\overline{d}(x, y) = \sup \overline{d}(x_i, y_i) = \sup \min(|x_i - y_i|, 1)\).
\end{enumerate}
\end{example}

\begin{definition}[Uniform Metric]
Define the \textbf{uniform metric} \(\overline{\rho}\) on \(\mathbb{R}^J\) (for any set \(J\)) by:
\[\overline{\rho}(x, y) = \sup_{\alpha \in J} \overline{d}(x_\alpha, y_\alpha) = \sup_{\alpha \in J} \min(|x_\alpha - y_\alpha|, 1)\]
\end{definition}

\begin{theorem}
The \textbf{Uniform topology} on \(\mathbb{R}^J\) is:
\begin{itemize}
    \item (Strictly) \textbf{finer} than the product topology on \(\mathbb{R}^J\).
    \item (Strictly) \textbf{coarser} than the box topology (provided \(J\) is infinite).
\end{itemize}
Diagram: Product \(\subset\) Uniform \(\subset\) Box.
\end{theorem}

\begin{remark}
Proof is commented out!
\end{remark}

% \begin{proof}
% \begin{itemize}
%     \item Comparison with Product Topology.

%     Take \(x \in \prod U_\alpha\) (basis element in product topology).

%     \(U_\alpha \subset \mathbb{R}\) open for \(\alpha = \alpha_1, \cdots, \alpha_n\); \(U_\alpha = \mathbb{R}\) otherwise.

%     For \(\alpha = \alpha_1, \cdots, \alpha_n\), take \(\varepsilon_1, \cdots, \varepsilon_n < 1\) such that \(U_{\varepsilon_i}(x_{\alpha_i}) \subset U_{\alpha_i}\).
    
%     Let \(\varepsilon = \min(\varepsilon_1, \cdots, \varepsilon_n)\), then \(U_\varepsilon^{(\overline{\rho})}(x) \subset \prod U_\alpha\), so Product \(\subset\) Uniform.

%     \item Comparison with Box Topology.
    
%     \(U_\varepsilon^{(\overline{\rho})}(x) = \prod U_\alpha\) is not quite generated by products?
    
%     Actually, \(U_\varepsilon^{(\overline{\rho})}(x) \subset \prod (x_\alpha - \varepsilon, x_\alpha + \varepsilon)\).
    
%     Box topology basis elements are arbitrary products. Box is finer.
% \end{itemize}
% \end{proof}

\begin{example}[Convergence Comparison]
\(x_n \to x\) in Box \(\implies x_n \to x\) in Uniform \(\implies x_n \to x\) in Product.
Example in \(\mathbb{R}^\omega\):
\begin{itemize}
    \item \(w_1 = (1, 1, 1, \cdots)\)
    \item \(w_2 = (0, 2, 2, 2, \cdots)\)
    \item \(w_3 = (0, 0, 3, 3, 3, \cdots)\)
\end{itemize}
\(\lim_{n \to \infty}^P w_n = (0, 0, 0, \cdots)\) (pointwise convergence). However, does it converge in Box or Uniform? In Box topology: a sequence \(x_n \to 0\) effectively requires \(x_n\) to be eventually 0 in all components simultaneously?

Problem: If \(x_n \to 0\) in box topology, what can be said about \(x_n\)?

Guess: \(\exists M: x_n^{(m)} = 0 \forall m > M\) (for components?).
\end{example}

\subsection{Sequence Lemma}

\begin{lemma}[The Sequence Lemma]
Let \(X\) be a topological space, \(A \subset X\).
If there is a sequence \(x_n \in A\) such that \(x_n \to x\), then \(x \in \overline{A}\). The \textbf{converse} is true if \(X\) is \textbf{metrizable}.
\end{lemma}

\begin{remark}
Proof is commented out!
\end{remark}

% \begin{proof}
% \begin{itemize}
%     \item[\(\Rightarrow\)]
%     If \(x_n \to x\), then \(\forall \text{ open } U \ni x\), \(U\) contains almost all terms \(x_n\). Thus,
%     \[U \cap A \neq \emptyset \implies x \in \overline{A}.\]

%     \item[\(\Leftarrow\)]
%     (If \(X\) is metrizable): Let \(x \in \overline{A}\) and take \(U_{1/n}(x)\). This is an open neighborhood of \(x\).

%     Pick \(x_n \in U_{1/n}(x) \cap A\) (non-empty since \(x \in \overline{A}\)), then \(d(x_n, x) < 1/n\).

%     For any \(\varepsilon > 0\), take \(N = 1/\varepsilon\). For \(n > N\), \(d(x_n, x) < \varepsilon\). Thus, \(x_n \to x\).
% \end{itemize}
% \end{proof}

\newpage

% Week 7
\section{Notes 13 - 11.24 - Nikita}

\subsection{Compact Spaces}

\begin{note}[Motivation]
(In analysis)
Let \(f: K \to \mathbb{R}\) be a continuous function, where \(K\) is a compact space (e.g., \(f: [0, 1] \to \mathbb{R}\)). Then \(f\) attains its maximum and minimum.
Being compact depends on topology.
For instance, in discrete topology, if \(X\) is infinite, it is not compact.
Board scrawl: \(\emptyset, X \in \mathcal{T} \implies X\) is compact (This refers to the trivial/anti-discrete topology).
\end{note}

\begin{definition}[Covering]
A collection \(\mathcal{A}\) of subsets of a space \(X\) is called a \textbf{``cover''} or \textbf{``a covering''} if the union of the elements of \(\mathcal{A}\) is equal to \(X\).
It is called an \textbf{open covering} if all these sets are \textbf{open}.
\[\{U_i\}_{i \in I}, \ U_i \text{ open}, \ \bigcup_{i \in I} U_i = X \ (\text{open covering of } X)\]
\end{definition}

\begin{definition}[Compact Space]
\(X\) is called a \textbf{compact space} if every open covering \(\mathcal{A}\) contains a \textbf{finite subcollection} which also covers \(X\).
(One can choose a finite subcovering from \textbf{any} covering of \(X\)).
\end{definition}

\begin{example}[Examples of Compactness]
    \leavevmode
    \begin{enumerate}
        \item \(X = (0, 1)\). \(U_n = (\frac{1}{n}, 1)\), \(n \in \mathbb{N}\). \(\bigcup U_n = X\).
        We cannot choose a finite subcovering. (Any finite union is \((\frac{1}{N}, 1) \neq X\)).
        \(\implies (0, 1) \subset \mathbb{R}\) is \textbf{not} compact in standard topology.

        \item \(\mathbb{R}\) with standard topology is \textbf{not compact}.
        \(U_n = (n, n+2)\), \(n \in \mathbb{Z}\). \(\bigcup U_n = \mathbb{R}\).
        Cannot choose a finite subcovering.

        \item \(Y = (0, 1]\) not compact.
        \(U_n = (\frac{1}{n}, 1]\). \(\bigcup U_n = Y\).
        Cannot choose a finite subcovering.

        \item \(\{0\} \cup \{1/n \mid n \in \mathbb{N}\} \subset \mathbb{R}\) \textbf{is compact}.
        The interval containing \(0\) (in the cover) contains all \(1/n\) for \(n\) big enough.

        \item \(\bigcup \{1/n \mid n \in \mathbb{N}\}\) is \textbf{not compact}.
    \end{enumerate}
\end{example}

\begin{remark}[Topologies and Compactness]
    \begin{itemize}
        \item In \textbf{discrete topology}, infinite sets are \textbf{not compact}.
        \item In \textbf{anti-discrete topology} (trivial topology), all sets are compact.
        \item \textbf{Finite spaces} are always compact.
    \end{itemize}
\end{remark}

\subsection{Compact Subspaces}

\begin{definition}[Cover of Subspace]
Let \(Y \subset X\). A collection \(\mathcal{A}\) of subsets of \(X\) \textbf{covers} \(Y\) if its union contains \(Y\).
\end{definition}

\begin{lemma}
Let \(Y \subset X\). Then \(Y\) is compact \(\iff\) from any open covering of \(Y\) \textbf{in \(X\)} one can choose a finite subcovering.
\end{lemma}

\begin{remark}
Proof is commented out!
\end{remark}

% \begin{proof}
% Topology on \(Y\): \(U \subset Y\) open in \(Y\) if \(\exists V \subset X\) open in \(X\) s.t. \(U = V \cap Y\).

% \begin{itemize}
%     \item[\(\Leftarrow\)]
%     Consider any family \(U_i\) of open sets in \(Y\), with \(\bigcup U_i = Y\).

%     \(\forall U_i, \exists V_i\) open in \(X\) s.t. \(U_i = V_i \cap Y\).

%     \(\implies \{V_i\}_{i \in I}\) is a covering of \(Y\) in \(X\) by open sets.
    
%     \(\implies\) One can choose a finite subcovering in \(X\), \(Y \subset V_{i_1} \cup \cdots \cup V_{i_k}\).
    
%     \(\implies Y = (V_{i_1} \cap Y) \cup \cdots \cup (V_{i_k} \cap Y) = U_{i_1} \cup \cdots \cup U_{i_k}\).

%     \item[\(\Rightarrow\)]
%     Consider any covering \(Y \subset \bigcup_{i \in I} V_i\), \(V_i \subset X\) open in \(X\).
    
%     \(Y = \bigcup_{i \in I} (V_i \cap Y)\). Note \((V_i \cap Y)\) is open in \(Y\).

%     Choose a finite subcovering \((V_{i_1} \cap Y) \cup \cdots \cup (V_{i_k} \cap Y) = Y\), then \(Y \subset \bigcup_{j=1}^k V_{i_j}\).
% \end{itemize}
% \end{proof}

\begin{theorem}
Every \textbf{closed subspace} of a \textbf{compact space} is compact.
\end{theorem}

\begin{remark}
Proof is commented out!
\end{remark}

% \begin{proof}
% Let \(Y \subset X\) be closed, \(X\) compact. Consider any open covering of \(Y\) in \(X\): \(Y \subset \bigcup_{i \in I} U_i\).

% \(\implies (\bigcup_{i \in I} U_i) \cup (X \setminus Y)\) is an open covering of \(X\) (\(X \setminus Y\) is open).

% \(\implies\) Choose a finite subcovering (\(X\) is compact).

% \(\implies\) This finite collection covers \(Y\) (and elements covering \(X \setminus Y\) can be discarded).

% \(\implies\) Finite subcovering of \(Y\).
% \end{proof}

\begin{theorem}[Compact in Hausdorff]
Every \textbf{compact subspace} of a \textbf{Hausdorff space} is \textbf{closed}.
(\(\mathbb{R}\) with standard topology is Hausdorff).
\end{theorem}

\begin{remark}
Proof is commented out!
\end{remark}

% \begin{proof}
% Let \(X\) be Hausdorff, \(Y \subset X\) compact. We show \(X \setminus Y\) is open.

% \(\iff \forall x \in X \setminus Y, \exists\) open \(U\) s.t. \(x \in U \subset X \setminus Y\).

% Pick any \(x \in X \setminus Y\).

% \(\forall y \in Y\), since \(x \neq y\) and \(X\) is Hausdorff, \(\exists U_y \ni y, V_y \ni x\) such that \(U_y \cap V_y = \varnothing\).

% Since \(Y \subset \bigcup_{y \in Y} U_y\) is compact, choose a finite subcovering \(Y \subset \bigcup_{i=1}^n U_{y_i}\).

% Consider \(V = \bigcap_{i=1}^n V_{y_i}\). This is an open set containing \(x\) (finite intersection of open sets).

% And \(V \cap Y = \varnothing\). (Proof: \(V \cap U_{y_i} \subset V_{y_i} \cap U_{y_i} = \varnothing \implies V \cap \bigcup U_{y_i} = \varnothing \implies V \cap Y = \varnothing\).)

% Thus, \(V \subset X \setminus Y\) and \(X \setminus Y\) is open.
% \end{proof}

\begin{lemma}[Separation]
Let \(Y\) be a compact subspace of Hausdorff \(X\), \(x \in X \setminus Y\).
Then \(\exists\) open \(U, V\) such that \(Y \subset U, x \in V, U \cap V = \varnothing\).
(Proved inside the Theorem above).
\end{lemma}

\subsection{Continuity and Compactness}

\begin{theorem}
The image of a compact set under a continuous map is compact.
\end{theorem}

\begin{remark}
Proof is commented out!
\end{remark}

% \begin{proof}
% Let \(f: X \to Y\) be continuous. \(X\) is compact.

% Consider open covering of \(f(X)\): \(f(X) \subset \bigcup_{i \in I} U_i\), \(U_i\) open in \(Y\).

% Since \(f\) is continuous, \(f^{-1}(U_i)\) is open in \(X \implies X = \bigcup_{i \in I} f^{-1}(U_i)\) (since \(f(X) \subset \bigcup U_i\)).

% Choose a finite subcovering \(X = \bigcup_{j=1}^n f^{-1}(U_{i_j})\), then \(f(X)\) is covered by \(U_{i_1}, \cdots, U_{i_n}\).
% \end{proof}

\begin{theorem}[Homeomorphism]
Let \(f: X \to Y\) be a continuous bijection. If \(X\) is compact and \(Y\) is Hausdorff, then \(f\) is a homeomorphism.
\end{theorem}

\begin{remark}
Proof is commented out!
\end{remark}

% \begin{proof}
% \(f^{-1}\) is continuous \(\iff\) images of open sets are open \(\iff f\) is a closed map \(\iff f(D)\) is closed \(\forall\) closed \(D \subset X\).

% Consider \(D \subset X\) closed.

% \(X\) compact \(\implies D\) is compact (closed in compact) \(\implies f(D)\) is compact in \(Y\).

% \(Y\) Hausdorff \(\implies f(D)\) is closed \(\implies f\) is a closed map.
% \end{proof}

\subsection{Product of Compact Spaces}

\begin{lemma}[Tube Lemma]
Let \(x_0 \in X\). Suppose \(Y\) is \textbf{compact}.
Suppose \(x_0 \times Y\) is covered by open sets \(W_i\) in \(X \times Y\).
Then one can choose a finite subcovering of it, and \(\exists\) open \(U \ni x_0\) such that
\[(U \times Y) \subset W_{i_1} \cup \cdots \cup W_{i_n}\]
\end{lemma}

\begin{remark}
Proof is commented out!
\end{remark}

% \begin{proof}
% May assume that all \(W_i\) are base sets \(W_i = U_i \times V_i\).

% \(\implies \{V_i\}\) covers \(Y\).

% \(\implies\) Use compactness of \(Y\), choose finite subcovering \(V_1, \cdots, V_n\),
% \[x_0 \times Y \subset (U_1 \times V_1) \cup \cdots \cup (U_n \times V_n) \supset U \times Y.\]

% Define \(U = \bigcap_{i=1}^n U_i = U_1 \cap U_2 \cap \cdots \cap U_n\), then \(U \times Y \subset \bigcup (U_i \times V_i)\) (since \(U \subset U_i\)).
% \end{proof}

\begin{theorem}[Tychonoff, Finite Case]
The product of finitely many compact spaces is compact.
\end{theorem}

\begin{remark}
Proof is commented out!
\end{remark}

% \begin{proof}
% Enough to prove for two sets, \(X, Y\) compacts.

% \(X \times Y\) (Any open set in \(X \times Y\) is a union of base spaces \(U \times V, U \subset X, V \subset Y\) open).

% Consider any open covering of \(X \times Y\) by \(W_i, i \in I\).

% \(\forall x \in X\), choose a finite subcovering of \(x \times Y\), and \(U_x\) from the Tube Lemma.

% \(X = \bigcup_{x \in X} U_x\), choose finite subcovering (\(X\) is compact):
% \[X = U_{x_1} \cup \cdots \cup U_{x_k}.\]

% Now take all finite coverings of \(U_{x_j} \times Y \implies\) obtained a finite subcovering of \(X \times Y\).
% \end{proof}

\subsection{Finite Intersection Property}

\begin{definition}[Finite Intersection Property]
A collection \(\mathcal{C}\) of sets has the \textbf{finite intersection property} if for any finite subcollection \(C_1, \cdots, C_n \in \mathcal{C}\),
\[\bigcap_{i=1}^n C_i \text{ is non-empty.}\]
\end{definition}

\begin{theorem}
Let \(X\) be a topological space. Then \(X\) is compact \(\iff\)
every collection of \textbf{closed} sets with finite intersection property has non-empty intersection:
\[\bigcap_{C \in \mathcal{C}} C \neq \varnothing\]
\end{theorem}

\begin{remark}
Proof is commented out!
\end{remark}

% \begin{proof}
% Let \(\mathcal{A}\) be a collection of open sets and \(\mathcal{C} = \{X \setminus A \mid A \in \mathcal{A}\}\) be the collection of complements (closed sets).
% \begin{enumerate}
%     \item \(\mathcal{A}\) open sets \(\iff \mathcal{C}\) collection of closed sets.
%     \item \(\mathcal{A}\) covers \(X \iff \bigcap_{C_i \in \mathcal{C}} C_i = \varnothing\).
%     \item A finite \(A_1, \cdots, A_n \in \mathcal{A}\) covers \(X \iff \bigcap_{i=1}^n (X \setminus A_i) = \varnothing\).
% \end{enumerate}

% Now, the Theorem follows.
% \end{proof}

\newpage

\section{Notes 14 - 12.01}

\subsection{Metrizability and Convergence}

\begin{recap}[The Sequence Lemma]
Let \(X\) be a topological space, \(A \subset X\).
\begin{itemize}
    \item If there exists a sequence \(x_n \in A\) such that \(\lim_{n \to \infty} x_n = x\), then \(x \in \overline{A}\).
    \item The converse is true if \(X\) is \textbf{metrizable} (or first-countable).
\end{itemize}
\end{recap}

\begin{remark}
Proof is commented out!
\end{remark}

% \begin{proof}
% \begin{itemize}
%     \item[\(\Rightarrow\)]
%     Take \(U \ni x\) open. Then \(x_n \in U\) for \(n \gg 0 \implies U \cap A \neq \emptyset \implies x \in \overline{A}\).

%     \item[\(\Leftarrow\)] (Assuming first-countable/metrizable.) 
    
%     Take \(U_n = U_{1/n}(x)\) (in metric case), \(x_n \in U_{1/n}(x) \cap A\) (exists since \(x \in \overline{A}\)).

%     If \(m > n\), \(U_{1/n}(x) \supset U_{1/m}(x)\), so \(x_m \in U_{1/n}(x)\) for each \(m > n\).
    
%     Any open set \(U\) contains \(U_\varepsilon = U_{1/n}\) for \(n\) large enough, so \(U \ni x_m\) for \(m > n\).
% \end{itemize}
% \end{proof}

\begin{theorem}[Heine's Definition of Limit]
Let \(f: X \to Y\). If \(f\) is continuous, then for every \(x_n \in X, x_n \to x \in X\), we have \(f(x_n) \to f(x)\).
The converse holds if \(X\) is metrizable (or first-countable).
\end{theorem}

\begin{remark}
Proof is commented out!
\end{remark}

% \begin{proof}
% \begin{itemize}
%     \item[\(\Rightarrow\)]
%     Let \(f\) be continuous and take \(V \ni f(x)\), \(V \subset Y\) open \(\implies f^{-1}(V) \subset X\) open.

%     Then \(f^{-1}(V) \ni x_n\) for \(n \gg 0\). Thus, \(f(x_n) \in f(f^{-1}(V)) \subset V\).
    
%     \item[\(\Leftarrow\)]
%     Recall that \(f\) is continuous \(\iff \forall A \subset X, f(\overline{A}) \subset \overline{f(A)}\).

%     Take \(A \subset X, x \in \overline{A}\). Then \(\exists x_n \in A, x_n \to x\) (since \(X\) metrizable/1st countable).

%     Then \(f(x_n) \to f(x)\), so \(f(x) \in \overline{f(A)} \implies f(\overline{A}) \subset \overline{f(A)}\).
% \end{itemize}
% \end{proof}

\subsection{Countability Axioms}

\begin{definition}[First-Countable Space]
\(X\) is \textbf{first-countable} if it has a countable basis at each \(x \in X\).
We use the following:
Given \(x\), there is a \textbf{countable} family of neighborhoods \(\mathcal{U} = \{U_1, U_2, \cdots\}\) (can assume nested \(U_1 \supset U_2 \supset \cdots\)) such that
\[\forall U \ni x \text{ open}, \ \exists n \text{ s.t. } U \supset U_n\]
\end{definition}

\begin{definition}[Second-Countable Space]
\(X\) is \textbf{second-countable} if it has a \textbf{countable basis} (for the topology on \(X\)). There exists a countable basis \(\mathcal{B}\) such that \(\forall x \in X, \forall U \ni x\) open, \(\exists B \in \mathcal{B}\) s.t. \(x \in B \subset U\).
\[\text{2nd-countable} \implies \text{1st-countable}\]
\end{definition}

\subsection{Examples and Counterexamples}

\begin{example}[Examples of Countability]
    \leavevmode
    \begin{enumerate}
        \item[\bfseries Ex 1.] \textbf{\(\mathbb{R}^n\) is 2nd-Countable}
        Has a countable basis \(\{U_\varepsilon(x) \mid x \in \mathbb{Q}^n, \varepsilon \in \mathbb{Q}_{>0}\}\).
        It is also 1st-countable.

        \item[\bfseries Ex 2.] \textbf{Finite Complement Topology on \(\mathbb{R}\)}
        Not 1st-countable.
        \(x \in U\) open \(\implies U = \mathbb{R} \setminus \{y_1, \cdots, y_k\}\).
        Suppose there exists a countable basis at \(x\): \(U_n = \mathbb{R} \setminus F_n\) (finite sets).
        Consider \(U = \mathbb{R} \setminus \{y\}\) where \(y \notin \bigcup F_n\) and \(y \neq x\).
        Then \(U\) is open neighborhood of \(x\), but \(U \not\supset U_n\) for any \(n\) (since \(y \in U_n\)).

        \item[\bfseries Ex 3.] \textbf{Uncountable Discrete Space}
        \(X\) uncountable with discrete topology.
        Then \(\forall x \in X\), the set \(\{x\}\) is open. So any basis of \(X\) should contain \(\{\{x\}\}_{x \in X}\).
        \(\implies X\) not 2nd-countable.
        But \(X\) is metrizable (discrete metric) \(\implies\) 1st-countable.

        \item[\bfseries Ex 4.] \textbf{\(\mathbb{R}^2\) with ``Amazon River metric''}
        \(d((x,y), (x',y')) = |y-y'|\) if \(x=x'\), else \(|y| + |x'-x| + |y'|\).
        Basis elements at \((x,y_0)\) look like vertical segments \(\{ (x,y) \mid x=x_0, y \in (y_0-\varepsilon, y_0+\varepsilon) \}\).
        \(\implies\) No countable basis (similar to discrete on the x-axis). Not 2nd-countable.
        But metric \(\implies\) 1st-countable.

        \item[\bfseries Ex 5.] \textbf{\(\mathbb{R}_\ell\) (Lower Limit Topology)}
        Basis \([a, b)\).
        \textbf{Not 2nd-countable}: Let \(\{B_n\}\) be a countable basis.
        \(x \in \mathbb{R}_\ell\). \(U = [x, \infty) \implies \exists n: x \in B_n \subset [x, \infty)\).
        Then \(x = \min B_n\).
        So \(\forall x \in \mathbb{R}_\ell, \exists B_n: x = \min B_n\).
        But there are countably many \(B_n\), uncountably many \(x\). Contradiction.
        \textbf{1st-countable}: \(x \in [x, x+1/n)\) is a countable local basis.
    \end{enumerate}
\end{example}

\subsection{Metrizability of Product Spaces}

\begin{remark}
Using the diagonal argument (sequence lemma) to disprove metrizability.
\(J\) is infinite:
\[\text{Box Top.} \supset \text{Uniform Metric Top.} \supset \text{Product Top.}\]
(Box is finer).
\end{remark}

\begin{proposition}
\(\mathbb{R}^\omega\) with \textbf{box topology} is not metrizable (hence \(\mathbb{R}^J\) not metrizable for any \(J\) infinite with box topology).
\end{proposition}

\begin{remark}
Proof is commented out!
\end{remark}

% \begin{proof}
% \(A = \{(x_1, x_2, \cdots) \mid x_i > 0\}\) (\(0 \in \overline{A}\)).

% Check: \(0 \in (a_1, b_1) \times (a_2, b_2) \times \cdots\) (basis element with \(a_i < 0 < b_i\) \(\forall i\)), then
% \[(\frac{b_1}{2}, \frac{b_2}{2}, \cdots) \in B \cap A.\]

% There is no sequence \((a_n) \in A\) such that \(a_n \to 0\).

% Let \(a_n = (x_{1n}, x_{2n}, \cdots)\) and take the box \(B = \prod_{k=1}^\infty (-x_{kk}, x_{kk})\).

% Then \(a_n \notin B\) for all \(n\) (since the \(n\)-th coordinate is \(x_{nn} \notin (-x_{nn}, x_{nn})\)) \(\implies a_n \not\to 0\).
% \end{proof}

\begin{proposition}
Let \(J\) be uncountable. Then \(\mathbb{R}^J\) in \textbf{product topology} is not metrizable.
\end{proposition}

\begin{remark}
Proof is commented out!
\end{remark}

% \begin{proof}
% \(A = \{(x_\alpha) \mid x_\alpha = 1 \text{ for all but finitely many coordinates} \}\) (\(0 \in \overline{A}\)).

% Let \(U = \prod U_\alpha\) be a basic open set of \(0\) (\(U_\alpha = \mathbb{R}\) if \(\alpha \neq \alpha_1, \cdots, \alpha_n\)).

% Then take \(x_\alpha = \begin{cases} 0 & \text{if } \alpha = \alpha_1, \cdots, \alpha_n \\ 1 & \text{if } \alpha \neq \alpha_1, \cdots, \alpha_n \end{cases} \implies x_\alpha \in U \cap A \implies 0 \in \overline{A}\).

% No sequence \(a^{(n)} \to 0\), \(a^{(n)} \in A\):
% Let \(a^{(n)} = (x_\alpha^{(n)})\) be such a sequence.

% \(\exists \beta \in J: x_\beta^{(n)} = 1 \forall n\) and take \(U = \pi_\beta^{-1}((-1, 1))\) (cylinder).

% Then \(U \not\ni a^{(n)} \forall n\) (since \(x_\beta^{(n)} = 1 \notin (-1, 1)\)) \(\implies a^{(n)} \not\to 0\).
% \end{proof}

\newpage

% Week 8
\section{Notes 15 - 12.05}

\subsection{Metrizability of Infinite Products}

\begin{recap}
\begin{enumerate}
\item \(\mathbb{R}^\omega\) is \textbf{not} metrizable in the \textbf{box topology}.
\item If \(J\) is uncountable, \(\mathbb{R}^J\) is \textbf{not} metrizable in the \textbf{product topology}.
\end{enumerate}
\end{recap}

\begin{theorem}
\(\mathbb{R}^\omega\) is \textbf{metrizable} in the \textbf{product topology}.
\end{theorem}

\begin{remark}
Proof is commented out!
\end{remark}

% \begin{proof}
% \(\mathbb{R}^\omega = \{(x_1, x_2, \cdots) \mid x_i \in \mathbb{R}\}\). Let \(\bar{d}(a, b) = \min(|a - b|, 1)\) (standard bounded metric on \(\mathbb{R}\)).

% Define \(D(x, y) = \sup_{i \in \mathbb{Z}_{>0}} \left\{ \frac{\bar{d}(x_i, y_i)}{i} \right\}\).

% \paragraph*{(1) This is a metric}
% \begin{itemize}
% \item Positive, zero iff equal, symmetric: clear.
% \item Triangle inequality: \(D(x, z) \le D(x, y) + D(y, z)\).
% \[\frac{\bar{d}(x_i, z_i)}{i} \le \frac{\bar{d}(x_i, y_i) + \bar{d}(y_i, z_i)}{i} \le D(x, y) + D(y, z).\]
% Taking sup over \(i\), we get \(D(x, z) \le D(x, y) + D(y, z)\).
% \end{itemize}

% \paragraph*{(2) It defines the product topology}
% We need to show the topologies strictly match.
% \begin{itemize}
% \item[(a)] \textbf{Metric \(\implies\) Product}:
% For any \(U\) open in metric topology, show it is open in product topology. \(\forall x \in U, \exists \varepsilon > 0\) such that \(U_D(x, \varepsilon) \subset U\). We find a basic open set \(V\) in product topology s.t. \(x \in V \subset U_D(x, \varepsilon)\). Take \(N\) such that \(\frac{1}{N} < \varepsilon\). Let \(V = (x_1 - \varepsilon, x_1 + \varepsilon) \times \cdots \times (x_N - \varepsilon, x_N + \varepsilon) \times \mathbb{R} \times \mathbb{R} \times \cdots\), then \(V \subset U_D(x, \varepsilon)\).

% (Check: let \(y \in V\). If \(i \le N\), \(\frac{\bar{d}(x_i, y_i)}{i} \le \frac{\varepsilon}{1} = \varepsilon\). If \(i > N\), \(\frac{\bar{d}(x_i, y_i)}{i} \le \frac{1}{i} < \frac{1}{N} < \varepsilon\). Thus, \(D(x, y) \le \max(\cdots) < \varepsilon\)).

% \item[(b)] \textbf{Product \(\implies\) Metric}:
% Let \(U = \prod U_i\) be basic open in product topology (\(U_i = \mathbb{R}\) for \(i > n\)). Let \(x \in U\). Need: \(U_D(x, \varepsilon) \subset U\). Since \(U_i\) is open in \(\mathbb{R}\), \(\exists \varepsilon_i > 0\) s.t. \((x_i - \varepsilon_i, x_i + \varepsilon_i) \subset U_i\) (for \(i = 1 \cdots n\)). Let \(\varepsilon = \min_{i=1 \cdots n} \{ \frac{\varepsilon_i}{i} \}\). Claim: \(U_D(x, \varepsilon) \subset U\). Let \(y \in U_D(x, \varepsilon)\), then
% \[\frac{\bar{d}(x_i, y_i)}{i} \le D(x, y) < \varepsilon \le \frac{\varepsilon_i}{i} \implies \bar{d}(x_i, y_i) < \varepsilon_i \implies |x_i - y_i| < \varepsilon_i \implies y_i \in U_i.\]
% \end{itemize}
% \end{proof}

\subsection{Properties of Ordered Sets and Compactness}

\subsubsection{Least Upper Bound Property}
\begin{note}[Least Upper Bound Property]
Let \(X\) be a linearly ordered set with \(<\) topology. Basis: \((a, b) = \{x \in X \mid a < x < b\}\).

\(A \subset X\): \(A\) is \textbf{bounded from above} if \(\exists b \in X\) s.t. \(a \le b, \forall a \in A\).

\(b\) is an upper bound if \(b' < b \implies \exists a \in A\) s.t. \(b' < a\) (implies \(b\) is \textbf{least} upper bound).
\end{note}

\begin{definition}[Least Upper Bound Property]
\((X, \le)\) has the \textbf{least upper bound property} if every nonempty subset \(A \subset X\) bounded from above has \(\sup A \in X\).
\end{definition}

\begin{theorem}[Compactness of Intervals]
Let \(X\) be an ordered set with the least upper bound property. Then \(\forall a \le b \in X\), the interval \([a, b]\) is compact.
\end{theorem}

\begin{remark}
Proof is commented out!
\end{remark}

% \begin{proof}
% Let \(\mathcal{A}\) be an open covering of \([a, b]\).

% \textbf{Step 1}: For all \(x \in [a, b], x \neq b\), there exists \(y > x\) such that \([x, y]\) is covered by at most two intervals from \(\mathcal{A}\).
% \begin{itemize}
% \item If \(x\) has an immediate successor \(y \in X\), take it. Then \([x, y] = \{x, y\}\), which can be covered by 2 sets from \(\mathcal{A}\).
% \item If not: Take \(A \in \mathcal{A}\) such that \(x \in A\). Since \(A\) is open, there exists \(z > x\) such that \([x, z) \subset A\). Take any \(y\) such that \(x < y < z\). Then \([x, y] \subset A\) is covered by \(A\).
% \end{itemize}

% \textbf{Step 2}: Let \(C\) be the set of points \(y > a\) such that \([a, y]\) has a finite subcover. By Step 1 applied to \(a\), \(C \neq \emptyset\). Also \(C\) is bounded by \(b\). Let \(c = \sup C \le b\).

% \textbf{Step 3}: Show that \(c \in C\).

% Let \(A \in \mathcal{A}\) such that \(c \in A\). There exists \(d < c\) such that \((d, c] \subset A\). Since \(c = \sup C\), there exists \(z \in C\) such that \(d < z \le c\). \([a, z]\) has a finite subcover. Adding \(A\) covers \([a, c]\). Then
% \[c \in C.\]

% \textbf{Step 4}: Show that \(c = b\).

% Suppose \(c < b\). By Step 1, \(\exists y > c\) such that \([c, y]\) is covered by at most 2 sets. Then \([a, y] = [a, c] \cup [c, y]\) has a finite subcover, so \(y \in C\), contradicting \(c = \sup C\)! Thus, \(c = b\).
% \end{proof}

\begin{corollary}
\([a, b] \subset \mathbb{R}\) is compact.
\end{corollary}

\begin{corollary}
\([a_1, b_1] \times \cdots \times [a_n, b_n] \subset \mathbb{R}^n\) is compact (Tychonoff/Product of compacts).
\end{corollary}

\begin{theorem}[Heine-Borel]
A subset \(A \subset \mathbb{R}^n\) is compact \(\iff\) \(A\) is closed and bounded (in Euclidean metric).
\end{theorem}

\begin{remark}
Proof is commented out!
\end{remark}

% \begin{proof}
% \begin{itemize}
% \item[\(\Rightarrow\)] \(A\) compact \(\implies A\) closed (subset of Hausdorff \(\mathbb{R}^n\)).

% Take covering by balls \(U_N(0)\) of radii \(1, 2, \cdots\). They cover \(A \implies\) there is a finite subcover \(U_{N_1}, \cdots, U_{N_k}\).
% Then \(A \subset U_N\) where \(N = \max(N_i)\). Thus, \(A\) is bounded.

% \item[\(\Leftarrow\)] \(A\) bounded \(\implies A \subset [-M, M]^n\) (large box).

% The box is compact (product of closed intervals). \(A\) is a closed subset of a compact set \(\implies A\) is compact.
% \end{itemize}
% \end{proof}

\begin{theorem}[Extreme Value Theorem]
Let \(f: X \to Y\) be continuous. \(Y\) ordered, \(X\) compact.
Then \(\exists c, d \in X\) such that
\[f(c) = \min f \le f(x) \le \max f = f(d) \ \forall x \in X\]
\end{theorem}

\begin{remark}
Proof is commented out!
\end{remark}

% \begin{proof}
% \(f(X) \subset Y\) is compact. Show that a compact subset of an ordered set has a largest element.

% Let \(A = f(X)\) and \(M = \sup A\) (exists by LUB property if \(Y=\mathbb{R}\), or construct cover).

% Take \(\{(-\infty, a) \mid a \in A\}\). If there is no largest element, this is an open covering of \(A\). Finite subcovering \((-\infty, a_1) \cup \cdots \cup (-\infty, a_n)\). Let \(a_{\max} = \max(a_i)\), then \(A \subset (-\infty, a_{\max})\).

% Contradiction! (Since \(a_{\max}\) is not covered.)
% \end{proof}

\subsection{Distance to a Set}

\begin{definition}
Let \((X, d)\) be a metric space, \(A \subset X, x \in X\).
\[d(x, A) = \inf \{ d(x, a) \mid a \in A \}\]
\end{definition}

\begin{proposition}
\(d(x, A)\) is a continuous function (for \(A\) fixed).
\end{proposition}

\begin{remark}
Proof is commented out!
\end{remark}

% \begin{proof}
% \(d(x, a) \le d(x, y) + d(y, a)\).

% Taking inf over \(a \in A\):
% \[d(x, A) \le d(x, y) + d(y, A) \implies d(x, A) - d(y, A) \le d(x, y).\]

% Symmetrically,
% \[d(y, A) - d(x, A) \le d(x, y) \implies |d(x, A) - d(y, A)| \le d(x, y).\]

% This implies continuity (Lipshitz continuous).
% \end{proof}

\begin{remark}[Goal]
Show that if \(f: X \to Y\) is continuous, \(X, Y\) metric spaces, \(X\) compact \(\implies f\) uniformly continuous.
\end{remark}

\newpage

\section{Notes 16 - 12.08}

\subsection{Norms on \(\mathbb{R}^n\)}

\begin{definition}[Norm]
A \textbf{norm} on \(\mathbb{R}^n\) is a function \(\| \cdot \|: \mathbb{R}^n \to \mathbb{R}\) satisfying
\begin{enumerate}
    \item \(\|x\| \ge 0\), and \(\|x\| = 0 \iff x = 0\).
    \item \(\|x + y\| \le \|x\| + \|y\|\) (Triangle Inequality).
    \item \(\|\lambda x\| = |\lambda| \cdot \|x\|\) for all \(x \in \mathbb{R}^n, \lambda \in \mathbb{R}\).
\end{enumerate}
\end{definition}
\begin{note}
Norm \(\to\) Metric \(\to\) Topology.
\end{note}

\begin{example}
    \leavevmode
    \begin{itemize}
        \item \(\|x\|_1 = \sum |x_i|\)
        \item \(\|x\|_2 = (\sum x_i^2)^{1/2}\) (Euclidean)
        \item \(\|x\|_\infty = \max |x_i|\)
    \end{itemize}
\end{example}

\begin{definition}[Equivalence of Norms]
Two norms \(\| \cdot \|_a, \| \cdot \|_b\) are \textbf{equivalent} if \(\exists m, M \in \mathbb{R}_{>0}\) such that
\[m \|x\|_a \le \|x\|_b \le M \|x\|_a\]
This implies \(U_\varepsilon^{(b)}(0) \subseteq U_{M \varepsilon}^{(a)}(0) \subseteq U_{(M/m) \varepsilon}^{(b)}(0)\) (Topology is the same).
\end{definition}

\begin{proposition}
Equivalence of norms is an equivalence relation.
\end{proposition}

\begin{theorem}
All norms on \(\mathbb{R}^n\) are equivalent.
\end{theorem}

\begin{remark}
Proof is commented out!
\end{remark}

% \begin{proof}
% Given \(\| \cdot \|_a, \| \cdot \|_b\). Let \(\| \cdot \|_b\) be the Euclidean norm \(\| \cdot \|_2\).

% Define a function \(f: \mathbb{R}^n \setminus \{0\} \to \mathbb{R}_{>0}\), \(f(x) = \frac{\|x\|_a}{\|x\|_b}\), which is a continuous function on \(\mathbb{R}^n \setminus \{0\}\).

% \(f(x) = f(\lambda x)\) for any \(\lambda \neq 0\) (homogeneity cancels out).

% So \(f(x)\) is completely determined by \(f|_{S^{n-1}}\), where \(S^{n-1} = \{ x \mid \|x\|_b = 1 \}\).

% (\(S^{n-1}\) is compact (closed and bounded in Euclidean metric).)

% Then \(f(x)\) has maximum and minimum values on \(S^{n-1}\). Let \(m = \min_{S^{n-1}} f\), \(M = \max_{S^{n-1}} f\).

% Since \(\|x\|_a > 0\) for \(x \neq 0\), \(m > 0\), then \(0 < m \le \frac{\|x\|_a}{\|x\|_b} \le M < \infty\). Thus, \(\| \cdot \|_a \sim \| \cdot \|_b\).
% \end{proof}

\begin{remark}
In infinite-dimensional spaces, norms are \textbf{not} always equivalent. \(S^\infty\) is \textbf{not compact} (closed and bounded, but not compact). Example: \(\ell_1 = \{ (x_k) \mid \sum |x_k| < \infty \}\).
\end{remark}

\subsection{Uniform Continuity}

\begin{remark}[Goal]
Prove that any continuous map between metric spaces \((X, d_X) \to (Y, d_Y)\), with \(X\) \textbf{compact}, is \textbf{uniformly continuous}.
\end{remark}

\begin{lemma}[The Lebesgue Number Lemma]
Let \((X, d)\) be a metric compact space and \(\mathcal{A}\) be an open covering of \(X\).

Then \(\exists \delta > 0\) (called a \textbf{Lebesgue number}) such that for any subset \(Y \subset X\) with \(\text{diam}(Y) < \delta\), we have \(Y \subset A\) for some \(A \in \mathcal{A}\). (\(\text{diam}(Y) = \sup \{ d(x, y) \mid x, y \in Y \}\)).
\end{lemma}

\begin{remark}
Proof is commented out!
\end{remark}

% \begin{proof}
% If \(X \in \mathcal{A}\), nothing to prove (any \(\delta\) works).

% Otherwise: take a finite subcovering \(\{A_1, \cdots, A_n\} \subset \mathcal{A}\).

% Let \(C_i = X \setminus A_i\), closed subsets in compact \(X\), and hence compact.

% For any \(x \in X\), \(x \in A_i\) for some \(i \implies x \notin C_i \implies d(x, C_i) \ge 0\).

% Define \(f(x) = \frac{1}{n} \sum_{i=1}^n d(x, C_i)\) (average distance). Show that \(f(x) > 0\):

% \(\forall x \in X\), choose \(i\) such that \(A_i \ni x\). Choose \(\varepsilon\) so that \(U_\varepsilon(x) \subset A_i, d(x, C_i) \ge \varepsilon\). \(f(x) \ge \frac{\varepsilon}{n}\).

% Let \(\delta = \min f(x) > 0\), then \(\delta\) is a Lebesgue number:

% Take \(B \subset X, \text{diam}B < \delta, x_0 \in B\). \(B \subset U_\delta(x_0)\).

% \(\delta \le f(x_0) \le d(x_0, C_m)\) (take \(\max d(x_0, C_i)\)) for some \(1 \le m \le n\).

% \(U_\delta(x_0) \subset A_m = X \setminus C_m\), so \(B \subset A_m\).
% \end{proof}

\begin{definition}[Uniform Continuity]
A function \(f: (X, d_X) \to (Y, d_Y)\) is \textbf{uniformly continuous} if
\(\forall \varepsilon > 0, \exists \delta > 0\) such that for each \(x, x' \in X\),
with \(d_X(x, x') < \delta\), we have \(d_Y(f(x), f(x')) < \varepsilon\).
\end{definition}

\begin{theorem}
Let \(f: X \to Y\) be a continuous map of metric spaces.
If \(X\) is \textbf{compact}, then \(f\) is uniformly continuous.
\end{theorem}

\begin{remark}
Proof is commented out!
\end{remark}

% \begin{proof}
% Take \(\varepsilon > 0\). Consider the covering of \(Y\) by open balls \(\{ U_{\varepsilon / 2}(y) \}_{y \in Y}\). Take \(\mathcal{A}\) to be the collection of preimages: \(\mathcal{A} = \{ f^{-1}(U_{\varepsilon / 2}(y)) \mid y \in Y \}\).

% This is an open covering of \(X\). Let \(\delta\) be a \textbf{Lebesgue number} of this covering \(\mathcal{A}\). Then if \(x, x' \in X\) and \(d_X(x, x') < \delta\), then the set \(\{x, x'\}\) has diameter \(< \delta\) and \(\{x, x'\} \subset f^{-1}(U_{\varepsilon / 2}(y))\) for some \(y \in Y \implies f(x), f(x') \in U_{\varepsilon / 2}(y) \implies d_Y(f(x), y) < \varepsilon / 2\) and \(d_Y(f(x'), y) < \varepsilon / 2\).

% By triangle inequality, \(d_Y(f(x), f(x')) < \varepsilon\).
% \end{proof}

\subsection{Cardinality of Compact Spaces}

\begin{theorem}[Cardinality of Compact Sets]
Let \(X\) be a non-empty \textbf{compact Hausdorff} space \textbf{without isolated points} (perfect space).
Then \(X\) is \textbf{uncountable}.
\end{theorem}

\begin{remark}
Proof is commented out!
\end{remark}

% \begin{proof}
% \textbf{Step 1}: Given \(x \in U\) open, show that \(\exists\) open \(V \subset U\), \(V \neq \emptyset\), such that \(\overline{V} \subset U\) and \(x \notin \overline{V}\).

% Choose \(y \in U, y \neq x\). Let \(W_1, W_2\) be disjoint open neighborhoods of \(x\) and \(y\) and \(V = W_2 \cap U\).

% Then \(V\) is open and non-empty. Thus, \(x \notin \overline{V}\) (since \(W_1 \cap V = \varnothing\)).

% \textbf{Step 2}: Given \(f: \mathbb{N} \to X\). We show \(f\) is not surjective.

% Apply Step 1 to \(x_0 = f(0), U = X\) and choose \(V_0\) open, \(\overline{V_0} \subset X, f(0) \notin \overline{V_0}\).

% Choose \(V_1 \subset V_0\) such that \(\overline{V_1} \subset V_0\) and \(f(1) \notin \overline{V_1}\).

% Inductively, \(\overline{V_0} \supset \overline{V_1} \supset \overline{V_2} \supset \cdots\).

% This is a nested sequence of non-empty closed sets in a compact set \(X \implies \bigcap_{n \in \mathbb{N}} \overline{V_n} \neq \emptyset\).

% Let \(x \in \bigcap \overline{V_n}\), then \(x \neq f(n)\) for any \(n\) (since \(x \in \overline{V_n} \subset V_{n-1}\) and \(f(n) \notin \overline{V_n}\)).

% Thus, \(f\) is not surjective.
% \end{proof}

\begin{corollary}
\([a, b] \subset \mathbb{R}\) is uncountable.
\end{corollary}

\begin{note}[One-Point Compactification (Preview)]
Idea: Given locally compact \(X\), construct \(Y = X \cup \{\infty\}\).
Topology:
\begin{enumerate}
    \item Open in \(X \implies\) Open in \(Y\).
    \item \(Y \setminus C\) where \(C \subset X\) compact \(\implies\) Open in \(Y\).
\end{enumerate}
Need to check topology axioms. \(Y\) is Compact Hausdorff. (Next Time.)
\end{note}

\newpage

% Week 9
\section{Notes 17 - 12.12}

\subsection{Different Notions of Compactness}

\begin{definition}[Compact]
\(X\) is \textbf{compact} if any open covering admits a finite subcovering.
\end{definition}

\begin{definition}[LPC]
\(X\) is \textbf{limit point compact} (LPC) if every infinite \(A \subset X\) has a limit point. (Fréchet compactness).
\end{definition}

\begin{definition}[SC]
\(X\) is \textbf{sequentially compact} (SC) if every sequence \(x_1, x_2, \cdots\) has a convergent subsequence. (\(\exists n_1 < n_2 < \cdots\) s.t. \(x_{n_k}\) converges). (Bolzano-Weierstrass property).
\end{definition}

\begin{theorem}[Theorem 1]
If \(X\) is a \textbf{metric space}, then
\(X\) is compact \(\iff\) LPC \(\iff\) SC.
\end{theorem}

\begin{theorem}[Theorem 2]
For \(X\) an arbitrary \textbf{topological space}, \(X\) compact \(\implies X\) LPC. (The converse is \textbf{NOT} true).
\end{theorem}

\begin{proof}[Proof (Compact \(\implies\) LPC)]
Let \(X\) be compact, \(A \subset X\) infinite. Assume \(A\) has no limit points, then \(A\) is closed (contains all its limit points, which are none). So \(X \setminus A\) is open.

Since \(a \in A\) is \textbf{not} a limit point, \(\exists U_a\) open such that \(U_a \cap A = \{a\}\).

Take all \(\{U_a\}_{a \in A}\) and \(X \setminus A\). This is an open cover of \(X\).

Since \(X\) is compact, there exists a finite subcover,
\[X \subset (X \setminus A) \cup U_{a_1} \cup \cdots \cup U_{a_k}.\]

Since \(U_{a_i} \cap A = \{a_i\}\), we must have \(A \subset \{a_1, \cdots, a_k\} \implies A\) is finite, contradiction!
\end{proof}

\begin{example}[Counterexample: LPC but not Compact]
Let \(Y = \{p, q\}\) with anti-discrete topology \(\mathcal{T}_Y = \{\emptyset, Y\}\). Let \(X = \mathbb{N} \times Y\) with product topology (\(\mathbb{N}\) is discrete).

Open sets in \(X\) look like \(U \times Y\) where \(U \subset \mathbb{N}\). Every nonempty open set contains pairs \(\{(n, p), (n, q)\}\) for \(n \in U\). Then every nonempty set \(A \subset X\) has a limit point.

Suppose \(A \ni (n, p)\). Then \((n, q)\) is a limit point of \(A\): every open neighborhood of \((n, q)\) contains \((n, p)\) (because open sets come in ``columns'') \(\implies X\) is LPC.

But \(X\) is \textbf{not compact}: Cover by \(U_n = \{n\} \times Y\). No finite subcover. \(X\) is also \textbf{not sequentially compact}: The sequence \((1, p), (2, p), \cdots\) has no convergent subsequence.
\end{example}

\begin{proof}[Partial Proof of Theorem 1 (Metric Spaces)]
\begin{enumerate}
    \item \textbf{Compact \(\implies\) LPC}: Theorem 2.
    \item \textbf{LPC \(\implies\) SC}: (Proof works for 1st-countable spaces).
    Take a sequence \((x_n)\) in \(X\).
    \begin{itemize}
        \item
        If the set of values of \((x_n)\) is finite, then one value is assumed infinitely many times \(\implies\) constant subsequence \(\implies\) convergent.

        \item
        If \((x_n)\) has infinitely many values, let \(A\) be the set of values.
        
        By LPC, \(A\) has a limit point \(a \in X\). Since \(X\) is metric (or 1st-countable), we can find a subsequence converging to \(a\). (Take \(U_1(a) \supset U_2(a) \supset \cdots\) and pick \(x_{n_k} \in U_k(a)\) with increasing indices.)
    \end{itemize}
    \item \textbf{SC \(\implies\) Compact}: Harder.
\end{enumerate}
\end{proof}

\subsection{Locally Compact Spaces}

\begin{definition}[Locally Compact Space]
\(X\) is \textbf{locally compact at \(x\)} if \(\exists\) open \(U \ni x\), and a \textbf{compact subspace} \(C\) of \(X\) such that \(x \in U \subset C\).
\(X\) is \textbf{locally compact} if it is locally compact at every \(x \in X\).
\end{definition}

\begin{example}
    \leavevmode
    \begin{enumerate}
        \item \(\mathbb{R}\) is locally compact. \(\forall x \in \mathbb{R}\), \(x \in (a, b) \subset [a, b]\), and \([a, b]\) is compact.
        \item \(\mathbb{R}^n\) is locally compact.
        \item \(\mathbb{Q} \subset \mathbb{R}\) i \textbf{not} locally compact.
        \item \(\mathbb{R}^\omega\) with product topology is \textbf{not} locally compact. If \(U\) is open, \(U\) contains a basis element \((a_1, b_1) \times \cdots \times (a_n, b_n) \times \mathbb{R} \times \mathbb{R} \times \cdots\). Then \(U\) is not contained in any compact set \(C\). (If \(U \subset C\), then \(\overline{U} \subset C \implies \overline{U}\) is compact. But \(\overline{U}\) contains factors of \(\mathbb{R}\), and hence not compact.)
    \end{enumerate}
\end{example}

\begin{note}
Compact Hausdorff spaces are ``nice''.
\end{note}

\subsection{One-Point Compactification}

\begin{theorem}[One-Point Compactification]
Let \(X\) be a \textbf{Hausdorff} space. Then \(X\) is locally compact
\(\iff \exists Y\) such that
\begin{enumerate}
    \item \(X \subset Y\).
    \item \(Y \setminus X = \{p\}\) is a single point.
    \item \(Y\) is a \textbf{compact Hausdorff} space.
\end{enumerate}
This \(Y\) is unique in the following sense: if \(Y, Y'\) are two such spaces, there exists a homeomorphism \(h: Y \to Y'\) fixing \(X\).
\end{theorem}

\begin{remark}
Proof is commented out!
\end{remark}

% \begin{proof}
% \textbf{Uniqueness}: Let \(Y = X \cup \{p\}, Y' = X \cup \{q\}\). Define \(h: Y \to Y'\) by \(h(x) = x\) for \(x \in X\), \(h(p) = q\). Show \(h\) is continuous. Take \(U \subset Y'\) open.
% \begin{itemize}
%     \item If \(U \subset X\), \(h^{-1}(U) = U\) is open in \(X \implies\) open in \(Y\).
%     \item If \(q \in U\), let \(C = Y' \setminus U\). \(C\) is closed in \(Y'\), \(C \subset X\).
%     Since \(Y'\) compact, \(C\) is compact.
%     \(h^{-1}(U) = Y \setminus C\).
%     By construction of topology, complements of compact sets \(\subset X\) are open neighborhoods of infinity.
% \end{itemize}

% \textbf{Construction}: Let \(Y = X \cup \{\infty\}\). Define topology on \(Y\):
% \begin{enumerate}
%     \item \(U \subset X\) open \(\implies U\) open in \(Y\).
%     \item \(Y \setminus C\) open in \(Y\) if \(C \subset X\) is compact.
% \end{enumerate}

% Need to prove this defines a topology and \(Y\) is compact Hausdorff. (Continued next time).
% \end{proof}

\newpage

\section{Notes 18 - 12.16}

\subsection{Locally Compact Spaces}

\begin{definition}[Locally Compact Space]
\leavevmode
\begin{itemize}
\item \(X\) is \textbf{locally compact} at \(x \in X\) if \(\exists U\) open, \(C\) compact such that \(x \in U \subset C\).
\item \(X\) is \textbf{locally compact} if it is locally compact at every \(x \in X\).
\end{itemize}
\end{definition}

\begin{recap}[One-Point Compactification]
Let \(X\) be a topological space. Then \(X\) is locally compact Hausdorff \(\iff \exists Y\) such that
\begin{enumerate}
\item \(X \subset Y\)
\item \(Y \setminus X = \{pt\}\)
\item \(Y\) is compact Hausdorff.
\end{enumerate}
\end{recap}

\begin{note}[Uniqueness]
If there are two \(Y, Y'\) satisfying (1)-(3), then \(\exists f: Y \xrightarrow{\sim} Y'\) such that \(f|_X = \text{Id}\).
(The map sends \(p \in Y \setminus X\) to \(p' \in Y' \setminus X\)).
\end{note}

\begin{remark}
Proof is commented out!
\end{remark}

% \begin{proof}[Proof (Continued)]
% \textbf{Step 2 (\(\Rightarrow\))}: The topology on \(Y\) is given by \(Y = X \cup \{\infty\}\):
% \begin{enumerate}
% \item \(U \subset X\) open. (Note: \(U \not\ni \infty\)).
% \item \(Y \setminus C\) is open, where \(C \subset X\) is compact. (Note: \(Y \setminus C \ni \infty\)).
% \end{enumerate}

% Check constraints:
% \begin{enumerate}
% \item \textbf{This is a topology} (Verified last time).
% \item \textbf{\(X \subset Y\) is a subspace}:
% \begin{itemize}
% \item[(a)] If \(U \subset Y, U \not\ni \infty\), then \(U\) is open in \(X\).
% \item[(b)] If \(U = Y \setminus C\) (so \(\infty \in U\)), then \(U \cap X = (Y \setminus C) \cap X = X \setminus C\). Since \(X\) is Hausdorff and \(C\) is compact, \(C\) is closed in \(X\). Thus \(X \setminus C\) is open in \(X\).
% \end{itemize}
% \item \textbf{\(Y\) is compact}:
% Let \(\mathcal{A}\) be an open covering of \(Y\). \(\mathcal{A}\) contains some set \(U_\infty = Y \setminus C\) containing \(\infty\).
% Take all sets in \(\mathcal{A}\) different from \(U_\infty\), and intersect them with \(X\) (these are open in \(X\)).
% This collection covers \(C\). Since \(C\) is compact in \(X\), \(\exists\) finite subcover \(\mathcal{A}'\) of \(C\).
% Then \(\mathcal{A}' \cup \{Y \setminus C\}\) is a finite subcover of \(Y\).
% \end{enumerate}

% \textbf{Step 3:} Show that \(Y\) is Hausdorff. Take \(x, y \in Y\).
% \begin{itemize}
% \item If \(x, y \neq \infty\), they are in \(X\). \(X\) is Hausdorff \(\implies \exists U \ni x, V \ni y\) disjoint open in \(X\). These are open in \(Y\).
% \item If \(y = \infty, x \in X\):
% Since \(X\) is locally compact at \(x\), \(\exists U\) open, \(C\) compact such that \(x \in U \subset C\).
% Let \(V = Y \setminus C\). Then \(V\) is open in \(Y\) and contains \(\infty\).
% \(U\) is open in \(Y\) and \(x \in U\).
% \(U \subset C \implies U \cap (Y \setminus C) = \varnothing \implies U \cap V = \varnothing\).
% \end{itemize}

% \textbf{Step 4 (\(\Leftarrow\))} Let \(Y\) satisfy (1)-(3). Let \(\infty \in Y\) be the point such that \(X = Y \setminus \{\infty\}\). We show \(X\) is locally compact Hausdorff:
% \begin{itemize}
% \item \textbf{Hausdorff}:
% Obvious (\(X\) is a subspace of Hausdorff \(Y\)).

% \item \textbf{Locally Compact}:
% For any point \(x \in X\):

% Since \(Y\) is Hausdorff, \(\exists U \ni x, V \ni \infty\) open in \(Y\) such that \(U \cap V = \varnothing\). Let \(C = Y \setminus V\). Then \(C\) is closed in \(Y \implies C\) is compact (subspace of compact). Since \(U \cap V = \varnothing\), \(U \subset Y \setminus V = C\). Also \(V \ni \infty \implies C \subset X\). So \(x \in U \subset C \subset X\), with \(U\) open in \(Y\) (hence in \(X\)) and \(C\) compact. Thus \(X\) is locally compact at \(x\).
% \end{itemize}
% \end{proof}

\begin{example}
\(Y\) is called the \textbf{one-point compactification} of \(X\).
Examples:
\begin{itemize}
\item \(\overline{\mathbb{R}} \cong S^1\)
\item \(\overline{\mathbb{R}^n} \cong S^n\)
\item \(\overline{\mathbb{C}} \cong \mathbb{C} \cup \{\infty\} \cong S^2\) (Riemann sphere).
Map: \(z \mapsto \frac{az+b}{cz+d}, \left(\begin{smallmatrix} a & b \\ c & d \end{smallmatrix}\right) \in \text{Mat}_2(\mathbb{C}), \det \neq 0\).
\end{itemize}
\end{example}

\subsection{Corollaries}

\begin{theorem}
Let \(X\) be a Hausdorff space. Then \(X\) is locally compact \(\iff \forall x \in X, \forall U \ni x\) open,
there exists \(V \ni x\) open, \(\overline{V} \subset U\), \(\overline{V}\) compact.
\end{theorem}

\begin{remark}
Proof is commented out!
\end{remark}

% \begin{proof}
% \begin{itemize}
% \item[\(\Leftarrow\)] \(x \in V \subset \overline{V} \implies X\) locally compact at \(x\).

% \item[\(\Rightarrow\)] Let \(Y \supset X\) be the one-point compactification.
% Let \(C = Y \setminus U\) (closed \(\implies\) compact).
% Then \(\exists V \ni x\) open, \(W \supset C\) open such that \(W \cap V = \varnothing\).
% \(\overline{V}\) is compact (closed in \(Y\)), \(\overline{V} \cap C = \varnothing \implies \overline{V} \subset U\).
% \end{itemize}
% \end{proof}

\begin{corollary}
Let \(X\) be locally compact Hausdorff.
Let \(A \subset X\) be open or closed. Then \(A\) is locally compact.
\end{corollary}

\begin{remark}
Proof is commented out!
\end{remark}

% \begin{proof}
% \begin{itemize}
% \item Let \(A \subset X\) be \textbf{closed}.

% Given \(x \in A\), since \(X\) is locally compact, \(\exists U\) open, \(C\) compact in \(X\) s.t. \(x \in U \subset C\). Then \(x \in (U \cap A) \subset (C \cap A) \subset A\). \(U \cap A\) is open in \(A\). \(C \cap A\) is closed in \(C\) (since \(A\) closed in \(X\)) \(\implies C \cap A\) is compact \(\implies A\) is locally compact.

% \item Let \(A \subset X\) be \textbf{open}.

% Let \(x \in A\). \(X\) locally compact, then \(\exists V\) open, \(K\) compact s.t. \(x \in V \subset K\). Using regularity of locally compact Hausdorff spaces: \(\exists x \in V \subset \overline{V} \subset A\) with \(\overline{V}\) compact. \(V\) is open in \(A\), \(\overline{V}\) is compact. Thus, \(A\) is locally compact at \(x\).
% \end{itemize}
% \end{proof}

\begin{corollary}
\(X\) is locally compact Hausdorff \(\iff X\) is homeomorphic to an open subspace of a compact Hausdorff space.

Prove as exercise.
\end{corollary}

\subsection{Urysohn's Metrization Theorem}

\begin{theorem}[Urysohn's Metrization Theorem]
Every \(X\) that is \textbf{second-countable} and \textbf{regular (T3)} is \textbf{metrizable} (sufficient, not necessary condition).
\end{theorem}

\begin{recap}[Countability Axioms]
\leavevmode
\begin{itemize}
\item \textbf{First-countable}: \(\forall x \in X, \exists \{B_n\}_{n \in \mathbb{N}} = \mathcal{B}_x\) such that every open \(U \ni x\) contains at least one \(B_n\).
\item \textbf{Second-countable}: \(\exists \{B_n\}_{n \in \mathbb{N}} = \mathcal{B}\) (countable basis) such that \(\forall x \in X, \forall U \ni x\) open, \(\exists n\) s.t. \(x \in B_n \subset U\).
\item \textbf{Example}: \(\mathbb{R}^n\) with \(B = \{ U_\varepsilon(x) \mid x \in \mathbb{Q}^n, \varepsilon \in \mathbb{Q}_{>0} \}\).
\end{itemize}
\end{recap}

\begin{xca}
Show that if \(X_n\) are 1st (resp. 2nd) countable, then \(\prod X_n\) is also 1st (resp. 2nd) countable.
\end{xca}

\begin{theorem}[Properties of 2nd Countable Spaces]
Let \(X\) be second-countable. Then
\begin{enumerate}
\item[(a)] Every open cover of \(X\) has a \textbf{countable subcover} (\(X\) is a \textbf{Lindelöf space}).
\item[(b)] There exists a countable subset \(A \subset X\) such that \(\overline{A} = X\) (\(X\) is \textbf{separable}).
\end{enumerate}
\end{theorem}

\begin{remark}
Proof is commented out!
\end{remark}

% \begin{proof}
% Let \(\mathcal{B} = \{B_n\}\) be a countable basis.
% \begin{enumerate}
% \item[(a)]
% Let \(\mathcal{A}\) be an open covering of \(X\). For each \(n\), if \(B_n\) is contained in some element of \(\mathcal{A}\), pick one such element \(A_n \in \mathcal{A}\) (Axiom of Choice). Let \(J = \{ n \mid \exists A \in \mathcal{A} \text{ s.t. } B_n \subset A \}\). Then \(\{A_n\}_{n \in J}\) is a countable subcollection.

% Claim: It covers \(X\).cLet \(x \in X\). Since \(\mathcal{A}\) covers \(X\), \(\exists A \in \mathcal{A}\) s.t. \(x \in A\). Since \(\mathcal{B}\) is a basis, \(\exists n\) s.t. \(x \in B_n \subset A\). Thus, \(n \in J\) and \(x \in B_n \subset A_n\). Therefore, \(x\) is covered.

% \item[(b)]
% For each \(B_n \neq \varnothing\), pick \(x_n \in B_n\). Let \(D = \{x_n \mid n \in \mathbb{N}\}\). This is countable.

% Claim: \(\overline{D} = X\). Take any \(x \in X\) and any open neighborhood \(U \ni x\). Basis condition \(\implies \exists B_k \subset U\) s.t. \(x \in B_k\). Then \(B_k \neq \varnothing \implies x_k \in B_k \subset U\). Thus, \(U \cap D \neq \varnothing\). Therefore, \(x \in \overline{D}\).
% \end{enumerate}
% \end{proof}

\newpage

% Week 10
\section{Notes 19 - 12.19}

\subsection{Countability Axioms Analysis}

\begin{theorem}
Let \(X\) be a second-countable space. Then:
\begin{enumerate}
\item Every covering of \(X\) by open sets has a countable subcovering (\(X\) is a \textbf{Lindelöf space}).
\item There exists a countable dense subset of \(X\): \(A \subset X\) such that \(\bar{A} = X\) (\(X\) is \textbf{separable}).
\end{enumerate}
\end{theorem}

\begin{remark}
The implications are as follows:
\begin{align*}
\text{2nd countable} &\implies \text{Lindelöf} \\
\text{2nd countable} &\implies \text{Separable}
\end{align*}
The converse does not hold.
    
A counterexample is the Sorgenfrey line \(\mathbb{R}_\ell\) (lower limit topology):
\begin{itemize}
\item \(\mathbb{R}_\ell\) is 1st-countable.
\item \(\mathbb{R}_\ell\) is Lindelöf and separable.
\item \(\mathbb{R}_\ell\) is \textbf{not} 2nd countable.
\end{itemize}
\end{remark}

\begin{example}[Analysis of \(\mathbb{R}_\ell\)]
\leavevmode
\begin{enumerate}
\item \textbf{First Countability:}
For \(x \in \mathbb{R}_\ell\), the collection \(\{[x, x + \frac{1}{n})\}_{n \in \mathbb{N}}\) is a countable local basis at \(x\). Thus, \(\mathbb{R}_\ell\) is first countable.

\item \textbf{Second Countability:}
\(\mathbb{R}_\ell\) is not second countable. If \(\mathcal{B}\) is a basis, then for any \(x \in \mathbb{R}\), there exists \(B_x \in \mathcal{B}\) such that \(x \in B_x \subseteq [x, x+1)\). This implies \(x = \min B_x\). Since all \(x\) are distinct, the map \(x \mapsto B_x\) is injective from the uncountable set \(\mathbb{R}\) to \(\mathcal{B}\). Thus \(\mathcal{B}\) must be uncountable.

\item \textbf{Separability:}
\(\mathbb{Q} \subset \mathbb{R}_\ell\) is dense. Thus \(\mathbb{R}_\ell\) is separable.
\end{enumerate}
\end{example}

\begin{proposition}
\(\mathbb{R}_\ell\) is a Lindelöf space.
\end{proposition}

\begin{remark}
Proof is commented out!
\end{remark}

% \begin{proof}
% Let \(\mathcal{A}\) be a covering of \(\mathbb{R}_\ell\) by basis elements \([a_\alpha, b_\alpha)\). We need to find a countable subset of \(\mathcal{A}\) covering \(\mathbb{R}_\ell\).

% Let \(C = \bigcup_\alpha (a_\alpha, b_\alpha)\). Note that \(C\) is open in \(\mathbb{R}\) with the standard topology.

% \textit{Claim:} \(\mathbb{R} \setminus C\) is countable.

% \textit{Proof of Claim:} Take \(x \in \mathbb{R} \setminus C\). Since \(x\) is covered by \(\mathcal{A}\), \(x \in [a_\beta, b_\beta)\) for some \(\beta\). Since \(x \notin C\) (and thus \(x \notin (a_\beta, b_\beta)\)), it must be that \(x = a_\beta\).
% Choose a rational number \(q_x \in (a_\beta, b_\beta) \cap \mathbb{Q}\). Then \(x < q_x\).
% If \(x, y \in \mathbb{R} \setminus C\) with \(x < y\), then \(q_x < q_y\). (Otherwise, if \(q_x > y\), then \(x < y < q_x\), implying \(y \in (a_\beta, b_\beta) \subset C\), a contradiction).
% The map \(x \mapsto q_x\) is an injection from \(\mathbb{R} \setminus C\) to \(\mathbb{Q}\). Hence \(\mathbb{R} \setminus C\) is countable.

% Since \(\mathbb{R} \setminus C\) is countable, we can choose a countable subcollection \(\mathcal{A}' \subset \mathcal{A}\) covering \(\mathbb{R} \setminus C\).
% Now, consider \(C\) as a subspace of \(\mathbb{R}_{std}\). \(C\) is covered by the collection \(\{(a_\alpha, b_\alpha)\}\), which are open in \(\mathbb{R}_{std}\). Since \(\mathbb{R}_{std}\) is second countable (and thus Lindelöf), there exists a countable subcover corresponding to \(\mathcal{A}'' \subset \mathcal{A}\).
% Then \(\mathcal{A}' \cup \mathcal{A}''\) is a countable subcover of \(\mathbb{R}_\ell\).
% \end{proof}

\begin{example}[Lindelöf Subspaces]
A subspace of a Lindelöf space is not necessarily Lindelöf.
Consider \(I_{ord}^2 = [0, 1] \times [0, 1]\) with the order topology given by the lexicographical order
\[(x, y) < (x', y') \iff x < x' \lor (x = x' \land y < y')\]

This space is compact (hence Lindelöf).
However, the subspace \([0, 1] \times (0, 1)\) is not Lindelöf. It can be covered by \(\{ \{x\} \times (0, 1) \}_{x \in [0, 1]}\), which has no countable subcover.
\end{example}

\begin{proposition}
\(\mathbb{R}_\ell^2 = \mathbb{R}_\ell \times \mathbb{R}_\ell\) is not Lindelöf.
\end{proposition}

\begin{note}
The line \(L = \{(x, -x) : x \in \mathbb{R}\}\) is closed and discrete in \(\mathbb{R}_\ell^2\), preventing a countable subcover for the complement cover strategy.
\end{note}

\subsection{Separation Axioms}

\begin{definition}[Separation Axioms]
A topological space \(X\) can satisfy the following separation axioms
\begin{enumerate}
\item[\textbf{(T1)}] \textbf{(Fréchet)}
Given \(x \neq y \in X\), there exists open \(U\) s.t. \(x \in U, y \notin U\).
        
Equivalent to: \(\{x\}\) is closed for every \(x \in X\).

\item[\textbf{(T2)}] \textbf{(Hausdorff)}
Given \(x \neq y \in X\), there exist open \(U, V\) s.t. \(x \in U, y \in V, U \cap V = \emptyset\).

\item[\textbf{(T3)}] \textbf{(Regular)}
\(X\) satisfies (T1) and for any \(x \in X\) and closed set \(B \not\ni x\), there exist open \(U, V\) s.t. \(x \in U, B \subset V, U \cap V = \emptyset\).

\item[\textbf{(T4)}] \textbf{(Normal)}
\(X\) satisfies (T1) and for any disjoint closed sets \(A, B\), there exist open \(U, V\) s.t. \(A \subset U, B \subset V, U \cap V = \emptyset\).
\end{enumerate}
\end{definition}

\begin{lemma}[Characterization of Regularity and Normality]
Let \(X\) be a (T1) topological space, then
\begin{enumerate}
\item \(X\) is regular iff \(\forall x \in X\) and open \(U \ni x\), there exists open \(V\) such that \(x \in V \subset \overline{V} \subset U\).
\item \(X\) is normal iff \(\forall\) closed \(A\) and open \(U \supset A\), there exists open \(V\) such that \(A \subset V \subset \overline{V} \subset U\).
\end{enumerate}
\end{lemma}

\begin{remark}
Proof is commented out!
\end{remark}

% \begin{proof}
% \textit{Proof of (a):}
% \begin{itemize}
% \item[\(\Rightarrow\)]
% Let \(X\) be regular and \(x \in U \subset X\) open. Let \(B = X \setminus U\), which is closed.

% Since \(x \notin B\), there exist disjoint open sets \(V, W\) such that \(x \in V\) and \(B \subset W\).

% Since \(V \cap W = \emptyset\), \(V \subset X \setminus W\). Since \(X \setminus W\) is closed, \(\overline{V} \subset X \setminus W \subset X \setminus B = U\).

% Thus, \(\overline{V} \subset U\).

% \item[\(\Leftarrow\)]
% Converse: Let \(x \in X, B\) be a closed set not containing \(x\). Let \(U = X \setminus B\) be open.

% By hypothesis, there exists open \(V\) s.t. \(x \in V \subset \overline{V} \subset U\).

% Let \(W = X \setminus \overline{V}\). Then \(W\) is open and contains \(B\) (since \(\overline{V} \subset U \implies X \setminus U \subset X \setminus \overline{V}\)).

% Also \(V \cap W = \emptyset\). Thus, \(X\) is regular.
% \end{itemize}

% \textit{Proof of (b) is similar.}
% \end{proof}

\begin{theorem}[Preservation Properties]
\leavevmode
\begin{enumerate}
\item \(X\) is Hausdorff \(\implies\) Every subspace \(Y \subset X\) is Hausdorff.
\item \(X_\alpha\) Hausdorff \(\implies \prod X_\alpha\) is Hausdorff.
\item \(X\) is Regular \(\implies\) Every subspace \(Y \subset X\) is Regular.
\item \(X_\alpha\) Regular \(\implies \prod X_\alpha\) is Regular.
\end{enumerate}
\end{theorem}

\begin{remark}
\leavevmode
\begin{enumerate}
\item If \(X\) is Normal, a subspace \(Y \subset X\) may \textbf{not} be normal.
\item If \(X_1, X_2\) are Normal, \(X_1 \times X_2\) may \textbf{not} be normal.
\end{enumerate}
\end{remark}

\begin{remark}
Proof is commented out!
\end{remark}

% \begin{proof}[Proof of (d)]
% Let \(X_\alpha\) be regular. This implies \(X_\alpha\) is Hausdorff, so \(\prod X_\alpha\) is Hausdorff.

% Thus, points are closed sets.

% To show regularity, use the Lemma. Let \(x = (x_\alpha) \in \prod X_\alpha\). Let \(U\) be a neighborhood of \(x\).

% There exists a basic open set \(\prod U_\alpha\) such that \(x \in \prod U_\alpha \subset U\), where \(U_\alpha = X_\alpha\) for all but finitely many \(\alpha\). For each \(\alpha\), since \(X_\alpha\) is regular, choose \(V_\alpha\) such that \(x_\alpha \in V_\alpha \subset \overline{V_\alpha} \subset U_\alpha\). (If \(U_\alpha = X_\alpha\), take \(V_\alpha = X_\alpha\)).

% Let \(V = \prod V_\alpha\). Then \(\overline{V} = \prod \overline{V_\alpha} \subset \prod U_\alpha \subset U\).
% Thus, \(\prod X_\alpha\) is regular.
% \end{proof}

\begin{example}[Normal Spaces]
\textbf{Show that \(\mathbb{R}_\ell\) is normal.}
    
\textit{Fact:} \(\mathbb{R}_\ell^2\) is not normal (but it is regular).
\end{example}

\begin{example}[Hausdorff but not Regular]
\(\mathbb{R}_K\) (K-topology).

Basis given by \((a, b)\) and \((a, b) \setminus K\), where \(K = \{1/n \mid n \geq 1\}\).

\(\mathbb{R}_K\) is Hausdorff but not regular. (\(K\) is closed, but cannot be separated from \(0\).)

Suppose \(U \ni 0\), \(V \supset K\), and \(U \cap V = \emptyset\).

\(U\) must contain some basis element \((a, b) \setminus K\) containing 0.

Choose \(n\) sufficient large such that \(1/n \in (a, b)\).

Since \(1/n \in K \subset V\), take a neighborhood of \(1/n\) inside \(V\), say \((c, d)\).

We can find points common to both neighborhoods leading to contradiction or limit point arguments showing separation is impossible.
\end{example}

\newpage

\section{Notes 20 - 12.22}

\subsection{Normality in Special Spaces}

\begin{theorem}
    Every second-countable regular space is normal.
\end{theorem}

\begin{remark}
Proof is commented out!
\end{remark}

% \begin{proof}
%     Let \(X\) be a second-countable regular space. Let \(A, B \subset X\) be disjoint closed sets.
%     Since \(X\) is regular, for every \(a \in A\), there exists an open neighborhood \(U_a\) of \(a\) such that \(\overline{U_a} \cap B = \emptyset\) (since \(X \setminus B\) is an open neighborhood of \(a\)).
    
%     Since \(X\) is second-countable, there exists a countable subcover of \(\{U_a\}_{a \in A}\). Let's denote the countable cover of \(A\) by \(\{U_n\}_{n=1}^\infty\).
%     Thus, \(A \subset \bigcup_{n=1}^\infty U_n\), and for each \(n\), \(\overline{U_n} \cap B = \emptyset\).
    
%     Similarly, we can find a countable family of open sets \(\{V_n\}_{n=1}^\infty\) covering \(B\) (that is, \(B \subset \bigcup_{n=1}^\infty V_n\)) such that for each \(n\), \(\overline{V_n} \cap A = \emptyset\).

%     Now we construct disjoint open sets \(U'\) and \(V'\) containing \(A\) and \(B\) respectively. Define
%     \[U'_n = U_n \setminus \bigcup_{i=1}^n \overline{V_i} \ \text{and} \ V'_n = V_n \setminus \bigcup_{i=1}^n \overline{U_i}\]

%     Note that \(U'_n\) and \(V'_n\) are open sets.
%     It holds that \(A \subset \bigcup U'_n\). Indeed, if \(a \in A\), then \(a \in U_n\) for some \(n\). Also, \(a \notin \overline{V_i}\) for any \(i\) (since \(\overline{V_i} \cap A = \emptyset\)). Thus \(a \in U'_n\).
%     Similarly, \(B \subset \bigcup V'_n\).

%     Let \(U' = \bigcup_{n=1}^\infty U'_n\) and \(V' = \bigcup_{n=1}^\infty V'_n\).
    
%     \textit{Claim: \(U' \cap V' = \emptyset\).}

%     Suppose for contradiction that \(x \in U' \cap V'\). Then \(x \in U'_j \cap V'_k\) for some indices \(j, k\).
%     Without loss of generality, assume \(j \le k\).
%     From the definition, \(x \in U'_j \implies x \in U_j\).
%     On the other hand, \(x \in V'_k = V_k \setminus \bigcup_{i=1}^k \overline{U_i}\). Since \(j \le k\), \(\overline{U_j}\) is one of the sets being subtracted. Thus \(x \notin \overline{U_j}\).
%     This contradicts \(x \in U_j \subset \overline{U_j}\).
%     Thus \(U' \cap V' = \emptyset\), so \(X\) is normal.
% \end{proof}

\begin{theorem}
    Every metrizable space \(X\) is normal.
\end{theorem}

\begin{remark}
Proof is commented out!
\end{remark}

% \begin{proof}
%     Let \(d\) be the metric on \(X\). Let \(A, B \subset X\) be disjoint closed sets.
%     For each \(a \in A\), since \(B\) is closed and \(a \notin B\), \(dist(a, B) > 0\). Thus there exists \(\varepsilon_a > 0\) such that \(B(a, \varepsilon_a) \cap B = \emptyset\).
%     Similarly, for each \(b \in B\), there exists \(\varepsilon_b > 0\) such that \(B(b, \varepsilon_b) \cap A = \emptyset\).

%     Define open sets
%     \[U = \bigcup_{a \in A} B(a, \frac{\varepsilon_a}{2}) \supset A, \ V = \bigcup_{b \in B} B(b, \frac{\varepsilon_b}{2}) \supset B\]

%     We claim \(U \cap V = \emptyset\).
%     Suppose not. Let \(z \in U \cap V\). Then \(z \in B(a, \frac{\varepsilon_a}{2})\) and \(z \in B(b, \frac{\varepsilon_b}{2})\) for some \(a \in A, b \in B\).
%     By triangle inequality
%     \[d(a, b) \le d(a, z) + d(z, b) < \frac{\varepsilon_a}{2} + \frac{\varepsilon_b}{2}\]

%     Without loss of generality, suppose \(\varepsilon_a \le \varepsilon_b\). Then
%     \[d(a, b) < \frac{\varepsilon_b}{2} + \frac{\varepsilon_b}{2} = \varepsilon_b\]

%     This implies \(a \in B(b, \varepsilon_b)\). But we chose \(\varepsilon_b\) such that \(B(b, \varepsilon_b) \cap A = \emptyset\), and \(a \in A\).
    
%     Contradiction! Thus, \(U \cap V = \emptyset\).
% \end{proof}

\begin{theorem}
    Every compact Hausdorff space is normal.
\end{theorem}

\begin{remark}
Proof is commented out!
\end{remark}

% \begin{proof}
%     Let \(X\) be a compact Hausdorff space.
%     First, recall that \(X\) is regular (a compact subset can be separated from a point in Hausdorff space).
    
%     Let \(A, B\) be disjoint closed subsets of \(X\). Since \(X\) is closed (in itself) and \(A\) is a closed subset of a compact space, \(A\) is compact.
    
%     For every \(a \in A\), since \(X\) is regular and \(B\) is closed (and \(a \notin B\)), there exist disjoint open sets \(U_a\) and \(V_a\) such that \(a \in U_a\) and \(B \subset V_a\).
    
%     The family \(\{U_a\}_{a \in A}\) forms an open cover of the compact set \(A\).
%     Thus, there exists a finite subcover \(\{U_{a_1}, ..., U_{a_n}\}\).
%     Let \(V_{a_1}, ..., V_{a_n}\) be the corresponding open sets containing \(B\).
    
%     Define
%     \[U = \bigcup_{i=1}^n U_{a_i} \ \text{and} \ V = \bigcap_{i=1}^n V_{a_i}\]

%     Then \(U\) is open and contains \(A\) (since it is the union of the subcover).
%     \(V\) is open (finite intersection of open sets) and contains \(B\) (since each \(V_{a_i}\) contains \(B\)).
    
%     Finally, check disjointness
%     \[U \cap V = \left(\bigcup_{i=1}^n U_{a_i}\right) \cap \left(\bigcap_{j=1}^n V_{a_j}\right) = \bigcup_{i=1}^n \left(U_{a_i} \cap \bigcap_{j=1}^n V_{a_j}\right)\]

%     Since \(\bigcap_{j=1}^n V_{a_j} \subset V_{a_i}\) and \(U_{a_i} \cap V_{a_i} = \emptyset\), each term in the union is empty.
%     Thus, \(U \cap V = \emptyset\).
% \end{proof}

\subsection{Urysohn's Lemma}

\begin{theorem}[Urysohn's Lemma]
    Given a normal space \(X\) and two disjoint closed sets \(A, B \subset X\), there exists a continuous function \(f: X \to [0, 1]\) such that
    \begin{align*}
        f(x) = 0 \ \text{for each } x \in A \\
        f(x) = 1 \ \text{for each } x \in B
    \end{align*}
\end{theorem}

\begin{remark}
Proof is commented out!
\end{remark}

% \begin{proof}
%     Let \(P = [0, 1] \cap \mathbb{Q}\). The proof proceeds in several steps.
    
%     \paragraph*{Step 1: Construction of Open Sets on \(P\)}
%     We want to define for each \(p \in P\) an open set \(U_p\) such that
%     \begin{equation}
%         A \subset U_p \subset \overline{U_p} \subset U_q \ \text{whenever } p < q \tag{\(*\)}
%     \end{equation}
%     and \(U_1 = X \setminus B\).
    
%     Enumerate the elements of \(P\) in a sequence \(\{r_n\}\) such that \(r_1 = 1\) and \(r_2 = 0\). (For instance: \(1, 0, 1/2, 1/3, 2/3, \cdots\)).
%     \begin{itemize}
%         \item Define \(U_{r_1} = U_1 = X \setminus B\). Since \(A \cap B = \emptyset\), \(A \subset U_1\).
%         \item Define \(U_{r_2} = U_0\). Since \(X\) is normal, \(A\) is closed and \(U_1\) is open with \(A \subset U_1\), there exists open \(U_0\) such that \(A \subset U_0 \subset \overline{U_0} \subset U_1\).
%     \end{itemize}
    
%     Let \(P_n = \{r_1, \cdots, r_n\}\). Suppose we have defined \(U_p\) for all \(p \in P_n\) satisfying condition \((*)\).
%     Let \(r = r_{n+1}\). Let \(p\) be the largest number in \(P_n\) smaller than \(r\), and \(q\) be the smallest number in \(P_n\) larger than \(r\). So \(p < r < q\) are immediate predecessor and successor in \(P_n\).
%     By hypothesis, \(\overline{U_p} \subset U_q\). Since \(\overline{U_p}\) is closed and \(U_q\) is open, by normality, there exists open \(U_r\) such that
%     \[\overline{U_p} \subset U_r \subset \overline{U_r} \subset U_q\]

%     This maintains the condition \((*)\) for \(P_{n+1}\). By induction, \(U_p\) is defined for all \(p \in P\).
    
%     \paragraph*{Step 2: Extension to Rationals}
%     Extend the definition to all \(p \in \mathbb{Q}\):
%     \begin{itemize}
%         \item If \(p < 0\), let \(U_p = \emptyset\).
%         \item If \(p > 1\), let \(U_p = X\).
%     \end{itemize}

%     Condition \((*)\) \(\overline{U_p} \subset U_q\) for \(p < q\) still holds.
    
%     \paragraph*{Step 3: Definition of the Function}
%     For \(x \in X\), define \(Q(x) = \{p \in \mathbb{Q} \mid x \in U_p\}\).
%     Since \(U_p = X\) for \(p > 1\), \(Q(x)\) is non-empty. Since \(U_p = \emptyset\) for \(p < 0\), \(Q(x)\) is bounded from below by 0. Define
%     \[f(x) = \inf Q(x)\]

%     Clearly \(f(x) \in [0, 1]\).
%     \begin{itemize}
%         \item If \(x \in A\), then \(x \in U_0 \subset U_p\) for all \(p \ge 0\). Thus \(0 \in Q(x)\) and \(f(x) \le 0 \implies f(x) = 0\).
%         \item If \(x \in B\), then \(x \notin U_1\). So \(x \notin U_p\) for any \(p \le 1\). \(Q(x)\) contains only rationals \(> 1\). Thus \(f(x) = 1\).
%     \end{itemize}
    
%     \paragraph*{Step 4: Continuity}
%     We verified two properties (proof omitted but standard):
%     \begin{enumerate}
%         \item[(i)] \(x \in U_r \implies f(x) \le r\)
%         \item[(ii)] \(x \notin U_r \implies f(x) \ge r\)
%     \end{enumerate}
    
%     To show continuity at \(x_0 \in X\): Let \((c, d)\) be an open interval containing \(f(x_0)\).
%     Choose \(p, q \in \mathbb{Q}\) such that \(c < p < f(x_0) < q < d\).
%     Let \(U = U_q \setminus \overline{U_p}\). This is an open set.
    
%     Check \(x_0 \in U\):
%     \begin{itemize}
%         \item \(f(x_0) < q \implies x_0 \in U_q\) (by contrapositive of (ii): if \(x \notin U_q \implies f(x) \ge q\)).
%         \item \(f(x_0) > p \implies x_0 \notin \overline{U_p}\) (Wait, actually: if \(x_0 \in \overline{U_p} \subset U_{r}\) for any \(p<r<f(x_0)\), then \(f(x_0) \le r\), contradiction).
%     \end{itemize}

%     More formally: Since \(p < f(x_0)\), there exists \(r \in \mathbb{Q}\) such that \(p < r < f(x_0)\). Then \(x_0 \notin U_r\) (by (i) contrapositive). Since \(\overline{U_p} \subset U_r\), \(x \notin \overline{U_p}\).
%     Thus \(x_0 \in U\).
    
%     For any \(z \in U\):
%     \begin{itemize}
%         \item \(z \in U_q \implies f(z) \le q\).
%         \item \(z \notin \overline{U_p} \implies z \notin U_p \implies f(z) \ge p\).
%     \end{itemize}
%     So \(f(z) \in [p, q] \subset (c, d)\).
%     Thus, \(f(U) \subset (c, d)\), so \(f\) is continuous.
% \end{proof}

\newpage

% Week 11
\section{Notes 21 - 12.23}

\subsection{Urysohn's Lemma (Recap)}

\begin{recap}[Urysohn's Lemma]
    Let \(X\) be a normal space, \(A, B \subset X\) closed, \(A \cap B = \emptyset\). Then there is a continuous function \(f: X \to [0, 1]\) such that
    \[f(x) = 0 \ \forall x \in A, \ f(x) = 1 \ \forall x \in B.\]

    (\(f\) \textit{separates} \(A\) and \(B\)).
\end{recap}

\begin{definition}[Completely Regular]
    If \(X\) satisfies \((T1)\) (points are closed) and any closed set \(B\) and point \(x_0 \notin B\) can be separated by a continuous function (that is, \(\exists f: X \to [0, 1]\) such that \(f(x_0) = 0, f(B) = 1\)), then \(X\) is \textbf{completely regular} (CR) or \((T_{3\frac{1}{2}})\).
\end{definition}

\begin{remark}
    Urysohn's Lemma implies \((T4) \implies (T_{3\frac{1}{2}})\).
    \begin{itemize}
        \item \((T3)\) (Regularity): Separate point \(x\) and closed \(A\) by disjoint \textit{open sets}.
        \item \((T_{3\frac{1}{2}})\) (Complete Regularity): Separate point \(x\) and closed \(A\) by a \textit{continuous function}.
        \item \((T4)\) (Normality): Separate closed \(A\) and closed \(B\) by disjoint \textit{open sets} (which implies separation by function via Urysohn).
    \end{itemize}
\end{remark}

\begin{proposition}
    \(X\) is completely regular \(\implies\) \(X\) is regular.
\end{proposition}

\begin{remark}
Proof is commented out!
\end{remark}

% \begin{proof}
%     Take \(x_0 \in X\), \(A \subset X\) closed, \(x_0 \notin A\).

%     Take \(f\) separating \(\{x_0\}\) and \(A\), that is, \(f: X \to [0, 1]\) with \(f(x_0) = 0, f(A) = 1\).

%     Let \(U = f^{-1}([0, \frac{1}{2})) \subset X\) and \(V = f^{-1}((\frac{1}{2}, 1]) \subset X\), both open.
    
%     Then \(x_0 \in U\) and \(A \subset V\).
%     Also \(U \cap V = \emptyset\), so \(X\) is regular.
% \end{proof}

\begin{theorem}
    \begin{enumerate}
        \item[(a)] Subspace of a CR space is CR.
        \item[(b)] Product of CR spaces is CR.
    \end{enumerate}
\end{theorem}

\begin{remark}
Proof is commented out!
\end{remark}

% \begin{proof}
%     \begin{enumerate}[(a)]
%         \item
%         Let \(X\) be CR, \(X \supset Y \ni x_0\). Let \(A \subset Y\) be closed in \(Y\) (with \(x_0 \notin A\)).

%         Let \(\overline{A}\) be the closure of \(A\) in \(X\).
%         Then we can find \(f: X \to [0, 1]\) continuous on \(X\) such that \(f(x_0) = 0\) and \(f(\overline{A}) = 1\).
%         Thus, \(f|_Y\) is the required function.

%         \item
%         Take \(A \subset \prod_{\alpha \in J} X_\alpha\) closed, \(b \notin A\), \(b = (b_\alpha)_{\alpha \in J}\).
%         \(X_\alpha\) CR \(\implies\) regular; \(\prod X_\alpha\) is regular.

%         Take \(U = \prod U_\alpha \ni b\) (open basis element) such that \(U \cap A = \emptyset\).
%         \(U_\alpha = X_\alpha\) if \(\alpha \neq \alpha_1, \cdots, \alpha_n\).
        
%         For each \(i=1, \cdots, n\), take \(f_i: X_{\alpha_i} \to [0, 1]\) such that \(f_i(b_{\alpha_i}) = 1\) and \(f_i(X_{\alpha_i} \setminus U_{\alpha_i}) = 0\).
        
%         Let \(\varphi_i(x) = f_i(\pi_{\alpha_i}(x))\) for \(x \in \prod X_\alpha\).
%         Define \(f = \varphi_1(x) \cdots \varphi_n(x) = \prod_{i=1}^n f_i(\pi_{\alpha_i}(x))\).
        
%         Then \(f(x) = 0 \ \forall x \in X \setminus U\).
%         So \(f(A) \equiv 0\), \(f(b) = \prod \varphi_i(b) = \prod f_i(b_{\alpha_i}) = 1\).
        
%         This \(f\) separates \(A\) and \(\{b\}\).
%     \end{enumerate}
% \end{proof}

\begin{remark}
    \begin{itemize}
        \item
        \((T_{3\frac{1}{2}}) \not\implies (T4)\). 
        Example: \(\mathbb{R}_l^2\) (Sorgenfrey plane).
        
        Reason: \(\mathbb{R}_l\) is normal \(\implies\) CR. Product of CR spaces is CR \(\implies \mathbb{R}_l^2\) is CR.
        
        But \(\mathbb{R}_l^2\) is \textbf{not} normal.
        
        \item
        Fact: There exist regular spaces that are not completely regular. (No proof given.)
    \end{itemize}
\end{remark}

\begin{proof}[Proof of Urysohn's Lemma for metric spaces]
    Let \((X, d)\) be a metric space and \(A, B \subset X\) be closed, \(A \cap B = \emptyset\).
    Define \(d_A(x) = \text{dist}(x, A) = \inf_{a \in A} d(x, a)\), which is continuous and
    \[d_A(x) = 0 \iff x \in \overline{A} = A.\]

    Similarly define \(d_B(x) = \text{dist}(x, B)\) and \(f(x) = \frac{d_A(x)}{d_A(x) + d_B(x)}\) (continuous).
    
    Since \(A \cap B = \emptyset\), for any \(x\), \(d_A(x)\) and \(d_B(x)\) cannot be both 0, so the denominator is non-zero.
    
    If \(x \in A\), \(d_A(x) = 0 \implies f(x) = 0\); If \(x \in B\), \(d_B(x) = 0 \implies f(x) = \frac{d_A(x)}{d_A(x)} = 1\).
\end{proof}

\subsection{Urysohn's Metrization Theorem}
\begin{theorem}[Urysohn's Metrization Theorem]
    Every regular space \(X\) with a countable basis is metrizable.
\end{theorem}

\begin{remark}
Proof is commented out!
\end{remark}

% \begin{proof}
%     Idea: Embed \(X\) into \(Y = [0, 1]^\omega \subset \mathbb{R}^\omega\).
%     \([0, 1]^\omega\) is called \textbf{Tychonoff cube}.

%     This can be \(\mathbb{R}^\omega\) with product topology (\textbf{coarser}) or \(\mathbb{R}^\omega\) with uniform metric topology (\textbf{finer}) defined by \(d(x, y) = \sup \bar{d}(x_i, y_i)\), where \(\bar{d}(a, b) = \min(|a-b|, 1)\).
    
%     \paragraph*{Step 1: Construct a family of functions}
%     Let \(\{B_n\}\) be a countable basis for \(X\).
%     Consider pairs \((m, n)\) such that \(\overline{B_n} \subset B_m\).
    
%     By regularity (and thus normality for second-countable spaces), there exists a continuous function \(g_{m, n}: X \to [0, 1]\) such that \(g_{m, n} = 1\) on \(\overline{B_n}\) and \(0\) outside \(B_m\).
    
%     This gives a countable family of functions \(\{f_n\}_{n \in \mathbb{N}}\) (reindexing the \(g_{m, n}\) via \(\mathbb{N} \times \mathbb{N} \to \mathbb{N}\)).
    
%     For any \(x_0 \in X\) and open \(U \ni x_0\), there exists \(B_m\) such that \(x_0 \in B_m \subset U\).
%     By regularity, there exists \(B_n\) such that \(x_0 \in B_n \subset \overline{B_n} \subset B_m\).
%     Then the function \(g_{m, n}\) satisfies \(g_{m, n}(x_0) = 1\) and vanishes outside \(U\).

%     \paragraph*{Step 2: Define the embedding}
%     Define \(F: X \to [0, 1]^\omega\) by \(F(x) = (f_1(x), f_2(x), \cdots)\).
%     \begin{enumerate}
%         \item[(a)] \(F\) is continuous because each component \(f_n\) is continuous (product topology).
%         \item[(b)] \(F\) is injective. If \(x \neq y\), since \(X\) is Hausdorff, there exists open \(U \ni x\) such that \(y \notin U\). By the separation property from Step 1, there exists some \(f_k\) such that \(f_k(x) > 0\) and \(f_k(y) = 0\). Thus \(F(x) \neq F(y)\).
%     \end{enumerate}
    
%     \paragraph*{Step 3: \(F\) is an embedding}
%     Let \(Z = F(X)\). \(F: X \to Z\) is a continuous bijection. We usually need to show \(F^{-1}\) is continuous, that is, \(F\) maps open sets to open sets in \(Z\).
    
%     Let \(U \subset X\) be open. Let \(z_0 \in F(U)\). Let \(x_0 = F^{-1}(z_0) \in U\).
%     We need to find an open set \(W \subset Z\) such that \(z_0 \in W \subset F(U)\).
    
%     Choose index \(N\) such that the corresponding function \(f_N\) separates \(x_0\) and \(X \setminus U\) (specifically, \(f_N(x_0) > 0\) and \(f_N(X \setminus U) = 0\)).
    
%     Let \(V = \pi_N^{-1}((0, \infty)) \cap Z = \{ (z_1, z_2, \cdots) \in Z \mid z_N > 0 \}\).
%     This is open in \(Z\).
%     Clearly \(z_0 \in V\) because \(\pi_N(z_0) = f_N(x_0) > 0\).
    
%     If \(z \in V\), let \(x = F^{-1}(z)\). Then \(f_N(x) = \pi_N(z) > 0\).
%     Since \(f_N\) vanishes on \(X \setminus U\), it must be that \(x \in U\).
%     Thus \(z = F(x) \in F(U)\).
%     So \(V \subset F(U)\), proving \(F(U)\) is open in \(Z\).
    
%     \paragraph*{Conclusion}
%     \(X\) is homeomorphic to \(F(X) \subset [0, 1]^\omega\). Since \([0, 1]^\omega\) is metrizable, \(X\) is metrizable.
% \end{proof}

\newpage

\section{Notes 22 - 12.26}

\subsection{The Tychonoff Theorem}

\begin{theorem}[Tychonoff Theorem]
    If \(\{X_\alpha\}_{\alpha \in J}\) is a family of compact spaces, then their product \(\prod_{\alpha \in J} X_\alpha\) is compact (in the product topology).
\end{theorem}

\begin{remark}
    \(X, Y\) compact \(\implies X \times Y\) compact. This is easy to prove, but the infinite case is hard to generalize using open covers directly. We use the closed set formulation.
\end{remark}

\subsection{Closed Set Formulation of Compactness}

Let \(X\) be a topological space, \(\mathcal{C} \subset 2^X\) a collection of subsets of \(X\).

\begin{definition}[Finite Intersection Property]
    A collection \(\mathcal{C}\) has the \textbf{finite intersection property (f.i.p.)} if for every finite subcollection \(\{C_1, \cdots, C_n\} \subset \mathcal{C}\), the intersection is non-empty:
    \[C_1 \cap \cdots \cap C_n \neq \emptyset\]
\end{definition}

\begin{theorem}
    \(X\) is compact iff for every collection of closed sets \(\mathcal{C}\) of \(X\) with the f.i.p., we have
    \[\bigcap_{C \in \mathcal{C}} C \neq \emptyset\]
    (that is, all sets in \(\mathcal{C}\) have a point in common).
\end{theorem}

\begin{example}
    Take \(X = (0, 1)\), \(C_n = (0, \frac{1}{n}]\).
    Intersection of any finite number of sets:
    \[C_{n_1} \cap \cdots \cap C_{n_m} = C_{\max(n_1, \cdots, n_m)} \neq \emptyset.\]
    
    However, \(\bigcap_{n \in \mathbb{N}} C_n = \bigcap_{n \in \mathbb{N}} (0, \frac{1}{n}] = \emptyset\).
    Thus, \((0, 1)\) is not compact.
\end{example}

\subsection{Proof Preparation}

\paragraph*{False Attempt Strategy}
Take \(X_1, X_2\) compact. Let \(\mathcal{A}\) be a collection of closed subsets of \(X_1 \times X_2\) with f.i.p.

Consider projections \(\{\pi_1(A) \mid A \in \mathcal{A}\}\). This has f.i.p. in \(X_1\).
Since \(X_1\) is compact, \(\bigcap \overline{\pi_1(A)} \neq \emptyset\). So \(\exists x_1 \in \bigcap \overline{\pi_1(A)}\).
Similarly, \(\exists x_2 \in \bigcap \overline{\pi_2(A)}\).
Then need to prove \((x_1, x_2) \in \bigcap_{A \in \mathcal{A}} A\).

\textbf{This is FALSE!}

Counterexample: \(X = [0, 1] \times [0, 1]\). Consider ellipses or sets concentrating around different points but projecting to the whole interval. (Sketch in notes shows non-trivial interaction.)

\paragraph*{Correct Strategy (Maximal F.I.P. Collections)}
Idea: Given a collection of subsets (closed sets) \(\mathcal{A}\) with f.i.p., extend it to a \textbf{maximal possible} collection of subsets \(\mathcal{D} \supset \mathcal{A}\) with f.i.p.

We fix the problem with closures later: take \textit{any} collection of subsets \(\mathcal{A}\) such that \(\{\overline{A}\}\) have f.i.p., look for \(x \in \bigcap \overline{A}\).

\begin{remark}[Zorn's Lemma]
    Let \(A\) be a strictly partially ordered set (poset) such that every linearly ordered subset (chain) has an upper bound in \(A\). Then \(A\) has a maximal element.
\end{remark}

\begin{lemma}[Lemma 1]
    Let \(X\) be a set, \(\mathcal{A}\) a collection of subsets of \(X\) with f.i.p.
    Then there exists a collection \(\mathcal{D} \supset \mathcal{A}\) such that
    \begin{enumerate}
        \item \(\mathcal{D}\) has f.i.p.
        \item \(\mathcal{D}\) is maximal (not contained in any other collection with f.i.p.).
    \end{enumerate}
\end{lemma}

\begin{remark}
Proof is commented out!
\end{remark}

% \begin{proof}
%     Let \(\mathbb{A} = \{ \mathcal{B} \supset \mathcal{A} \mid \mathcal{B} \text{ has f.i.p.} \}\) (a collection with f.i.p.). Order \(\mathbb{A}\) by inclusion \(\subset\).

%     Need to show every chain has an upper bound.
    
%     Let \(\mathbb{B} \subset \mathbb{A}\) be a simply ordered (chain) subset.
%     Define \(\mathcal{C} = \bigcup_{\mathcal{B} \in \mathbb{B}} \mathcal{B}\) (candidate for upper bound).
%     Clearly \(\mathcal{B} \supset \mathcal{A}\) for all \(\mathcal{B} \in \mathbb{B}\), so \(\mathcal{C} \supset \mathcal{A}\).
    
%     Does \(\mathcal{C}\) have f.i.p.?
%     Take \(C_1, \cdots, C_n \in \mathcal{C}\).
%     Then \(C_i \in \mathcal{B}_i\) for some \(\mathcal{B}_i \in \mathbb{B}\).
%     Since \(\mathbb{B}\) is a chain, there exists \(k\) such that \(\mathcal{B}_k \supseteq \mathcal{B}_i\) for all \(i\).
%     Then \(C_1, \cdots, C_n \in \mathcal{B}_k\).
%     Since \(\mathcal{B}_k\) has f.i.p., \(C_1 \cap \cdots \cap C_n \neq \emptyset\).
    
%     So \(\mathcal{C} \in \mathbb{A}\) and is an upper bound.
%     By Zorn's Lemma, there exists a maximal element \(\mathcal{D}\).
% \end{proof}

\begin{lemma}[Lemma 2]
    Let \(X\) be a set, \(\mathcal{D}\) a maximal collection of subsets of \(X\) with f.i.p.
    \begin{enumerate}
        \item[(a)] \(\mathcal{D}\) is closed under taking finite intersections: if \(D_1, \cdots, D_n \in \mathcal{D}\), then \(D_1 \cap \cdots \cap D_n \in \mathcal{D}\).
        \item[(b)] If \(A \subset X\) intersects every \(D \in \mathcal{D}\), then \(A \in \mathcal{D}\).
    \end{enumerate}
\end{lemma}

\begin{remark}
Proof is commented out!
\end{remark}

% \begin{proof}
% \begin{enumerate}[(a)]
%     \item
%     Take \(D_1, \cdots, D_n \in \mathcal{D}\). Let \(B = D_1 \cap \cdots \cap D_n\). If \(B \notin \mathcal{D}\), then \(\mathcal{E} = \mathcal{D} \cup \{B\} \supsetneq \mathcal{D}\).

%     Then \(\mathcal{E}\) has f.i.p.
%     Check: Take \(D'_1, \cdots, D'_m, B \in \mathcal{E}\).

%     Then \(D'_1 \cap \cdots \cap D'_m \cap B = D'_1 \cap \cdots \cap D'_m \cap D_1 \cap \cdots \cap D_n\).

%     This is a finite intersection of elements in \(\mathcal{D}\), so it is not empty \(\neq \emptyset\).

%     Contradiction to maximality. So \(B \in \mathcal{D}\).

%     \item
%     Given such an \(A\), take \(\mathcal{E} = \mathcal{D} \cup \{A\}\).
%     Then \(\mathcal{E}\) has f.i.p.: need to show \(D_1 \cap \cdots \cap D_n \cap A \neq \emptyset\).

%     By assumption, \(D_1 \cap \cdots \cap D_n \cap A = D \cap A \neq \emptyset\), where \(D = D_1 \cap \cdots \cap D_n \in \mathcal{D}\) by (a).
    
%     Since \(\mathcal{D}\) is maximal, \(\mathcal{D} = \mathcal{E}\), so \(A \in \mathcal{D}\).
% \end{enumerate}
% \end{proof}

\subsection{Proof of Tychonoff's Theorem}

\begin{remark}
Proof is commented out!
\end{remark}

% \begin{proof}
%     Let \(X = \prod X_\alpha\), \(X_\alpha\) compact. Take \(\mathcal{A}\) coll. of subsets of \(X\) with f.i.p.

%     Need to prove \(\bigcap_{A \in \mathcal{A}} \overline{A} \neq \emptyset\).
    
%     By Lemma 1, take \(\mathcal{D} \supset \mathcal{A}\) with f.i.p. maximal.
%     \(\bigcap_{D \in \mathcal{D}} \overline{D} \neq \emptyset\).
    
%     Take \(X_\alpha, \alpha \in J, \pi_\alpha: X \to X_\alpha\).
%     \(\{\pi_\alpha(D) \mid D \in \mathcal{D}\}\) is a coll. of subsets of \(X_\alpha\) with f.i.p.

%     Since \(X_\alpha\) compact, \(\exists x_\alpha \in \bigcap_{D \in \mathcal{D}} \overline{\pi_\alpha(D)}\).
    
%     Now prove \(x = (x_\alpha)_{\alpha \in J} \in \overline{D}\) for every \(D \in \mathcal{D}\).
    
%     \(X\) has \textit{subbasis} of topology given by \(\prod U_\alpha = \pi_\beta^{-1}(U_\beta)\), where
%     \[\begin{cases}
%         U_\alpha = X_\alpha & \text{if } \alpha \neq \beta, \text{ for } \beta \in J \text{ fixed}, \\
%         U_\beta \subset X_\beta & \text{open}.
%     \end{cases}\]

%     Show that if \(\pi_\beta^{-1}(U_\beta)\) is any subbasis element, \(x \in \pi_\beta^{-1}(U_\beta)\), then \(\pi_\beta^{-1}(U_\beta)\) intersects every element of \(\mathcal{D}\).
    
%     \(x_\beta \in U_\beta\), \(x_\beta \in \overline{\pi_\beta(D)}\), \(\forall D \in \mathcal{D}\).
%     So \(y_\beta = \pi_\beta(y) \in U_\beta \cap \pi_\beta(D)\), \(y \in \pi_\beta^{-1}(U_\beta) \cap D\).
    
%     Then \(\pi_\beta^{-1}(U_\beta) \in \mathcal{D}\) (by Lemma 2(b)).
    
%     Every basis element cont. \(x\) is a finite intersection
%     \[U = \pi_{\beta_1}^{-1}(U_{\beta_1}) \cap \cdots \cap \pi_{\beta_n}^{-1}(U_{\beta_n}) \in \mathcal{D} \ (\text{by Lemma 2(a)}).\]
    
%     Then every basis element \(U\) cont. \(x\) intersects \(D \implies x \in \overline{D} \implies \bigcap \overline{D} \ni x\).
% \end{proof}

\newpage

% Week 12
\section{Notes 23 - 12.29}

\subsection{Stone-\v{C}ech Compactification (1937)}

\begin{definition}[Compactification]
    A \textbf{compactification} of a Hausdorff space \(X\) is a Hausdorff space \(Y \supset X\) such that \(Y = \overline{X}\).
    We say that compactifications \(Y_1\) and \(Y_2\) are \textbf{equivalent} if there exists a homeomorphism \(h: Y_1 \to Y_2\) such that \(h(x) = x\) for all \(x \in X\).
\end{definition}

\begin{example}
    \leavevmode
    \begin{enumerate}
        \item \(X = (0, 1)\), \(Y = S^1 \subset \mathbb{R}^2\). \(x \mapsto (\cos 2\pi x, \sin 2\pi x)\). (One-point compactification).
        \item \(X = (0, 1)\), \(Y = [0, 1]\). (Standard compactification).
        \item \(X = (0, 1)\), \(X \hookrightarrow Y \subset \mathbb{R}^2\) via \(x \mapsto (x, \sin \frac{1}{x})\).
        Here \(Y = \overline{X_0}\) where \(X_0 = \{(x, \sin \frac{1}{x}) \mid x \in (0, 1) \}\).
        What is \(Y_0 \setminus X_0 = A\)? \(A = \{0\} \times [-1, 1] \cup \{(1, \sin 1)\}\).
    \end{enumerate}
\end{example}

\subsection{Complete Regularity and Embedding}

\begin{proposition}
    If \(X \subset Y\) and \(Y\) is compact, then \(X\) is completely regular.
\end{proposition}

\begin{recap}[Completely Regular]
    \(X\) is completely regular if for every closed set \(Z \subset X\) and point \(y \in X \setminus Z\) (or \(y \in U \subset X\)), there exists a continuous function \(f: X \to \mathbb{R}\) such that \(f(y) > 0\) and \(f(Z) = 0\) (or \(f(z) = 0 \forall z \in X \setminus U\)). (Property \(T_{3\frac{1}{2}}\)).
\end{recap}

\begin{remark}
\begin{enumerate}
    \item \(X\) is completely regular as a subspace of \(Y\).
    \item If \(X\) is completely regular, it has a compactification.
    \item Urysohn Metrization Theorem: \(X \hookrightarrow [0, 1]^\omega \subset \mathbb{R}^\omega\) (regular, countable basis).
\end{enumerate}
\end{remark}

\begin{theorem}[Embedding Theorem]
    Let \(X\) be \(T_1\). Suppose \(X\) has an indexed family of continuous functions \(\{f_\alpha\}_{\alpha \in J}, f_\alpha: X \to \mathbb{R}\), such that for all \(x_0 \in X\) and open \(U \ni x_0\), there exists \(f_\alpha\) such that \(f_\alpha(x_0) > 0\) and \(f_\alpha(X \setminus U) = 0\).
    Then \(F: X \to \mathbb{R}^J\) defined by \(x \mapsto \{f_\alpha(x)\}_{\alpha \in J}\) is an embedding \(X \hookrightarrow \mathbb{R}^J\).
    If additionally \(f_\alpha: X \to [0, 1]\), then \(F: X \hookrightarrow [0, 1]^J\).
\end{theorem}

\begin{remark}
Proof is commented out!
\end{remark}

% \begin{proof}
%     \textbf{\(F\) is injective:}
%     \(\forall x \neq y \implies F(x) \neq F(y)\).
%     Take \(f_\alpha\) separating \(\{x\}\) and \(\{y\}\).
    
%     We have \(f_\alpha(x) > 0, f_\alpha(y) = 0 \implies f_\alpha(x) \neq f_\alpha(y) \implies F(x) \neq F(y)\).

%     \textbf{\(F\) is continuous} since all \(f_\alpha\) are continuous.

%     Need to show: \(F\) takes open sets to open sets.

%     But \(f_\alpha: X \setminus U \to \{0\}\), so for closed sets this is true.
% \end{proof}

\begin{theorem}
    \(X\) is completely regular iff it is homeomorphic to a subspace of \([0, 1]^J\) for some \(J\).
\end{theorem}

\begin{remark}
Proof is commented out!
\end{remark}

% \begin{proof}
%     \(h: X \hookrightarrow [0, 1]^J\) (\([0, 1]^J\) is compact by Tychonoff's Thm).

%     \(\overline{h(X)}\) is closed, subspace of compact \(\implies\) compact.
% \end{proof}

\begin{lemma}
    Let \(X\) be a Hausdorff space, \(h: X \to Z\) an embedding with \(Z\) compact Hausdorff.
    Then there exists a compactification \(Y \supset X\) s.t. there is an embedding \(H: Y \to Z\) with \(H(x) = h(x) \forall x \in X\).
\end{lemma}

\begin{remark}
Proof is commented out!
\end{remark}

% \begin{proof}
%     \(X_0 = h(X)\), \(Y_0 = \overline{h(X)}\) in \(Z\). So \(\overline{X_0} = Y_0\).

%     Want: \(Y \simeq Y_0\). Construct it as follows:

%     Take \(k: A \xrightarrow{1:1} Y \setminus X\).
%     \(Y = X \cup A\), \(H(x) = h(x) \forall x \in X\), \(H(a) = k(a) \forall a \in A\).
    
%     \(U\) is open in \(Y \iff H(U)\) open in \(Z\).

%     Since \(H|_X = h\), then \(X\) is a subspace of \(Y\).
%     \(H: Y \hookrightarrow Z\) embedding.

%     Let \(H_i: Y_i \to Z\), \(H_i: X \to h(X) = X_0\), \(Y_i \to \overline{X_0} \subset Z\), where \(i = 1, 2\) and \(Y_i\) compact.

%     Since \(H_i(Y_i) \supset X_0\), \(H_i(Y_i)\) is closed (compact implies closed in Hausdorff).
    
%     \(H_i(Y_i) = \overline{X_0}\), \(H_i: Y_i \xrightarrow{\sim} \overline{X_0}\), so \(H_2^{-1} \circ H_1: Y_1 \xrightarrow{\sim} Y_2\) is a homeomorphism.
% \end{proof}

\subsection{Examples of Extension Problems}

\begin{xca}
Given a function \(f\) on \(X\), can we extend it to a function on \(Y = \overline{X}\)? (Assuming \(f\) is \textbf{BOUNDED}.)
\end{xca}

\begin{example}
    \leavevmode
    \begin{enumerate}
        \item
        \(f: (0, 1) \to \mathbb{R}\) is extendable to \(S^1\) iff \(\lim_{x \to 0^+} f(x) = \lim_{x \to 1^-} f(x)\).

        \item
        \(f(x)\) can be extended to \([0, 1] \implies\) Bounded.
        \begin{itemize}
            \item \(\frac{1}{x}\) defined on \((0, 1]\), not on \([0, 1]\). Not bounded.
            \item \(\sin \frac{1}{x}\) bounded, not extendable to \(0\).
            \item \(f\) extendable \(\iff \lim_{x \to 0^+} f(x), \lim_{x \to 1^-} f(x)\) exist (are finite).
        \end{itemize}

        \item
        \(\sin \frac{1}{x}\) can be extended to \(\overline{X_0}\).
        \[\gamma: (0, 1) \xrightarrow{h} \mathbb{R}^2 \xrightarrow{\pi_2} \mathbb{R}, \ x \longmapsto (x, \sin \frac{1}{x}) \longmapsto \sin \frac{1}{x}\]

        On \(Y_0 = \overline{X_0}\) (in \(\mathbb{R}^2\)), we have \((0, y) \xrightarrow{\pi_2} y\).
    
        The map \(F: Y_0 \to \mathbb{R}\), \((x, \sin \frac{1}{x}) \longmapsto \sin \frac{1}{x}\), \((0, y) \longmapsto y\), is continuous!
    
        (Note: \(x \mapsto \sin \frac{1}{x}\) is \(\pi_2 \circ h\), so continuous.)
    \end{enumerate}
\end{example}

\subsection{Stone-\v{C}ech Compactification Construction}

\begin{theorem}
    Let \(X\) be completely regular. There exists a compactification \(Y\) of \(X\) such that \textit{every} bounded continuous map \(f: X \to \mathbb{R}\) extends uniquely to \(\hat{f}: Y \to \mathbb{R}\). (\(Y\) is called the \textbf{Stone-\v{C}ech compactification} of \(X\).)
\end{theorem}

\begin{remark}
    Uniqueness will be discussed next time (26/1/5).
\end{remark}

\begin{remark}
Proof is commented out!
\end{remark}

% \begin{proof}
%     Let \(\{f_\alpha\}_{\alpha \in J}\) be the collection of \textit{all} bounded continuous functions on \(X\), \(J\) indexing set.
%     For each \(\alpha \in J\), \(f_\alpha: X \to [\inf f_\alpha, \sup f_\alpha] = I_\alpha\).

%     Define \(h: X \to \prod_{\alpha \in J} I_\alpha\), \(h(x) = (f_\alpha(x))_{\alpha \in J}\) (which is compact by Tychonoff thm!).

%     Since \(X\) is completely regular, \(\{f_\alpha\}\) separates points from closed sets (and points from points), and hence \(h\) is an embedding (Embedding Theorem).
    
%     Let \(Y\) be the compactification obtained from this embedding
%     \[H: Y \to \prod_{\alpha \in J} I_\alpha \ (H(x) = h(x), \ \forall x \in X)\]

%     Take any bounded function \(f\) on \(X\), \(\exists \beta \in J: f = f_\beta\).
%     \[Y \xrightarrow{H} \prod_{\alpha \in J} I_\alpha \xrightarrow{\pi_\beta} I_\beta\]

%     So \(\hat{f} = \hat{f}_\beta = \pi_\beta \circ H\) continuous.
% \end{proof}

\newpage

\end{document}
